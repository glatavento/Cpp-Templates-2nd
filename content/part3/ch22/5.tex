\section{可选桥接}
FunctionPtr模板可以直接替代函数指针,但还不支持函数指针提供的一个操作:测试两个FunctionPtr对象是否会调用同一个函数。添加这样的操作需要用等号操作更新FunctorBridge:

\begin{cpp}
virtual bool equals(FunctorBridge const* fb) const = 0;
\end{cpp}

以及SpecificFunctorBridge中的实现,当具有相同类型时比较存储的函数对象:

\begin{cpp}
virtual bool equals(FunctorBridge<R, Args...> const* fb) const override {
	if (auto specFb = dynamic_cast<SpecificFunctorBridge const*>(fb)) {
		return functor == specFb->functor;
	}
	// functors with different types are never equal:
	return false;
}
\end{cpp}

最后,为FunctionPtr实现operator==它首先检查空函子,然后委托给FunctorBridge:

\begin{cpp}
friend bool
operator==(FunctionPtr const& f1, FunctionPtr const& f2) {
	if (!f1 || !f2) {
		return !f1 && !f2;
	}
	return f1.bridge->equals(f2.bridge);
}
friend bool
operator!=(FunctionPtr const& f1, FunctionPtr const& f2) {
	return !(f1 == f2);
}
\end{cpp}

这个实现正确,但它有一个缺点:若给FunctionPtr赋值或初始化函数对象没有合适的operator==(例如,包含Lambda),程序将无法编译。因为FunctionPtrsd还没有使用operator==,而且许多其他类模板(如std::vector)可以用没有operator==的类型实例化,只要不使用operator==就好。

operator==的问题由类型擦除导致,因为当对FunctionPtr赋值或初始化,就会丢失函数对象的类型。因此需要在赋值或初始化完成前,捕获关于该类型所需的所有信息。这些信息包括对函数对象的operator==的调用,所以不能确定什么时候需要使用该操作符。

\begin{notice}
机制上,调用operator==的代码会实例化,因为类模板的所有虚函数(本例中是SpecificFunctorBridge)通常在类模板本身实例化时实例化。
\end{notice}

这里可以使用基于SFINAE的特征(第19.4节中讨论),在调用operator==之前,使用一个复杂的特征查询是否可用:

\filename{bridge/isequalitycomparable.hpp}
\begin{cpp}
#include <utility> // for declval()
#include <type_traits> // for true_type and false_type

template<typename T>
class IsEqualityComparable {
	private:
	// test convertibility of == and ! == to bool:
	static void* conv(bool); // to check convertibility to bool
	template<typename U>
	static std::true_type test(decltype(conv(std::declval<U const&>() ==
											std::declval<U const&>())),
							decltype(conv(!(std::declval<U const&>() ==
											std::declval<U const&>())))
							);
	
	// fallback:
	template<typename U>
	static std::false_type test(...);
	public:
	static constexpr bool value = decltype(test<T>(nullptr,
													nullptr))::value;
};
\end{cpp}

IsEqualityComparable特征应用了表达式测试特征的形式,如第19.4.1节所介绍的:两个test()重载,其中一个包含用decltype包装的测试表达式,另一个通过省略号接受任意参数。第一个test()函数尝试使用==比较两个T类型的const对象,并确保结果可以隐式转换为bool(对于第一个参数)并传递给逻辑否定操作符operator!,结果可转换为bool。如果两个操作都定义良好,则参数类型都为void*。

使用IsEqualityComparable特征,可以构造TryEquals类模板,对给定类型调用==(当该类型可用时),或者在没有合适的==存在时抛出异常:

\filename{bridge/tryequals.hpp}
\begin{cpp}
#include <exception>
#include "isequalitycomparable.hpp"

template<typename T,
bool EqComparable = IsEqualityComparable<T>::value>
struct TryEquals
{
	static bool equals(T const& x1, T const& x2) {
		return x1 == x2;
	}
};

class NotEqualityComparable : public std::exception {};

template<typename T>
struct TryEquals<T, false> {
	static bool equals(T const& x1, T const& x2) {
		throw NotEqualityComparable();
	}
};
\end{cpp}

最后,通过在SpecificFunctorBridge的实现中使用TryEquals,当存储的函数对象类型匹配且函数对象支持==操作时,就能够在FunctionPtr中提供==操作:

\begin{cpp}
virtual bool equals(FunctorBridge<R, Args...> const* fb) const override {
	if (auto specFb = dynamic_cast<SpecificFunctorBridge const*>(fb)) {
		return TryEquals<Functor>::equals(functor, specFb->functor);
	}
	// functors with different types are never equal:
	return false;
}
\end{cpp}




























