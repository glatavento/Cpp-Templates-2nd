\chapter{特征的实现}
模板能够对各种类型的类和函数进行参数化,引入尽可能多的模板参数以实现对类型或算法的定制。通过这种方式,“模板化”组件可以实例化,以满足外部代码的需求。然而,引入几十个模板参数来实现最大参数化其实没有必要。必须在外部代码中指定所有相应的参数也非常繁琐,而且模板参数会使组件与其外部代码之间的关系复杂化。

我们引入的大多数参数都有默认值,参数可以由几个主要参数决定,这些参数可以省略。也可以给其他参数提供依赖于主要参数的默认值,这可以满足大多数情况,但默认值偶尔需要重写(对于特殊应用程序)。然而,其他参数与主要参数无关:除了存在几乎符合要求的默认值外,其本身就是主要参数。

特征(或特征模板)是C++编程工具,极大地促进了工业化模板设计中出现的那种参数管理。本章中,将证明这种模板是有用的,并演示各种技术,这些技术能够编写更健壮和强大的工具。

这里提供的大多数特性都可以在C++标准库中,以某种形式使用。为了简单起见,我们给出简化的实现,其中省略了工业化实现(如标准库的那些)中的一些细节。出于这个原因,我们还使用了自己的命名方案,其很容易与标准特征对应。
\subfile{ch19/1.tex}
\subfile{ch19/2.tex}
\subfile{ch19/3.tex}
\subfile{ch19/4.tex}
\subfile{ch19/5.tex}
\subfile{ch19/6.tex}
\subfile{ch19/7.tex}
\subfile{ch19/8.tex}
\subfile{ch19/9.tex}
\subfile{ch19/10.tex}
\subfile{ch19/11.tex}