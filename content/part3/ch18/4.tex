\section{使用概念}
反对模板静态多态的一个论点是,接口的绑定通过实例化相应的模板来完成,没有公共接口(类)用来编程。若所有实例化的代码都有效,那么模板就可以工作。若不是,可能会导致难以理解的错误消息,甚至导致有效但意外的行为。

由于这个原因,C++语言设计人员致力于为模板参数显式提供(和检查)接口的能力。这样的接口在C++中称为概念,表示模板参数必须满足一组约束后,才能成功实例化模板。

尽管各路开发者这些年在这个领域做了许多工作,但是概念直到C++17,才成为标准C++的一部分。一些编译器提供了对这种特性的实验性支持,但这些概念可能会成为C++17之后标准的一部分。

\begin{notice}
例如,GCC 7提供了-fconcepts选项。
\end{notice}

概念可以理解为静态多态的一种“接口”:

\filename{poly/conceptsreq.hpp}
\begin{cpp}
#include "coord.hpp"
template<typename T>
concept GeoObj = requires(T x) {
	{ x.draw() } -> void;
	{ x.center_of_gravity() } -> Coord;
	...
};
\end{cpp}

这里,使用关键字概念来定义概念GeoObj,约束类型具有可调用成员draw()和center\_of\_gravity(),并具有适当的结果类型。

现在,可以重写一些示例模板,以包含一个require子句,用GeoObj概念约束模板参数:

\filename{poly/conceptspoly.hpp}
\begin{cpp}
#include "conceptsreq.hpp"
#include <vector>

// draw any GeoObj
template<typename T>
requires GeoObj<T>
void myDraw (T const& obj) {
	obj.draw(); // call draw() according to type of object
}

// compute distance of center of gravity between two GeoObjs
template<typename T1, typename T2>
requires GeoObj<T1> && GeoObj<T2>
Coord distance (T1 const& x1, T2 const& x2) {
	Coord c = x1.center_of_gravity() - x2.center_of_gravity();
	return c.abs(); // return coordinates as absolute values
}

// draw homogeneous collection of GeoObjs
template<typename T>
requires GeoObj<T>
void drawElems (std::vector<T> const& elems) {
	for (std::size_type i=0; i<elems.size(); ++i) {
		elems[i].draw(); // call draw() according to type of element
	}
}
\end{cpp}

对于能够参与(静态)多态行为的类型来说,这种方法属于非侵入:

\begin{cpp}
// concrete geometric object class Circle
// - not derived from any class or implementing any interface
class Circle {
	public:
	void draw() const;
	Coord center_of_gravity() const;
	...
};
\end{cpp}

这些类型仍然定义在没有特定基类或需求子句的情况下,并且仍然可以是基本类型,或源自独立框架的类型。

附录E包含了对C++概念的更详细的讨论,这些概念将在下一个C++标准中出现。



















