\section{其他方法}

最后,介绍并讨论一些定义特征的其他方法。

\subsection{If-Then-Else}

上一节中,根据另一种类型特征HasPlusT的结果,PlusResultT特征的最终定义是一个完全不同的实现。可以用一个特殊的类型模板IfThenElse来表达这个if-else行为,其接受一个布尔型非类型模板参数,可以从两个类型参数中选择一个:

\filename{traits/ifthenelse.hpp}
\begin{cpp}
#ifndef IFTHENELSE_HPP
#define IFTHENELSE_HPP

// primary template: yield the second argument by default and rely on
// a partial specialization to yield the third argument
// if COND is false
template<bool COND, typename TrueType, typename FalseType>
struct IfThenElseT {
	using Type = TrueType;
};

// partial specialization: false yields third argument
template<typename TrueType, typename FalseType>
struct IfThenElseT<false, TrueType, FalseType> {
	using Type = FalseType;
};

template<bool COND, typename TrueType, typename FalseType>
using IfThenElse = typename IfThenElseT<COND, TrueType, FalseType>::Type;
#endif // IFTHENELSE_HPP
\end{cpp}

下面的示例,通过定义一个类型函数来确定给定值的最低整数类型,来演示此模板的应用:

\filename{traits/smallestint.hpp}
\begin{cpp}
#include <limits>
#include "ifthenelse.hpp"
template<auto N>
struct SmallestIntT {
	using Type =
	typename IfThenElseT<N <= std::numeric_limits<char>::max(), char,
	typename IfThenElseT<N <= std::numeric_limits<short>::max(), short,
	typename IfThenElseT<N <= std::numeric_limits<int>::max(), int,
	typename IfThenElseT<N <= std::numeric_limits<long>::max(), long,
	typename IfThenElseT<N <= std::numeric_limits<long long>::max(),
	long long, // then
	void // fallback
	>::Type
	>::Type
	>::Type
	>::Type
	>::Type;
};
\end{cpp}

,与普通C++的if-then-else不同,"then"和"else"分支的模板参数在选择前求值,因此两个分支不可能包含格式错误的代码,或者格式错误。考虑一个特征,为给定的有符号类型生成相应的无符号类型。有一个标准特征std::make\_unsigned,可以执行这个转换,但是要求传递的类型必须是有符号的整型,并且不能是bool类型;否则,将导致未定义行为(参见第D.4节)。可以实现特征来产生相应的无符号类型,否则直接传递的类型(若给出了不恰当的类型,就可以避免了未定义行为)可能是一个好主意。简单的实现不起任何作用:

\begin{cpp}
// ERROR: undefined behavior if T is bool or no integral type:
template<typename T>
struct UnsignedT {
	using Type = IfThenElse<std::is_integral<T>::value
		&& !std::is_same<T,bool>::value,
	typename std::make_unsigned<T>::type,
	T>;
};
\end{cpp}

UnsignedT<bool>的实例化仍具有未定义行为,因为编译器仍然会试图对这个类型进行转换

\begin{cpp}
typename std::make_unsigned<T>::type
\end{cpp}

为了解决这个问题,需要添加间接层,以便将IfThenElse参数本身作为封装结果的类型函数使用:

\begin{cpp}
// yield T when using member Type:
template<typename T>
struct IdentityT {
	using Type = T;
};

// to make unsigned after IfThenElse was evaluated:
template<typename T>
struct MakeUnsignedT {
	using Type = typename std::make_unsigned<T>::type;
};

template<typename T>
struct UnsignedT {
	using Type = typename IfThenElse<std::is_integral<T>::value
									&& !std::is_same<T,bool>::value,
									MakeUnsignedT<T>,
									IdentityT<T>
									>::Type;
};
\end{cpp}

UnsignedT这个定义中,IfThenElse的类型参数都是类型函数本身的实例。但在IfThenElse选择一个类型函数前,类型函数不会进行实际计算。不过,IfThenElse选择类型函数实例(MakeUnsignedT或IdentityT)。然后,::Type计算选定的Type函数实例以生成Type。

这完全依赖于IfThenElse构造中,未选择的包装器类型从未完全实例化。特别地,以下代码无效:

\begin{cpp}
template<typename T>
struct UnsignedT {
	using Type = typename IfThenElse<std::is_integral<T>::value
		&& !std::is_same<T,bool>::value,
		MakeUnsignedT<T>::Type,
		T
	>::Type;
};
\end{cpp}

必须在以后为MakeUnsignedT<T>添加::Type,还需要IdentityT助手在else分支中为T添加::Type。

所以,不能这样使用

\begin{cpp}
template<typename T>
	using Identity = typename IdentityT<T>::Type;
\end{cpp}

可以声明这样一个别名模板,可能在其他地方有用,但是不能在IfThenElse的定义中使用。因为Identity<T>会立即导致IdentityT<T>的完整实例化,从而会检测其Type成员。

IfThenElseT模板在C++标准库中可用,格式为std::conditional<>(参见第D.5节)。UnsignedT特征将有如下定义:

\begin{cpp}
template<typename T>
struct UnsignedT {
	using Type
	= typename std::conditional_t<std::is_integral<T>::value
								&& !std::is_same<T,bool>::value,
								MakeUnsignedT<T>,
								IdentityT<T>
								>::Type;
};
\end{cpp}

\subsection{检测非抛出操作}

确定某个特定操作是否会引发异常有时有用。例如,移动构造函数应该标记为noexcept,表明在任何情况下都不会抛出异常。然而,特定类的移动构造函数是否抛出异常,通常取决于其成员和基类的移动构造函数是否抛出异常。考虑一个简单的类模板Pair的移动构造函数:

\begin{cpp}
template<typename T1, typename T2>
class Pair {
	T1 first;
	T2 second;
	public:
	Pair(Pair&& other)
	: first(std::forward<T1>(other.first)),
	second(std::forward<T2>(other.second)) {
	}
};
\end{cpp}

当T1或T2的移动操作可以抛出异常时,Pair的移动构造函数可以抛出异常。给定一个特征IsNothrowMoveConstructibleT,可以通过在Pair的移动构造函数中使用计算出来的noexcept异常规范来表示这个属性:

\begin{cpp}
Pair(Pair&& other) noexcept(IsNothrowMoveConstructibleT<T1>::value &&
		 IsNothrowMoveConstructibleT<T2>::value)
: first(std::forward<T1>(other.first)),
second(std::forward<T2>(other.second)) { }
\end{cpp}

剩下的工作就是实现IsNothrowMoveConstructibleT特征。可以使用noexcept操作符直接实现此特征,确定给定表达式是否不抛出异常:

\filename{traits/isnothrowmoveconstructible1.hpp}
\begin{cpp}
#include <utility> // for declval
#include <type_traits> // for bool_constant

template<typename T>
struct IsNothrowMoveConstructibleT
: std::bool_constant<noexcept(T(std::declval<T>()))> {};
\end{cpp}

这里使用了操作符版本的noexcept,用于确定表达式不抛出异常。因为结果是一个布尔值,可以直接传递它来定义基类std::bool\_constant<>,可用于定义std::true\_type和std::false\_type(参见19.3.3节)。

\begin{notice}
C++11和C++14中,必须将基类指定为std::integral\_constant<bool,...>,而非std::bool\_constant<...>。
\end{notice}

因为不是SFINAE友好的(参见19.4.4节),所以这个实现应该改进:若特征实例化的类型没有可用的移动或复制构造函数——使得表达式T(std::declval<T\&\&>())无效——整个程序呈现病态:

\begin{cpp}
lass E {
	public:
	E(E&&) = delete;
};
...
std::cout << IsNothrowMoveConstructibleT<E>::value; // compile-time ERROR
\end{cpp}

类型特征应该产生false值,而不是终止编译。

必须在计算结果的表达式求值之前检查其是否有效,要先弄清楚移动构造是否有效,然后再检查是否有例外。因此,修改了特征的第一个版本,添加了一个默认为void的模板参数参和一个使用std::void\_t的局部参数,该参数只有当移动构造有效时才有效:

\filename{traits/isnothrowmoveconstructible2.hpp}
\begin{cpp}
#include <utility> // for declval
#include <type_traits> // for true_type, false_type, and bool_constant<>

// primary template:
template<typename T, typename = std::void_t<>>
struct IsNothrowMoveConstructibleT : std::false_type
{};

// partial specialization (may be SFINAE'd away):
template<typename T>
struct IsNothrowMoveConstructibleT
		<T, std::void_t<decltype(T(std::declval<T>()))>>
: std::bool_constant<noexcept(T(std::declval<T>()))> {};
\end{cpp}

若替换偏特化中的std::void\_t<...>有效,特化版本匹配,基类说明符中的noexcept(...)表达式可以安全地求值。否则,非实例化偏特化就会丢弃,而会实例化主模板(产生std::false\_type结果)。

若不能直接调用移动构造函数,就无法检查是否会抛出异常。移动构造函数是public,且不删除(不够的),还要求对应的类型不是抽象类(对抽象类的引用或指针都可以)。所以,类型特征命名为IsNothrowMoveConstructible,而不是HasNothrowMoveConstructor。对于其他情况,需要编译器的支持。 

C++标准库提供了相应的类型特征std::is\_move\_constructible<>,在第D.3.2节中描述。

\subsection{特征的便利性}

对于类型特征的常见抱怨是它们非常冗长,因为类型特征的每次使用通常需要尾随::type,并在上下文中需要前置typename关键字,两者都是样板。当组合多个类型特征时,这可能会导致一些尴尬的格式化,比如在运行的数组加法操作符的例子中,若正确地实现它,确保没有返回常量或引用类型:

\begin{cpp}
template<typename T1, typename T2>
Array<
	typename RemoveCVT<
		typename RemoveReferenceT<
			typename PlusResultT<T1, T2>::Type
		>::Type
	>::Type
>
operator+ (Array<T1> const&, Array<T2> const&);
\end{cpp}

通过使用别名模板和变量模板,可以更方便地使用特征,分别生成类型或值。但这些快捷方式有时可能不可用的,这时必须使用原始的类模板。我们已经在MemberPointerToIntT示例中讨论过这样的情况,但接下来将进行更通用的讨论。

\subsubsection{别名模板和特征}

别名模板提供了一种减少代码冗长的方法。可以直接使用别名模板,而不是将类型特征表示为具有类型成员type的类模板。例如,以下三个别名模板包装了上面使用的类型特征:

\begin{cpp}
template<typename T>
using RemoveCV = typename RemoveCVT<T>::Type;

template<typename T>
using RemoveReference = typename RemoveReferenceT<T>::Type;

template<typename T1, typename T2>
using PlusResult = typename PlusResultT<T1, T2>::Type;
\end{cpp}

有了这些别名模板,可以将加法操作符简化为

\begin{cpp}
template<typename T1, typename T2>
Array<RemoveCV<RemoveReference<PlusResult<T1, T2>>>>
operator+ (Array<T1> const&, Array<T2> const&);
\end{cpp}

第二个版本显然更短,使特征的组成更明显。这样的改进使别名模板更适合于类型特征的某些使用。

然而,对类型特征使用别名模板也有缺点:

\begin{enumerate}
\item
别名模板不能特化(如第16.3节所述),由于许多写入特征的技术都依赖特化,别名模板无论如何都需要重定向到类模板。

\item 
有些特征是由用户特化的,例如描述特定加法操作是否可交换的特征,当大多数使用涉及别名模板时,特化类模板可能会令人困惑。

\item
别名模板的使用总是会实例化该类型(例如,底层的类模板特化),这使得避免实例化对给定类型没有意义的特征变得更加困难(如19.7.1节所讨论的)。
\end{enumerate}

最后一点的另一种表达方式是,别名模板不能与元函数转发一起使用(参见19.3.2节)。

为类型特征使用别名模板既有积极的一面,也有消极的一面,所以建议像本节中,以及在C++标准库中那样使用它们:为两个类模板提供特定的命名约定(我们选择了T后缀和type类型成员),为别名模板提供略有不同的命名约定(我们去掉了T后缀),并根据底层类模板定义每个别名模板。通过这种方式,可以通过别名模板提供更清晰代码,但为了更高级的使用,也可以返回到类模板。

由于历史原因,C++标准库有不同的约定。类型特征类模板产生类型中的类型,并且没有特定的后缀(很多是在C++11中引入的)。因为没有后缀的名称已经标准化(参见D.1节),相应的别名模板(直接产生类型)在C++14中引入,并赋予了\_t后缀。

\subsubsection{可变参模板和特征}

返回值的特征需要添加::value(或类似的成员选择)来产生特征的结果。constexpr变量模板(在第5.6节中介绍)提供了一种减少这种冗长的方法。

下面的变量模板包装了第19.3.3节定义的IsSameT特征和第19.5节定义的IsConvertibleT特征:

\begin{cpp}
template<typename T1, typename T2>
	constexpr bool IsSame = IsSameT<T1,T2>::value;
	
template<typename FROM, typename TO>
	constexpr bool IsConvertible = IsConvertibleT<FROM, TO>::value;
\end{cpp}

可以将代码简化为

\begin{cpp}
if (IsSame<T,int> || IsConvertible<T,char>) ...
\end{cpp}

从而替代原始方式

\begin{cpp}
if (IsSameT<T,int>::value || IsConvertibleT<T,char>::value) ...
\end{cpp}

由于历史原因,C++标准库有不同的约定。产生值结果的特征类模板没有特定的后缀,它们中的许多在C++11标准中引入。直接产生结果值的相应变量模板在C++17中使用了\_v后缀(参见D.1节)。

\begin{notice}
C++标准化委员会受到长期存在的传统约束,即所有标准名称都由小写字符和可选的下划线分隔。也就是说,像isSame或isSame这样的名称不太可能加入标准中(使用这种拼写风格的概念除外)。
\end{notice}




