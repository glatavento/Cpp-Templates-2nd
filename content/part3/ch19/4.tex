\section{基于SFINAE的特征}

SFINAE原则(替换失败不为过;参见第8.4节和第15.7节)将模板参数推导过程中无效类型和表达式形成过程中的错误(这会导致程序格式不正确)转化为推导失败,从而允许重载解析选择不同的候选项。虽然SFINAE最初的目的是避免函数模板重载的假失败,但它还支持显著的编译时技术,可以确定特定类型或表达式是否有效。这可以通过编写特征,来确定类型是否具有特定成员、是否支持特定操作或是否为类。

实现基于SFINAE特性的两种主要方法:SFINAE排除函数重载和SFINAE排除偏特化。

\subsection{去除函数重载}

对基于SFINAE的特征的第一次尝试说明,使用带有函数重载的SFINAE来确定类型是否默认可构造的基本技术,这样就可以创建不需要初始化值的对象。对于给定的类型T,像T()这样的表达式必须有效。

实现如下所示:

\filename{traits/isdefaultconstructible1.hpp}
\begin{cpp}
#include "issame.hpp"

template<typename T>
struct IsDefaultConstructibleT {
	private:
	// test() trying substitute call of a default constructor for T passed as U:
	template<typename U, typename = decltype(U())>
		static char test(void*);
	// test() fallback:
	template<typename>
		static long test(...);
	public:
	static constexpr bool value
		= IsSameT<decltype(test<T>(nullptr)), char>::value;
};
\end{cpp}

使用函数重载实现基于SFINAE的特征的通常方法是声明两个名为test()的重载函数模板,具有不同的返回类型:

\begin{cpp}
template<...> static char test(void*);
template<...> static long test(...);
\end{cpp}

第一个重载设计为只在请求的检查成功时匹配(将在下面讨论如何实现这一点)。第二个重载是回退:

\begin{notice}
回退声明有时可以是普通成员函数声明,而不是成员函数模板。
\end{notice}

它总是匹配调用,但是因为它匹配“带有省略号”(即可变参数),任何其他匹配都将首选(参见C.2节)。

“返回值”的值取决于选择了哪个test()成员的重载:

\begin{cpp}
static constexpr bool value
	= IsSameT<decltype(test<T>(nullptr)), char>::value;
\end{cpp}

如果选择了返回类型为char的第一个test()成员,value将初始化为isSame<char,char>,为true。否则,将初始化为isSame<long,char>,该值为false。

现在,需要处理想要测试的特定属性。目标是使第一个test()重载,在检查条件适用时有效。本例中,想知道是否可以默认构造一个传递类型T的对象。为了实现这一点,将T传递为U,并为test()的第一个声明提供第二个未命名(虚拟)模板参数,该实参初始化时且转换有效时,参数有效。使用的表达式只有在隐式或显式默认构造函数存在时,U()才有效。表达式使用decltype,使其成为初始化类型参数的有效表达式。不能推导第二个模板参数,因为没有传递相应的参数。不会提供显式的模板参数。因此会进行替换,如果替换失败,根据SFINAE,test()的声明将丢弃,以便匹配回退声明。

因此,可以这样使用该特征:

\begin{cpp}
IsDefaultConstructibleT<int>::value // yields true

struct S {
	S() = delete;
};
IsDefaultConstructibleT<S>::value // yields false
\end{cpp}

注意,不能在第一个test()中直接使用模板参数T:

\begin{cpp}
template<typename T>
struct IsDefaultConstructibleT {
	private:
	// ERROR: test() uses T directly:
	template<typename, typename = decltype(T())>
		static char test(void*);
	// test() fallback:
	template<typename>
		static long test(...);
	public:
	static constexpr bool value
		= IsSameT<decltype(test<T>(nullptr)), char>::value;
};
\end{cpp}

这样不行,对于任何T替换所有成员函数,因此对于不是默认可构造的类型,代码无法编译,而不忽略第一个test()重载。通过将类模板参数T传递给函数模板参数U,仅为第二次test()重载创建了特定的SFINAE。

\subsubsection{基于SFINAE特征的替代实现策略}

1998年发布第一个C++标准前,基于SFINAE的特性就已经可以实现了。

\begin{notice}
SFINAE规则当时有更多的限制:当替换模板参数导致了一个不规范的类型构造(例如,T::X,其中T是int), SFINAE按预期工作,导致了一个无效的表达式(例如,sizeof(f()),其中f()返回void),SFINAE不起作用,并立即发出错误信息。
\end{notice}

这种方法的关键在于声明两个重载的函数模板,返回不同的返回类型:

\begin{cpp}
template<...> static char test(void*);
template<...> static long test(...);
\end{cpp}

然而,最初发布的技术使用返回类型的大小来确定选择哪个重载(因为无法使用nullptr和constexpr,所以使用0和enum):

\begin{notice}
本书的第一版可能是这种技术的首次亮相。
\end{notice}

\begin{cpp}
enum { value = sizeof(test<...>(0)) == 1 };
\end{cpp}

某些平台上,sizeof(char)==sizeof(long)可能为true。例如,在数字信号处理器(DSP)或旧的Cray机器上,所有的整型基本类型可以具有相同的大小。根据定义,sizeof(char)等于1,在这些机器上,sizeof(long)甚至sizeof(long long)也等于1。

鉴于上述观察结果,希望确保test()函数的返回类型在所有平台上具有不同的大小。例如,定义之后

\begin{cpp}
using Size1T = char;
using Size2T = struct { char a[2]; };
\end{cpp}

或

\begin{cpp}
using Size1T = char(&)[1];
using Size2T = char(&)[2];
\end{cpp}

可以这样定义test()的重载:

\begin{cpp}
template<...> static Size1T test(void*); // checking test()
template<...> static Size2T test(...); // fallback
\end{cpp}

这里,要么返回Size1T,一个大小为1的单个字符,要么返回(一个结构体)一个包含两个字符的数组,该数组在所有平台上的大小至少为2。

使用其中一种方法的代码仍然很常见。

注意,传递给func()的调用参数的类型并不重要。重要的是传递的参数与期望的类型匹配。例如,可以定义传递整数42:

\begin{cpp}
template<...> static Size1T test(int); // checking test()
template<...> static Size2T test(...); // fallback
...
enum { value = sizeof(test<...>(42)) == 1 };
\end{cpp}

\subsubsection{使基于SFINAE的特征成为特征}

如第19.3.3节所述,返回布尔值的谓词特征应该返回一个派生自std::true\_type或std::false\_type的值。这样,也可以解决在某些平台上sizeof(char)与sizeof(long)相同的问题。

为此,需要IsDefaultConstructibleT的间接定义。特征本身应该从助手类的Type派生,助手类生成必要的基类。这里,可以简单地提供相应的基类作为test()重载的返回类型:

\begin{cpp}
template<...> static std::true_type test(void*); // checking test()
template<...> static std::false_type test(...); // fallback
\end{cpp}

基类的Type成员可以声明为:

\begin{cpp}
using Type = decltype(test<FROM>(nullptr));
\end{cpp}

并且,不再需要IsSameT特性。

因此,IsDefaultConstructibleT的完整改进实现如下所示:

\filename{traits/isdefaultconstructible2.hpp}
\begin{cpp}
#include <type_traits>

template<typename T>
struct IsDefaultConstructibleHelper {
	private:
	// test() trying substitute call of a default constructor for T passed as U:
	template<typename U, typename = decltype(U())>
		static std::true_type test(void*);
	// test() fallback:
	template<typename>
		static std::false_type test(...);
	public:
	using Type = decltype(test<T>(nullptr));
};

template<typename T>
struct IsDefaultConstructibleT : IsDefaultConstructibleHelper<T>::Type {
};
\end{cpp}

若第一个test()函数模板有效,是首选重载,因此成员IsDefaultConstructibleHelper::Type由它的返回类型std::true\_type初始化。因此,IsDefaultConstructibleT<…>派生于std::true\_type。

若第一个test()函数模板无效,会因为SFINAE而禁用,IsDefaultConstructibleHelper::Type由test()回退的返回类型初始化,也就是std::false\_type。结果,IsDefaultConstructibleT<…>派生自std::false\_type。

\subsection{去除偏特化}

实现基于SFINAE的特征的第二种方法使用偏特化,可以使用这个例子来找出T类型是否默认可构造:

\filename{traits/isdefaultconstructible3.hpp}
\begin{cpp}
#include "issame.hpp"
#include <type_traits> // defines true_type and false_type

// helper to ignore any number of template parameters:
template<typename...> using VoidT = void;

// primary template:
template<typename, typename = VoidT<>>
struct IsDefaultConstructibleT : std::false_type
{
};

// partial specialization (may be SFINAE'd away):
template<typename T>
struct IsDefaultConstructibleT<T, VoidT<decltype(T())>> : std::true_type
{
};
\end{cpp}

与上面针对谓词特征的IsDefaultConstructibleT改进版本一样,定义了从std::false\_type派生的一般情况,默认情况下,假定类型没有默认构造函数。

这里有趣的特性是第二个模板参数,默认为助手类的VoidT类型,可以使用编译时类型构造的偏特化。

本例中,只需要一个构造:

\begin{cpp}
decltype(T())
\end{cpp}

再次检查T的默认构造函数是否有效。若对于特定的T构造无效,SFINAE会丢弃整个偏特化,返回主模板。否则,偏特化有效并且首选。

C++17中,C++标准库引入了一个类型特征std::void\_t<>,对应于这里引入的类型VoidT。C++17前,可以如上面那样自行定义,甚至可以在命名空间std中这样定义:

\begin{notice}
命名空间std中定义void\_t在形式上是无效的:不允许用户在命名空间std添加声明。实践中,当前的编译器没有强制这样的限制,其行为也不会意外(标准指出这样做会导致“未定义行为”,可以出现任何情况)。
\end{notice}

\begin{cpp}
#include <type_traits>

#ifndef __cpp_lib_void_t
namespace std {
	template<typename...> using void_t = void;
}
#endif
\end{cpp}

C++14开始,C++标准化委员会建议编译器和标准库通过定义一致同意的特性宏,来表明它们实现了标准的哪些部分。这不是标准一致性的要求,但是实现者通常会遵循这些建议来帮助其用户进行分辨。

\begin{notice}
撰写本文时,发现Microsoft Visual C++是个不幸的例外。
\end{notice}

宏\_\_cpp\_lib\_void\_t是用来指示标准库实现std::void\_t的宏,因此上面的代码以它为条件。

显然,这种定义类型特征的方法看起来比重载函数模板的第一种方法更简洁,但需要在模板参数声明中明确条件的能力。使用带有函数重载的类模板,使我们能够使用辅助函数或类型。

\subsection{使用Lambda}

无论使用哪种技术,总是需要一些样板代码来定义特征:重载和调用两个test()成员函数或实现多个偏特化。接下来,将展示在C++17中如何通过在泛型Lambda中指定要检查的条件,来最小化这个样板代码。

首先,引入一个由两个嵌套的泛型Lambda表达式构造的工具:

\filename{traits/isvalid.hpp}
\begin{cpp}
#include <utility>

// helper: checking validity of f(args...) for F f and Args... args:
template<typename F, typename... Args,
		typename = decltype(std::declval<F>()(std::declval<Args&&>()...))>
std::true_type isValidImpl(void*);

// fallback if helper SFINAE'd out:
template<typename F, typename... Args>
std::false_type isValidImpl(...);

// define a lambda that takes a lambda f and returns whether calling f with args is valid
inline constexpr
auto isValid = [](auto f) {
	return [](auto&&... args) {
		return decltype(isValidImpl<decltype(f),
		decltype(args)&&...
		>(nullptr)){};
	};
};

// helper template to represent a type as a value
template<typename T>
struct TypeT {
	using Type = T;
};

// helper to wrap a type as a value
template<typename T>
constexpr auto type = TypeT<T>{};

// helper to unwrap a wrapped type in unevaluated contexts
template<typename T>
T valueT(TypeT<T>); // no definition needed
\end{cpp}

从isValid的定义开始:一个constexpr变量,其类型是Lambda闭包类型。声明必须使用占位符类型(代码中是auto),因为C++没有办法直接表示闭包类型。C++17前,Lambda表达式不能出现在常量表达式中,这就是为什么这段代码只在C++17中有效。因为isValid有一个闭包类型,所以可以调用。但是返回的本身是Lambda闭包类型的对象,其由内部的Lambda表达式产生。

深入研究内部Lambda表达式的细节之前,先了解一下isValid的用法:

\begin{cpp}
constexpr auto isDefaultConstructible
	= isValid([](auto x) -> decltype((void)decltype(valueT(x))()) {
		});
\end{cpp}

已经知道isDefaultConstructible有一个Lambda闭包类型,是一个检查类型是否为默认构造的函数对象(将在后面的内容中看到原因)。换句话说,isValid是一个特征工厂:根据参数生成特征检查对象的组件。

类型辅助变量模板允许将类型表示为值。通过这种方式获得的值x可以通过decltype(valueT(x))返回原始类型,

\begin{notice}
这对简单的辅助模板是一种基本技术,其位于Boost.Hana中!
\end{notice}

这正是在上面传递给isValid的Lambda函数中所做的。若提取的类型不能默认构造,那么decltype(valueT(x))()就无效,要么会得到一个编译错误,再要么相关的声明会进行SFINAE(由于isValid定义的细节,我们将实现后一种效果)。

\begin{cpp}
isDefaultConstructible can be used as follows:
isDefaultConstructible(type<int>) // true (int is default-constructible)
isDefaultConstructible(type<int&>) // false (references are not default-constructible)
\end{cpp}

要查看所有部分是如何工作的,考虑isValid中的内部Lambda表达式的变化,isValid的参数f绑定到isDefaultConstructible定义中指定的泛型Lambda参数。通过在isValid的定义中执行替换,从而得到了等价代码:

\begin{notice}
这段代码在C++中是无效的,因为出于编译技术的原因,Lambda表达式不能直接出现在decltype操作数中,但其含义明确。
\end{notice}

\begin{cpp}
constexpr auto isDefaultConstructible
= [](auto&&... args) {
	return decltype(
	isValidImpl<
	decltype([](auto x)
	-> decltype((void)decltype(valueT(x))())),
	decltype(args)&&...
	>(nullptr)){};
};
\end{cpp}

回头看看上面isvalidmpl()的第一个声明,就会注意到包含了默认模板参数列表

\begin{cpp}
decltype(std::declval<F>()(std::declval<Args&&>()...))>
\end{cpp}

试图调用第一个模板参数类型的值,该参数是isDefaultConstructible定义中Lambda的闭包类型,并将参数类型的值(decltype(args)\&\&…)传递给isDefaultConstructible。由于Lambda中只有一个参数x,args的扩展只能有一个参数;上面的static\_assert示例中,该参数的类型为TypeT<int>或TypeT<int\&>。在TypeT<int\&>情况下,decltype(valueT(x))是int\&这使得decltype(valueT(x))()无效,因此在isValidImpl()的第一个声明中对默认模板参数的替换失败,并通过SFINAE去除。所以只剩下第二个声明(否则将是一个较小的匹配),其产生一个false\_type值。也就是,isDefaultConstructible在传递type<int\&>时产生false\_type。若传递了type<int>,替换不会失败,并且选择isValidImpl()的第一个声明,产生一个true\_type类型的值。要使SFINAE工作,替换必须在要替换模板的上下文中产生。要替换的模板是sValidImpl的第一个声明和传递给isValid的泛型Lambda的调用操作符。因此,要测试的构造必须出现在Lambda的返回类型中,而不是其主体中!

isDefaultConstructible特征与以前的特征实现有一点不同,其需要函数式调用,而不是指定模板参数。可以说是一种可读性更强的表示法,使用之前的方式可以得到:

\begin{cpp}
template<typename T>
using IsDefaultConstructibleT
	= decltype(isDefaultConstructible(std::declval<T>()));
\end{cpp}

由于这是一个传统的模板声明,只能出现在命名空间作用域中,而isDefaultConstructible的定义可以在块作用域中引入。

这种技术似乎并不引人注目,因为实现中涉及的表达式和使用风格都比以前的技术更复杂。然而,当isValid就位并理解,许多特征就可以通过一个声明实现。例如,测试是否访问名为first的成员,可以相当干净(见第19.6.4节的完整示例):

\begin{cpp}
constexpr auto hasFirst
	= isValid([](auto x) -> decltype((void)valueT(x).first) {
	});
\end{cpp}

\subsection{SFINAE友好的特征}

类型特征能够回应特定的查询,而不会导致程序病态。基于SFINAE的特征通过SFINAE捕捉来解决这个问题,将这些错误会转化为负面结果。

然而,目前出现的一些特征(如第19.3.4节中描述的PlusResultT特征)在出现错误时表现不佳。回想一下该节中对PlusResultT的定义:

\filename{traits/plus2.hpp}
\begin{cpp}
#include <utility>

template<typename T1, typename T2>
struct PlusResultT {
	using Type = decltype(std::declval<T1>() + std::declval<T2>());
};

template<typename T1, typename T2>
using PlusResult = typename PlusResultT<T1, T2>::Type;
\end{cpp}

这个定义中,加法用于SFINAE不保护的上下文中。因此,若程序试图对没有合适加法操作符的类型计算PlusResultT,对PlusResultT本身的计算将导致程序格式不正确,如下尝试声明不相关类型A和B的添加数组的返回类型:

\begin{notice}
简单起见,返回值只使用PlusResultT<T1,T2>::Type。实践中,还应该使用RemoveReferenceT<>和RemoveCVT<>来计算返回类型,以避免返回引用。
\end{notice}

\begin{cpp}
template<typename T>
class Array {
	...
};

// declare + for arrays of different element types:
template<typename T1, typename T2>
Array<typename PlusResultT<T1, T2>::Type>
operator+ (Array<T1> const&, Array<T2> const&);
\end{cpp}

若没有为数组元素定义相应的加法操作符,这里使用PlusResultT<>将出现错误。

\begin{cpp}
class A {
};
class B {
};
void addAB(Array<A> arrayA, Array<B> arrayB) {
	auto sum = arrayA + arrayB; // ERROR: fails in instantiation of PlusResultT<A, B>
	...
}
\end{cpp}

实际的问题不在于这种失败发生在明显格式不正确的代码中(没有办法将A数组添加到B数组中),而是发生在对加法操作无的模板参数推导期间,PlusResultT<A,B>的实例化。

这产生了一个后果:即使添加A和B数组时,也添加了特定的重载,程序也可能无法编译。因为C++没有指定若另一个重载会更好,函数模板中的类型是否可以实例化:

\begin{cpp}
// declare generic + for arrays of different element types:
template<typename T1, typename T2>
Array<typename PlusResultT<T1, T2>::Type>
operator+ (Array<T1> const&, Array<T2> const&);

// overload + for concrete types:
Array<A> operator+(Array<A> const& arrayA, Array<B> const& arrayB);

void addAB(Array<A> const& arrayA, Array<B> const& arrayB) {
	auto sum = arrayA + arrayB; // ERROR?: depends on whether the compiler
	... // instantiates PlusResultT<A,B>
}
\end{cpp}

若编译器不需要对加法操作符的第一个(模板)声明执行推导和替换,就能确定加法操作符的第二个声明是更好的匹配。

然而,在推导和替换函数模板候选时,类模板定义实例化期间发生的事情都不是函数模板替换的一部分,SFINAE不能避免的形成无效类型或表达式。不是仅丢弃函数模板候选,而是立即发出一个错误,因为我们试图对PlusResultT<>中的A和B类型的两个元素使用加法操作符:

\begin{cpp}
template<typename T1, typename T2>
struct PlusResultT {
	using Type = decltype(std::declval<T1>() + std::declval<T2>());
};
\end{cpp}

为了解决这个问题,必须使PlusResultT对SFINAE友好,通过一个合适的定义使它更有弹性,即使decltype表达式是病态的。

按照上一节描述的PlusResultT的例子,定义了一个HasPlusT特性,其可以检测给定类型是否有合适的加法操作:

\filename{traits/hasplus.hpp}
\begin{cpp}
#include <utility> // for declval
#include <type_traits> // for true_type, false_type, and void_t

// primary template:
template<typename, typename, typename = std::void_t<>>
struct HasPlusT : std::false_type
{
};

// partial specialization (may be SFINAE'd away):
template<typename T1, typename T2>
struct HasPlusT<T1, T2, std::void_t<decltype(std::declval<T1>()
						+ std::declval<T2>())>>
: std::true_type
{
};
\end{cpp}

若产生结果,PlusResultT可以使用现有的实现。否则,PlusResultT需要一个安全的默认值。对于一组模板参数没有任何意义的结果特征,最好的默认值是不提供成员Type。这样,若特征在SFINAE中使用——上面数组加法操作符模板的返回类型——缺失的成员Type将导致模板参数推导失败,这正是数组加法操作符模板所期望的行为。

下面的PlusResultT实现了这种行为:

\filename{traits/plus3.hpp}
\begin{cpp}
#include "hasplus.hpp"

template<typename T1, typename T2, bool = HasPlusT<T1, T2>::value>
struct PlusResultT { // primary template, used when HasPlusT yields true
	using Type = decltype(std::declval<T1>() + std::declval<T2>());
};

template<typename T1, typename T2>
struct PlusResultT<T1, T2, false> { // partial specialization, used otherwise
};
\end{cpp}

这个版本的PlusResultT中,添加了一个模板参数,该参数带有一个默认参数,用于确定前两个参数,是否像上面的HasPlusT特征所确定的那样支持加法。然后,偏特化PlusResultT获取参数的false,并且偏特化定义没有成员,从而避免了所提到的问题。对于支持加法的情况,默认参数的计算结果为true,并使用现有的Type成员定义选择主模板。因此,我们践行了约定,即PlusResultT只在加法操作定义良好的情况下,才提供结果类型。(注意,添加的参数永远不应该有显式的模板参数。)

再次考虑添加Array<A>和Array<B>的情况。在最新的PlusResultT模板实现中,PlusResultT<A,B>的实例化不会有Type成员,因为A和B的值不可求和。因此,数组加法操作符模板的结果类型无效,SFINAE将从考虑中剔除函数模板。因此,将选择特定于Array<A>和Array<B>的重载加法操作符。

作为一个通用的设计原则,若给出合理的模板参数作为输入,特征模板应该不会在实例化时失败。而一般的方法通常会进行两次检查:

\begin{itemize}
\item
减法操作是否有效

\item
检查计算结果
\end{itemize}

已经在PlusResultT中看到了,通过调用HasPlusT<>来确定在PlusResultImpl<>中加法操作符的调用是否有效。

将这一原则应用到19.3.1节介绍的ElementT:从容器类型生成元素类型。同样,因为答案依赖于(容器)类型,其有一个成员类型value\_type,所以只有当容器类型有这样一个value\_type成员时,主模板才尝试定义成员类型:

\begin{cpp}
template<typename C, bool = HasMemberT_value_type<C>::value>
struct ElementT {
	using Type = typename C::value_type;
};

template<typename C>
struct ElementT<C, false> {
};
\end{cpp}

使特征对SFINAE友好的第三个例子在第19.7.2节,其中IsNothrowMoveConstructibleT首先检查移动构造函数是否存在,然后检查其是否使用noexcept声明。













