\section{现代C++的元编程}

C++元编程技术随着时间的推移而发展(本章末尾的附注回顾了这一领域的里程碑),我们讨论并分类现代C++中常用的各种元编程方法。

\subsection{值元编程}

本书的第一版中,我们限制在最初的C++标准中引入的特性上(1998年发布,2003年进行了很小的修改)。那个世界里,编写简单的编译时(“元”)计算是一个挑战,我们在本章中花了大量的时间来研究这个问题;一个相当高级的示例,在编译时使用递归模板实例化计算整数值的平方根。正如8.2节中介绍的,在C++11中,尤其是C++14,通过引入constexpr函数,降低了挑战的难度。

\begin{notice}
C++11的constexpr功能足以解决许多常见的挑战,但编程模型并不总是令人满意的(例如,没有循环语句,所以迭代计算必须利用递归函数调用;参见23.2节)。C++14启用了循环语句和其他构造方式。
\end{notice}

C++14后,计算平方根的编译时函数很容易写成这样:

\filename{meta/sqrtconstexpr.hpp}
\begin{cpp}
template<typename T>
constexpr T sqrt(T x) {
	// handle cases where x and its square root are equal as a special case to simplify
	// the iteration criterion for larger x:
	if (x <= 1) {
		return x;
	}
	// repeatedly determine in which half of a [lo, hi] interval the square root of x is located,
	// until the interval is reduced to just one value:
	T lo = 0, hi = x;
	for (;;) {
		auto mid = (hi+lo)/2, midSquared = mid*mid;
		if (lo+1 >= hi || midSquared == x) {
			// mid must be the square root:
			return mid;
		}
		// continue with the higher/lower half-interval:
		if (midSquared < x) {
			lo = mid;
		}
		else {
			hi = mid;
		}
	}
}
\end{cpp}

该算法通过对已知包含x平方根的区间反复二分来寻找答案(根号0和1为特殊情况,以保持收敛准则简单)。这个sqrt()函数可以在编译或运行时求值:

\begin{cpp}
static_assert(sqrt(25) == 5, ""); // OK (evaluated at compile time)
static_assert(sqrt(40) == 6, ""); // OK (evaluated at compile time)

std::array<int, sqrt(40)+1> arr; // declares array of 7 elements (compile time)
long long l = 53478;
std::cout << sqrt(l) << '\n'; // prints 231 (evaluated at run time)
\end{cpp}

这个函数的实现在运行时可能不是最高效的(运行时利用机器的特性通常会有很好的性能回报),但因为执行的是编译时计算,所以绝对效率不如可移植性重要。在这个平方根示例中没有高级的“魔术模板”,只有函数模板的常见模板参数推导。这里的代码是“普通的C++”,读起来并不困难。

上面所做的是值元编程(即对编译时值的计算进行编程),但还有两种元编程可以用现代C++执行(比如C++14和C++17):类型元编程和混合元编程。

\subsection{类型元编程}

第19章讨论某些特征模板时,已经遇到了一种类型计算的形式,其以一个类型作为输入,并由此产生一个新类型。RemoveReferenceT类模板计算引用类型的基础类型,但在第19章中开发的示例只计算了相当基本的类型操作。通过依赖递归模板实例化(基于模板的元编程的支柱),可以执行相当复杂的类型计算。

看看下面的例子:

\filename{meta/removeallextents.hpp}
\begin{cpp}
// primary template: in general we yield the given type:
template<typename T>
struct RemoveAllExtentsT {
	using Type = T;
};

// partial specializations for array types (with and without bounds):
template<typename T, std::size_t SZ>
struct RemoveAllExtentsT<T[SZ]> {
	using Type = typename RemoveAllExtentsT<T>::Type;
};
template<typename T>
struct RemoveAllExtentsT<T[]> {
	using Type = typename RemoveAllExtentsT<T>::Type;
};

template<typename T>
using RemoveAllExtents = typename RemoveAllExtentsT<T>::Type;
\end{cpp}

RemoveAllExtents是一个类型元函数(即生成结果类型的计算设备),将从类型中移除任意数量的顶层“数组层”,可以这样使用它:

\begin{cpp}
RemoveAllExtents<int[]> // yields int
RemoveAllExtents<int[5][10]> // yields int
RemoveAllExtents<int[][10]> // yields int
RemoveAllExtents<int(*)[5]> // yields int(*)[5]
\end{cpp}

\begin{notice}
C++标准库提供了相应的类型特征std::remove\_all\_extents。详见第D.4节。
\end{notice}

元函数通过让匹配顶层数组情况的偏特化,递归地“调用”元函数本身来执行其任务。

若能够使用的值都是标量值,那么使用值进行计算将非常有限。不过,编程语言都有一个值容器构造,可以增强该语言的能力(而且大多数语言都有各种容器类型,如数组/vector、哈希表等)。类型元编程也是如此:添加“类型容器”构造大大增加了规程的适用性。现代C++包含了支持开发这种容器的机制。第24章开发了一个Typelist<...>类模板,就是这样的类型容器。

\subsection{混合元编程}

通过值和类型元编程,可以在编译时计算值和类型。但我们最终对运行时效果感兴趣,所以在需要类型和常量的地方在运行时代码中使用元编程。然而,元编程可以做的不止这些:可以在编译时以编程方式汇编代码位,并具有运行时效果。我们称之为混合元编程。

为了说明这一原理,先从一个简单的例子开始:计算两个std::array值的点积。回想一下,std::array是固定长度的容器模板,声明如下:

\begin{cpp}
namespace std {
	template<typename T, size_t N> struct array;
}
\end{cpp}

其中N是数组中类型为T的元素个数。给定两个相同数组类型的对象,其点积可以这样计算:

\begin{cpp}
template<typename T, std::size_t N>
auto dotProduct(std::array<T, N> const& x, std::array<T, N> const& y) {
	T result{};
	for (std::size_t k = 0; k<N; ++k) {
		result += x[k]*y[k];
	}
	return result;
}
\end{cpp}

直接编译for循环将产生分支指令,与直线执行相比,在某些机器上可能会造成一些开销

\begin{cpp}
result += x[0]*y[0];
result += x[1]*y[1];
result += x[2]*y[2];
result += x[3]*y[3];
...
\end{cpp}

现代编译器会将循环优化为对目标平台最有效的形式。为了便于讨论,可以以一种避免循环的方式重写dotProduct()实现:

\begin{notice}
这称为循环展开。我们通常不建议在可移植代码中显式展开循环,因为决定最佳展开策略的细节高度依赖于目标平台和循环体;编译器通常会更好地考虑这些因素。
\end{notice}

\begin{cpp}
template<typename T, std::size_t N>
struct DotProductT {
	static inline T result(T* a, T* b) {
		return *a * *b + DotProduct<T, N-1>::result(a+1,b+1);
	}
};

// partial specialization as end criteria
template<typename T>
struct DotProductT<T, 0> {
	static inline T result(T*, T*) {
		return T{};
	}
};

template<typename T, std::size_t N>
auto dotProduct(std::array<T, N> const& x,
				std::array<T, N> const& y) {
	return DotProductT<T, N>::result(x.begin(), y.begin());
}
\end{cpp}

这个新实现将工作委托给一个类模板DotProductT,使我们能够使用递归模板实例化和类模板偏特化来结束递归。DotProductT的每次实例化,会产生点积和数组其他组件的点积。对于类型std::array<T,N>,因此将有N个主模板实例和一个终止偏特化的实例。为了提高效率,编译器必须每次调用内联静态成员函数result()。即使启用了中等级别的编译器优化,这也是正确的。

\begin{notice}
我们在这里显式地指定inline关键字,是因为一些编译器(尤其是Clang)将此作为提示,以尝试更努力地进行内联调用。从语言的角度来看,这些函数隐式内联,因为它们在其外围类的体中定义。
\end{notice}

这段代码的主要特点是,混合了确定代码整体结构的编译时计算(通过递归模板实例化实现)和确定特定运行时效果的运行时计算(调用result())。

前面提到过,由于“类型容器”的可用性,类型元编程得到了极大的增强。混合元编程中,固定长度的数组类型很有用。尽管如此,混合元编程真正的“英雄容器”是元组。元组是一个值序列,每个值都有一个可选择的类型。C++标准库包含一个支持该概念的std::tuple类模板。例如,

\begin{cpp}
std::tuple<int, std::string, bool> tVal{42, "Answer", true};
\end{cpp}

定义了一个变量tVal,它聚合了三个类型的值:int,std::string和bool(按照特定的顺序)。由于类元组容器在现代C++编程中的重要性,我们将在第25章中详细开发一个容器。上面的tVal类型非常类似于一个简单的结构类型:

\begin{cpp}
struct MyTriple {
	int v1;
	std::string v2;
	bool v3;
};
\end{cpp}

鉴于在std::array和std::tuple中,有数组类型和(简单)结构类型的灵活对应,很自然地想知道简单联合类型的对应是否也对混合计算有用:答案是肯定的。C++标准库在C++17中为此引入了std::variant模板,我们在第26章开发了一个类似的组件。

因为与struct一样,std::tuple和std::variant都是异构类型,因此使用此类类型的混合元编程有时也称为异构元编程。

\subsection{单元类型的混合元编程}

另一个展示混合计算能力的例子是库,能够计算不同单元类型的值的结果。值计算在运行时执行,但结果元计算在编译时确定。

用一个高度简化的例子来说明,用它们占主单位的比例(分数)来表示单位。时间的主要单位是秒,则毫秒用比例1/1000表示,分钟用比例60/1表示。关键在于定义比率类型,其中每个值都有自己的类型:

\filename{meta/ratio.hpp}
\begin{cpp}
template<unsigned N, unsigned D = 1>
struct Ratio {
	static constexpr unsigned num = N; // numerator
	static constexpr unsigned den = D; // denominator
	using Type = Ratio<num, den>;
};
\end{cpp}

现在可以定义编译时计算,比如添加两个单元:

\filename{meta/ratioadd.hpp}
\begin{cpp}
// implementation of adding two ratios:
template<typename R1, typename R2>
struct RatioAddImpl
{
	private:
	static constexpr unsigned den = R1::den * R2::den;
	static constexpr unsigned num = R1::num * R2::den + R2::num * R1::den;
	public:
	typedef Ratio<num, den> Type;
};

// using declaration for convenient usage:
template<typename R1, typename R2>
using RatioAdd = typename RatioAddImpl<R1, R2>::Type;
\end{cpp}

在编译时计算两个比率的总和:

\begin{cpp}
using R1 = Ratio<1,1000>;
using R2 = Ratio<2,3>;
using RS = RatioAdd<R1,R2>; // RS has type Ratio<2003,2000>
std::cout << RS::num << '/' << RS::den << '\n'; // prints 2003/3000

using RA = RatioAdd<Ratio<2,3>,Ratio<5,7>>; // RA has type Ratio<29,21>
std::cout << RA::num << '/' << RA::den << '\n'; // prints 29/21
\end{cpp}

现在可以为duration定义一个类模板,参数化为任意值类型和一个单位类型,该单位类型是Ratio<>的实例:

\filename{meta/duration.hpp}
\begin{cpp}
// duration type for values of type T with unit type U:
template<typename T, typename U = Ratio<1>>
class Duration {
	public:
	using ValueType = T;
	using UnitType = typename U::Type;
	private:
	ValueType val;
	public:
	constexpr Duration(ValueType v = 0)
	: val(v) {
	}
	constexpr ValueType value() const {
		return val;
	}
};
\end{cpp}

有趣的是定义了一个加法操作符来加和两个duration:

\filename{meta/durationadd.hpp}
\begin{cpp}
// adding two durations where unit type might differ:
template<typename T1, typename U1, typename T2, typename U2>
auto constexpr operator+(Duration<T1, U1> const& lhs,
						 Duration<T2, U2> const& rhs) {
	// resulting type is a unit with 1 a nominator and
	// the resulting denominator of adding both unit type fractions
	using VT = Ratio<1,RatioAdd<U1,U2>::den>;
	// resulting value is the sum of both values
	// converted to the resulting unit type:
	auto val = lhs.value() * VT::den / U1::den * U1::num +
				rhs.value() * VT::den / U2::den * U2::num;
	return Duration<decltype(val), VT>(val);
}
\end{cpp}

允许参数有不同的单位类型,U1和U2。使用这些单位类型来计算结果持续时间,从而得到一个对应的单位分数(分子为1的分数)的单位类型。有了所有这些,可以编译以下代码:

\begin{cpp}
int x = 42;
int y = 77;

auto a = Duration<int, Ratio<1,1000>>(x); // x milliseconds
auto b = Duration<int, Ratio<2,3>>(y); // y 2/3 seconds
auto c = a + b; // computes resulting unit type 1/3000 seconds
				// and generates run-time code for c = a*3 + b*2000
\end{cpp}

关键的“混合”效果是,对于sum c,编译器在编译时确定结果单元类型Ratio<1,3000>,并生成代码在运行时计算结果值,该值会根据结果单元类型进行调整。

因为值类型是模板参数,所以可以将Duration类用于int以外的值类型,甚至可以使用异构值类型(只要定义了这些类型的值如何相加):

\begin{cpp}
auto d = Duration<double, Ratio<1,3>>(7.5); // 7.5 1/3 seconds
auto e = Duration<int, Ratio<1>>(4); // 4 seconds

auto f = d + e; // computes resulting unit type 1/3 seconds
				// and generates code for f = d + e*3
\end{cpp}

此外,若值在编译时已知,编译器甚至可以在编译时执行值计算,因为持续时间的operator+是constexpr。

C++标准库类模板std::chrono使用了这种方法,并进行了一些改进,例如使用预定义的单元(例如,std::chrono::milliseconds),支持字面时间段(例如,10ms),以及如何处理溢出。































