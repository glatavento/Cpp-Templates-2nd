\section{非类型}

类型列表提供了使用一组丰富的算法和操作描述,以及一系列操作类型的能力。其在处理编译时可以处理一组值,例如:多维数组的边界或另一个类型列表的索引。

有几种方法可以生成编译时值的类型列表。一种简单的方法包括定义一个CTValue类模板(编译时值命名),在一个类型列表中表示特定类型的值:

\begin{notice}
标准库定义了std::integral\_constant模板,它是CTValue的加强版本。
\end{notice}

\filename{typelist/ctvalue.hpp}
\begin{cpp}
template<typename T, T Value>
struct CTValue
{
	static constexpr T value = Value;
};
\end{cpp}

使用CTValue模板,可以表示一个包含前几个素数的整数值的类型列表:

\begin{cpp}
using Primes = Typelist<CTValue<int, 2>, CTValue<int, 3>,
						CTValue<int, 5>, CTValue<int, 7>,
						CTValue<int, 11>>;
\end{cpp}

通过这种表示,可以对值的类型列表执行数值计算,例如:计算这些质数的乘积。

首先,MultiplyT模板接受两个相同类型的编译时值,并生成一个相同类型的新的编译时值,将输入值相乘:

\filename{typelist/multiply.hpp}
\begin{cpp}
template<typename T, typename U>
struct MultiplyT;

template<typename T, T Value1, T Value2>
struct MultiplyT<CTValue<T, Value1>, CTValue<T, Value2>> {
	public:
	using Type = CTValue<T, Value1 * Value2>;
};

template<typename T, typename U>
using Multiply = typename MultiplyT<T, U>::Type;
\end{cpp}

然后,通过使用MultiplyT,下面的表达式得到所有质数的乘积:

\begin{cpp}
Accumulate<Primes, MultiplyT, CTValue<int, 1>>::value
\end{cpp}

不过,类型列表和CTValue的这种用法相当繁琐,特别是对于所有值都相同类型的情况。可以通过引入一个别名模板CTTypelist进行优化。提供了一个同类的值列表,描述为CTValues的类型列表:

\filename{typelist/cttypelist.hpp}
\begin{cpp}
template<typename T, T... Values>
using CTTypelist = Typelist<CTValue<T, Values>...>;
\end{cpp}

现在可以使用CTTypelist编写一个等效的(但更简洁的)质数定义:

\begin{cpp}
using Primes = CTTypelist<int, 2, 3, 5, 7, 11>;
\end{cpp}

这种方法的唯一缺点是别名模板只是别名,因此错误消息可能最终会打印CTValueTypes的底层类型列表。为了解决这个问题,可以创建一个全新的类型列表类Valuelist,可以存储值:

\filename{typelist/valuelist.hpp}
\begin{cpp}
template<typename T, T... Values>
struct Valuelist {};

template<typename T, T... Values>
struct IsEmpty<Valuelist<T, Values...>> {
	static constexpr bool value = sizeof...(Values) == 0;
};

template<typename T, T Head, T... Tail>
struct FrontT<Valuelist<T, Head, Tail...>> {
	using Type = CTValue<T, Head>;
	static constexpr T value = Head;
};

template<typename T, T Head, T... Tail>
struct PopFrontT<Valuelist<T, Head, Tail...>> {
	using Type = Valuelist<T, Tail...>;
};

template<typename T, T... Values, T New>
struct PushFrontT<Valuelist<T, Values...>, CTValue<T, New>> {
	using Type = Valuelist<T, New, Values...>;
};

template<typename T, T... Values, T New>
struct PushBackT<Valuelist<T, Values...>, CTValue<T, New>> {
	using Type = Valuelist<T, Values..., New>;
};
\end{cpp}

通过提供IsEmpty,FrontT,PopFrontT和PushFrontT,已经使Valuelist成为好用的类型列表,并且可以在本章定义的算法中使用。PushBackT作为一种算法特化提供,以减少编译时此操作的成本。例如,Valuelist可以与之前定义的InsertionSort算法一起使用:

\filename{typelist/valuelisttest.hpp}
\begin{cpp}
template<typename T, typename U>
struct GreaterThanT;

template<typename T, T First, T Second>
struct GreaterThanT<CTValue<T, First>, CTValue<T, Second>> {
	static constexpr bool value = First > Second;
};
void valuelisttest() {
	using Integers = Valuelist<int, 6, 2, 4, 9, 5, 2, 1, 7>;
	
	using SortedIntegers = InsertionSort<Integers, GreaterThanT>;
	
	static_assert(std::is_same_v<SortedIntegers,
								Valuelist<int, 9, 7, 6, 5, 4, 2, 2, 1>>,
				"insertion sort failed");
}
\end{cpp}

可以通过使用字面符操作符来初始化CTValue,

\begin{cpp}
auto a = 42_c; // initializes a as CTValue<int,42>
\end{cpp}

参阅第25.6节了解详细信息。

\subsection{可推导的非类型参数}

C++17中,CTValue可以通过使用单个可推导的非类型形参(用auto)进行改进:

\filename{typelist/ctvalue17.hpp}
\begin{cpp}
template<auto Value>
struct CTValue
{
	static constexpr auto value = Value;
};
\end{cpp}

这就省去了每次使用CTValue时指定类型的需要:

\begin{cpp}
using Primes = Typelist<CTValue<2>, CTValue<3>, CTValue<5>,
						CTValue<7>, CTValue<11>>;
\end{cpp}

C++17的值列表也可以这样,但结果不一定更好。如15.10.1节所述,带有推导类型的非类型参数包允许每个参数的类型不同:

\begin{cpp}
template<auto... Values>
class Valuelist { };

int x;
using MyValueList = Valuelist<1, 'a', true, &x>;
\end{cpp}

虽然这样的异构值列表可能有用,但与前面要求所有元素都具有相同类型的值列表不同。虽然可以要求所有元素都具有相同的类型(在15.10.1节中也讨论了这一点),但空的Valuelist<>肯定没有已知的元素类型。













