\section{基础操作}



\subsection{比较}

元组是包含其他值的结构类型。要比较两个元组,比较元素就足够了,可以实现operator==的定义来比较两个定义的元素:

\filename{typelist/tupleeq.hpp}
\begin{cpp}
// basis case:
bool operator==(Tuple<> const&, Tuple<> const&)
{
	// empty tuples are always equivalent
	return true;
}

// basis case:
bool operator==(Tuple<> const&, Tuple<> const&)
{
	// empty tuples are always equivalent
	return true;
}
\end{cpp}

与许多关于类型列表和元组的算法一样,元素比较先访问头部元素,然后递归访问尾部元素。操作符!=、<、>、<=和>=的顺序类似。

\subsection{输出}

本章将创建新的元组类型,因此能够在执行程序中看到这些元组是很有用的。以下操作符<{}<可以打印任何可打印的元组元素类型:

\filename{typelist/tupleio.hpp}
\begin{cpp}
#include <iostream>

void printTuple(std::ostream& strm, Tuple<> const&, bool isFirst = true)
{
	strm << ( isFirst ? '(' : ')' );
}

template<typename Head, typename... Tail>
void printTuple(std::ostream& strm, Tuple<Head, Tail...> const& t,
				bool isFirst = true)
{
	strm << ( isFirst ? "(" : ", " );
	strm << t.getHead();
	printTuple(strm, t.getTail(), false);
}

template<typename... Types>
std::ostream& operator<<(std::ostream& strm, Tuple<Types...> const& t)
{
	printTuple(strm, t);
	return strm;
}
\end{cpp}

现在,创建和显示元组就很容易:

\begin{cpp}
std::cout << makeTuple(1, 2.5, std::string("hello")) << '\n';
\end{cpp}

输出为

\begin{shell}
(1, 2.5, hello)
\end{shell}





















