\section{启用/禁用函数模板}

算法特化包括提供根据模板参数的属性选择的不同函数模板,函数模板的部分排序(16.2.2节)和重载解析(附录C)都不足以表达更高级的算法特化形式。

C++标准库为此提供了std::enable\_if(第6.3节)。本节讨论如何通过引入相应的别名模板来进行实现一个辅助类,将其称为EnableIf(以避免名称冲突)。

与std::enable\_if一样,EnableIf别名模板可以用于在特定条件下启用(或禁用)特定函数模板。advanceIter()算法的随机访问版本可以这样实现:

\begin{cpp}
template<typename Iterator>
constexpr bool IsRandomAccessIterator =
	IsConvertible<
		typename std::iterator_traits<Iterator>::iterator_category,
		std::random_access_iterator_tag>;
		
template<typename Iterator, typename Distance>
EnableIf<IsRandomAccessIterator<Iterator>>
advanceIter(Iterator& x, Distance n) {
	x += n; // constant time
}
\end{cpp}

EnableIf特化用于随机访问迭代器时,启用advanceIter()的相应版本。EnableIf的两个参数是一个布尔条件,指示是否应该启用该模板,以及当条件为true时,由EnableIf展开生成的类型。上面的例子中,使用类型特征IsConvertible(第19.5节和第19.7.3节中介绍)作为条件,定义了类型特征IsRandomAccessIterator。当Iterator的具体类型可用作随机访问迭代器时,才会考虑advanceIter()实现的这个版本(即其与一个可转换为std::random\_access\_iterator\_tag的标记相关联)。

EnableIf有一个相当简单的实现:

\filename{typeoverload/enableif.hpp}
\begin{cpp}
template<bool, typename T = void>
struct EnableIfT {
};

template<typename T>
struct EnableIfT<true, T> {
	using Type = T;
};

template<bool Cond, typename T = void>
using EnableIf = typename EnableIfT<Cond, T>::Type;
\end{cpp}

EnableIf展开为一个类型,因此实现为一个别名模板。我们希望对其实现使用偏特化(参见第16章),但是别名模板不能偏特化。这里,可以引入辅助类模板EnableIfT,其可以完成相应的实际工作,并使用别名模板EnableIf从辅助模板中简单地选择结果类型。当条件为true时,EnableIfT<…>::Type只计算第二个模板参数T。当条件为false时,EnableIf不会产生一个有效的类型,因为EnableIfT的主类模板没有Type成员。这将是一个错误,但在SFINAE(替换失败不是过,在第15.7节中描述)上下文中——函数模板的返回类型——会导致模板参数推导失败,从而忽略函数模板。

\begin{notice}
EnableIf也可以放在默认的模板参数中,这比放在结果类型中更有优势。参见第20.3.2节关于EnableIf位置的讨论。
\end{notice}

对于advanceIter(),使用EnableIf意味着,当Iterator参数是随机访问迭代器时,函数模板可用(并且返回类型为void),当Iterator参数不是随机访问迭代器时,函数模板将从对象中移除。可以把EnableIf看作是一种“保护”模板,当模板参数不满足模板实现要求的时,才实例化的方法。因为这个advanceIter()只能用随机访问迭代器实例化,所以相应的操作只能在随机访问迭代器上使用。虽然以这种方式使用EnableIf并不绝对可靠——用户可以断言某个类型是随机访问迭代器,而不需要提供必要的操作——但它可以帮助及早的诊断错误。

现在已经建立了显式地“激活”所应用类型的更特化的模板,这还不够:还必须“停用”不太特化的模板,因为编译器无法对这两个模板进行“排序”,若两个版本都适用,就会出现歧义错误。不过,实现这一点并不困难:只需要在不太特化的模板上使用相同的EnableIf模式,对条件表达式求反即可。这样做可以确保激活Iterator类型两个模板中的一个。因此,advanceIter()的非随机访问迭代器版本如下所示:

\begin{cpp}
template<typename Iterator, typename Distance>
EnableIf<!IsRandomAccessIterator<Iterator>>
advanceIter(Iterator& x, Distance n)
{
	while (n > 0) { // linear time
		++x;
		--n;
	}
}
\end{cpp}

\subsection{多个特化版本}

前面的模式可以推广到需要两个以上不同实现:为每个实现提供EnableIf构造,其条件对一组具体模板参数互斥。这些条件通常会利用各种特征表达的属性。

例如,引入advanceIter()算法的第三种实现:这次希望通过指定负距离来允许“后退”移动。

\begin{notice}
算法特化只用于在计算时间或资源使用方面提供效率增益。然而,一些算法的特化也提供了更多的功能,例如(在本例中)按顺序向后移动的能力。
\end{notice}

这对于输入迭代器无效,但对于随机访问迭代器有效。标准库还包含双向迭代器的概念,其允许向后移动,而不需要随机访问。实现这种情况需要复杂一点的逻辑:每个函数模板必须使用EnableIf和一个条件,该条件与表示算法不同实现的所有其他函数模板的条件互斥。这将需要以下一组条件:

\begin{itemize}
\item 
随机访问迭代器:随机访问情况(常量耗时,向前或向后)

\item
双向迭代器,而不是随机访问:双向情况(线性耗时,向前或向后)

\item
输入迭代器且非双向:一般情况(线性耗时,向前)
\end{itemize}

下面一组函数模板实现了这一点:

\filename{typeoverload/advance2.hpp}
\begin{cpp}
#include <iterator>

// implementation for random access iterators:
template<typename Iterator, typename Distance>
EnableIf<IsRandomAccessIterator<Iterator>>
advanceIter(Iterator& x, Distance n) {
	x += n; // constant time
}

template<typename Iterator>
constexpr bool IsBidirectionalIterator =
	IsConvertible<
		typename std::iterator_traits<Iterator>::iterator_category,
		std::bidirectional_iterator_tag>;

// implementation for bidirectional iterators:
template<typename Iterator, typename Distance>
EnableIf<IsBidirectionalIterator<Iterator> &&
		!IsRandomAccessIterator<Iterator>>
advanceIter(Iterator& x, Distance n) {
	if (n > 0) {
		for ( ; n > 0; ++x, --n) { // linear time
		}
	} else {
		for ( ; n < 0; --x, ++n) { // linear time
		}
	}
}

// implementation for all other iterators:
template<typename Iterator, typename Distance>
EnableIf<!IsBidirectionalIterator<Iterator>>
advanceIter(Iterator& x, Distance n) {
	if (n < 0) {
		throw "advanceIter(): invalid iterator category for negative n";
	}
	while (n > 0) { // linear time
		++x;
		--n;
	}
}
\end{cpp}

通过使每个函数模板的EnableIf条件与其他每个函数模板的EnableIf条件互斥,可以确保对于给定的参数集,最多有一个函数模板能成功推导模板参数。

示例说明了将EnableIf用于算法特化的缺点之一:每次引入算法的新实现时,都需要重新访问所有算法实现的条件,以确保所有实现都互斥。而使用标签调度(20.2节)引入双向迭代器变体,只需要使用标记std::bidirectional\_iterator\_tag添加一个新的advanceIterImpl()重载即可。

标签调度和EnableIf这两种技术很有用:标签调度支持基于分层标签的简单调度,而EnableIf支持基于由类型特征确定的属性集的高级调度。

\subsection{使用EnableIfGo}

EnableIf用于函数模板的返回类型。但这种方法不适用于构造函数模板或转换函数模板,因为没有指定的返回类型。

\begin{notice}
虽然转换函数模板确实有返回类型,但转换到该类型的模板参数必须可推导(参见第15章),这样转换函数模板才能正常工作。
\end{notice}

此外,EnableIf的使用会使返回类型很难阅读,所以可以将EnableIf嵌入到默认的模板参数中:

\filename{typeoverload/container1.hpp}
\begin{cpp}
#include <iterator>
#include "enableif.hpp"
#include "isconvertible.hpp"

template<typename Iterator>
constexpr bool IsInputIterator =
	IsConvertible<
		typename std::iterator_traits<Iterator>::iterator_category,
		std::input_iterator_tag>;

template<typename T>
class Container {
	public:
	// construct from an input iterator sequence:
	template<typename Iterator,
			typename = EnableIf<IsInputIterator<Iterator>>>
	Container(Iterator first, Iterator last);
	
	// convert to a container so long as the value types are convertible:
	template<typename U, typename = EnableIf<IsConvertible<T, U>>>
	operator Container<U>() const;
};
\end{cpp}

若添加另一个重载(更高效的随机访问迭代器Container构造函数版本),将会导致错误:

\begin{cpp}
// construct from an input iterator sequence:
template<typename Iterator,
		typename = EnableIf<IsInputIterator<Iterator> &&
							!IsRandomAccessIterator<Iterator>>>
Container(Iterator first, Iterator last);

template<typename Iterator,
		typename = EnableIf<IsRandomAccessIterator<Iterator>>>
Container(Iterator first, Iterator last); // ERROR: redeclaration
										  // of constructor template
\end{cpp}

问题是,除了默认模板参数外,两个构造函数模板完全相同,但是在确定两个模板是否等价时不会考虑默认模板参数。

可以通过添加另一个默认的模板参数来解决这个问题,这样两个构造函数模板的模板参数数量就不同了:

\begin{cpp}
// construct from an input iterator sequence:
template<typename Iterator,
		typename = EnableIf<IsInputIterator<Iterator> &&
							!IsRandomAccessIterator<Iterator>>>
Container(Iterator first, Iterator last);

template<typename Iterator,
		typename = EnableIf<IsRandomAccessIterator<Iterator>>,
		typename = int> // extra dummy parameter to enable both constructors
Container(Iterator first, Iterator last); // OK now
\end{cpp}

\subsection{编译时if}

C++17的constexpr if特性(请参阅第8.5节)可以替代EnableIf。C++17中,可以这样重写advanceIter():

\filename{typeoverload/advance3.hpp}
\begin{cpp}
template<typename Iterator, typename Distance>
void advanceIter(Iterator& x, Distance n) {
	if constexpr(IsRandomAccessIterator<Iterator>) {
		// implementation for random access iterators:
		x += n; // constant time
	}
	else if constexpr(IsBidirectionalIterator<Iterator>) {
		// implementation for bidirectional iterators:
		if (n > 0) {
			for ( ; n > 0; ++x, --n) { // linear time for positive n
			}
		} else {
			for ( ; n < 0; --x, ++n) { // linear time for negative n
			}
		}
	}
	else {
		// implementation for all other iterators that are at least input iterators:
		if (n < 0) {
			throw "advanceIter(): invalid iterator category for negative n";
		}
		while (n > 0) { // linear time for positive n only
			++x;
			--n;
		}
	}
}
\end{cpp}

这就清楚多了!更特化的代码路径(用于随机访问迭代器)会支持它们的类型实例化。因此,只要操作使用if constexpr保护,那代码中包含并非所有迭代器中都有的操作(例如+=)就是安全的。

然而,其也有缺点。当通用组件中的差异完全可以在函数模板体中表达时,才可能使用constexpr if。下面的情况中,仍然需要EnableIf:

\begin{itemize}
\item
涉及到不同的“接口”

\item
需要不同的类定义

\item
对于某些模板参数列表,不存在有效的实例化。
\end{itemize}

可以用下面的模式来处理最后一种情况:

\begin{cpp}
template<typename T>
void f(T p) {
	if constexpr (condition<T>::value) {
		// do something here...
	}
	else {
		// not a T for which f() makes sense:
		static_assert(condition<T>::value, "can't call f() for such a T");
	}
}
\end{cpp}

不过这样做并不可取,因为其不适用于SFINAE:函数f<T>()没有从候选列表中删除,因此可能会抑制另一个重载解析的结果。替代方案中,使用EnableIf f<T>()将删除EnableIf<…>失败情况的替换。

\subsection{概念}

目前为止所介绍的技术表现的都还好,但是有点笨拙,可能会使用大量的编译器资源,并且在出现错误的情况下,可能会出现不清晰的诊断信息。因此,许多泛型库的作者都期望有一种能够更直接地达到同样效果的语言特性。出于这个原因,概念的特性可能会添加到语言中;请参阅第6.5节、第18.4节和附录E。

例如,期望重载的Container构造函数如下所示:

\filename{typeoverload/container4.hpp}
\begin{cpp}
template<typename T>
class Container {
	public:
	// construct from an input iterator sequence:
	template<typename Iterator>
	requires IsInputIterator<Iterator>
	Container(Iterator first, Iterator last);
	
	// construct from a random access iterator sequence:
	template<typename Iterator>
	requires IsRandomAccessIterator<Iterator>
	Container(Iterator first, Iterator last);
	
	// convert to a container so long as the value types are convertible:
	template<typename U>
	requires IsConvertible<T, U>
	operator Container<U>() const;
};
\end{cpp}

require子句(在E.1节中讨论)描述了模板的需求。若没有满足要求,该模板就不认为是候选模板。因此,其是EnableIf思想的更直接的表达,语言本身也支持这种表达。

与EnableIf相比,requires子句有更多的优点。约束包含(在第E.3.1节)提供了模板之间的排序,只在require子句中有所不同,从而消除了标签调度的需要。此外,可以将require子句附加到非模板上。仅当类型T可使用小于操作符进行比较时,才提供sort()成员函数:

\begin{cpp}
template<typename T>
class Container {
	public:
	...
	requires HasLess<T>
	void sort() {
		...
	}
};
\end{cpp}





