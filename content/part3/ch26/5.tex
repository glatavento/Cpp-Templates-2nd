\section{访问}

is()和get()成员函数可以检查动态值是否为特定类型,并访问该类型的值。但检查变量中的所有可能类型很快就会变成冗长的if语句链。下面打印了名为v的Variant<int, double, string>的值:

\begin{cpp}
if (v.is<int>()) {
	std::cout << v.get<int>();
}
else if (v.is<double>()) {
	std::cout << v.get<double>();
}
else {
	std::cout << v.get<string>();
}
\end{cpp}

要一般化输出存储在变量中的值,需要递归实例化函数模板和辅助函数。例如:

\filename{variant/printrec.cpp}
\begin{cpp}
#include "variant.hpp"
#include <iostream>

template<typename V, typename Head, typename... Tail>
void printImpl(V const& v) {
	if (v.template is<Head>()) {
		std::cout << v.template get<Head>();
	}
	else if constexpr (sizeof...(Tail) > 0) {
		printImpl<V, Tail...>(v);
	}
}

template<typename... Types>
void print(Variant<Types...> const& v) {
	printImpl<Variant<Types...>, Types...>(v);
}

int main() {
	Variant<int, short, float, double> v(1.5);
	print(v);
}
\end{cpp}

对于一个简单的操作来说,这代码体积已经相当大了。为了简化,通过使用visit()操作扩展Variant来解决这个问题。外部传入访问函数对象,该对象的函数操作符的调用将使用动态值。因为动态值可以是Variant的任何类型,所以这个函数操作符很可能是重载的,或者本身就是一个函数模板。泛型Lambda提供了一个模板函数操作符,可以表示Variant v的打印操作:

\begin{cpp}
v.visit([](auto const& value) {
			std::cout << value;
		});
\end{cpp}

泛型Lambda大致等价于下面的函数对象,对于还不支持泛型Lambda的编译器也有用:

\begin{cpp}
class VariantPrinter {
	public:
	template<typename T>
	void operator()(T const& value) const
	{
		std::cout << value;
	}
};
\end{cpp}

visit()操作的核心类似于递归打印操作:遍历变量的类型,检查动态值是否具有给定的类型(使用is<T>()),然后在找到合适的类型时进行打印:

\filename{variant/variantvisitimpl.hpp}
\begin{cpp}
template<typename R, typename V, typename Visitor,
		 typename Head, typename... Tail>
R variantVisitImpl(V&& variant, Visitor&& vis, Typelist<Head, Tail...>) {
	if (variant.template is<Head>()) {
		return static_cast<R>(
		std::forward<Visitor>(vis)(
		std::forward<V>(variant).template get<Head>()));
	}
	else if constexpr (sizeof...(Tail) > 0) {
		return variantVisitImpl<R>(std::forward<V>(variant),
		std::forward<Visitor>(vis),
		Typelist<Tail...>());
	}
	else {
		throw EmptyVariant();
	}
}
\end{cpp}

variantVisitImpl()是一个带有许多模板参数的非成员函数模板。模板参数R描述了访问操作的结果类型,在稍后返回。V是Variant类型,访问器是Visitor的类型。Head和Tail用于分解Variant中的类型,从而影响递归。

第一个if执行(运行时)检查,以确定给定变量的活动值是否为Head类型:若是,则通过get()从变量中提取值,并传递给visitor,终止递归。第二个if在需要考虑更多元素时执行递归。若没有匹配的类型,则该Variant不包含值,

\begin{notice}
这个案例在第26.4.3节有详细的讨论
\end{notice}

这种情况下,实现可以抛出EmptyVariant异常。

除了VisitResult提供的结果类型计算(在下一节中讨论),visit()的实现也很简单:

\filename{variant/variantvisit.hpp}
\begin{cpp}
template<typename... Types>
	template<typename R, typename Visitor>
VisitResult<R, Visitor, Types&...>
Variant<Types...>::visit(Visitor&& vis)& {
	using Result = VisitResult<R, Visitor, Types&...>;
	return variantVisitImpl<Result>(*this, std::forward<Visitor>(vis),
									Typelist<Types...>());
}

template<typename... Types>
template<typename R, typename Visitor>
VisitResult<R, Visitor, Types const&...>
Variant<Types...>::visit(Visitor&& vis) const& {
	using Result = VisitResult<R, Visitor, Types const &...>;
	return variantVisitImpl<Result>(*this, std::forward<Visitor>(vis),
									Typelist<Types...>());
}

template<typename... Types>
template<typename R, typename Visitor>
VisitResult<R, Visitor, Types&&...>
Variant<Types...>::visit(Visitor&& vis) && {
	using Result = VisitResult<R, Visitor, Types&&...>;
	return variantVisitImpl<Result>(std::move(*this),
									std::forward<Visitor>(vis),
									Typelist<Types...>());
}
\end{cpp}

实现直接委托给variantVisitImpl,传递Variant本身,转发访问器,并提供完整的类型列表。这三种实现之间唯一的区别是,是否将Variant本身作为Variant\&、Variant const\&或Variant\&\&进行传递。

\subsection{结果类型}

visit()的结果类型仍然是个谜。给定的访问器可能具有不同的函数操作符重载,以产生不同的结果类型,结果类型依赖于其形参类型的模板函数操作符,或者某种组合。看看下面的泛型Lambda:

\begin{cpp}
[](auto const& value) {
	return value + 1;
}
\end{cpp}

这个Lambda的结果类型取决于输入类型:给定int型,则将产生int型值;给定double型,将产生double型值。若这个泛型Lambda传递给了Variant<int, double>的visit()操作,结果类型应该是什么?

这里没有正确的答案,因此visit()操作允许提供结果类型。在另一个Variant<int, double>中捕获结果,可以将结果类型指定为visit()作为第一个模板参数:

\begin{cpp}
v.visit<Variant<int, double>>([](auto const& value) {
								return value + 1;
							});
\end{cpp}

当没有通用解决方案时,指定结果类型就很重要。但要求在所有情况下指定结果类型可能会让代码显得非常冗长,visit()使用默认模板参数和简单元程序组合了这两个选项。回顾visit()的声明:

\begin{cpp}
template<typename R = ComputedResultType, typename Visitor>
VisitResult<R, Visitor, Types&...> visit(Visitor&& vis) &;
\end{cpp}

上面的例子中显式地指定了模板参数R,有一个默认参数,因此不需要总是指定。该默认参数是一个不完整的哨兵类型ComputedResultType:

\begin{cpp}
class ComputedResultType;
\end{cpp}

要计算它的结果类型,visit将所有的模板参数传递给VisitResult,这是一个别名模板,提供了对新类型特征VisitResultT的访问:

\filename{variant/variantvisitresult.hpp}
\begin{cpp}
// an explicitly-provided visitor result type:
template<typename R, typename Visitor, typename... ElementTypes>
class VisitResultT {
	public:
	using Type = R;
};

template<typename R, typename Visitor, typename... ElementTypes>
using VisitResult =
typename VisitResultT<R, Visitor, ElementTypes...>::Type;
\end{cpp}

VisitResultT的主要定义了处理R参数显式指定的情况,因此Type定义为R。当R接收到默认参数ComputedResultType时,启用偏特化:

\begin{cpp}
template<typename Visitor, typename... ElementTypes>
class VisitResultT<ComputedResultType, Visitor, ElementTypes...> {
	...
}
\end{cpp}

这种偏特化负责为常见情况计算适当的结果类型,这是下一节的主题。

\subsection{常见的结果类型}

当调用为每个变量的元素类型产生不同类型的访问器时,如何将这些类型组合成一个单独的结果类型用于visit()?有一些简单的情况——若访问器为每个元素类型返回相同的类型,应该是visit()的结果类型。

C++已经有了合理结果类型的概念,在第1.3.3节中介绍过:三元表达式b ? x : y的类型是x和y类型之间的公共类型。若x有int类型,y有double类型,公共类型是double,因为int可以提升为double。也可以在类型特征中捕捉到共同类型的概念:

\filename{variant/commontype.hpp}
\begin{cpp}
using std::declval;

template<typename T, typename U>
class CommonTypeT {
	public:
	using Type = decltype(true? declval<T>() : declval<U>());
};

template<typename T, typename U>
using CommonType = typename CommonTypeT<T, U>::Type;
\end{cpp}

公共类型的概念扩展到一组类型:公共类型是集合中的所有类型,都可以提升到的类型。对于访问器,我们希望计算访问器在使用变量中的每个类型调用时,将产生的结果的通用类型:

\filename{variant/variantvisitresultcommon.hpp}
\begin{cpp}
#include "accumulate.hpp"
#include "commontype.hpp"

// the result type produced when calling a visitor with a value of type T:
template<typename Visitor, typename T>
using VisitElementResult = decltype(declval<Visitor>()(declval<T>()));

// the common result type for a visitor called with each of the given element types:
template<typename Visitor, typename... ElementTypes>
class VisitResultT<ComputedResultType, Visitor, ElementTypes...> {
	using ResultTypes =
		Typelist<VisitElementResult<Visitor, ElementTypes>...>;
	public:
	using Type =
		Accumulate<PopFront<ResultTypes>, CommonTypeT, Front<ResultTypes>>;
};
\end{cpp}

VisitResult的计算分两个阶段进行。首先,VisitElementResult计算调用访问器的值为T类型时,产生的结果类型。这个元函数应用于每个给定的元素类型,以确定访问器可以产生的所有结果类型,并在类型列表ResultTypes中进行捕获。

接下来,计算使用24.2.6节中描述的Accumulate算法,将公共类型计算应用于结果类型的类型列表。其初始值(Accumulate的第三个参数)是第一个结果类型,通过CommonTypeT与ResultTypes类型列表剩余部分的连续值组合在一起。最终结果是通用类型,访问器的所有结果类型都可以转换为通用类型,若结果类型不兼容,则会出现错误。

C++11后,标准库提供了一个相应的类型特征,std::common\_type<>,有效地结合了CommonTypeT和Accumulate,使用这种方法生成任意数量传递类型的通用类型(参见D.5节)。通过使用std::common\_type<>,VisitResultT的实现可以更简单:

\filename{variant/variantvisitresultstd.hpp}
\begin{cpp}
template<typename Visitor, typename... ElementTypes>
class VisitResultT<ComputedResultType, Visitor, ElementTypes...> {
	public:
	using Type =
		std::common_type_t<VisitElementResult<Visitor, ElementTypes>...>;
};
\end{cpp}

下面的示例程序输出了通过传入泛型Lambda来生成的类型,该泛型Lambda将其得到的值加1:

\filename{variant/visit.cpp}
\begin{cpp}
#include "variant.hpp"
#include <iostream>
#include <typeinfo>

int main() {
	Variant<int, short, double, float> v(1.5);
	auto result = v.visit([](auto const& value) {
							return value + 1;
						});
	std::cout << typeid(result).name() << '\n';
}
\end{cpp}

因为结果是所有结果类型都可以转换的类型,所以这个程序的输出将是double的type\_info名。







