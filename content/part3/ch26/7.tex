\section{后记}
Andrei Alexandrescu在一系列文章[alexandrescuatedunions]中详细介绍了可辨别联合。我们对Variant的处理使用了相同的技术,比如使用对齐缓冲区,就地存储和访问来提取值。一些差异是由于基础语言造成的:Andrei使用的是C++98,因此不能使用可变参数模板或继承构造函数。Andrei还花了相当多的时间来计算对齐,C++11直接引入了对齐,使得这项工作变得很简单。设计差异在于对辨别器的处理:虽然我们选择使用整型辨别器来指示,当前存储在变体中的是哪种类型,但Andrei使用了一种“静态虚函数表”的方法,使用函数指针来构造、复制、查询和销毁底层元素类型。有趣的是,这种静态虚函数表方法作为开放可辨别联合的优化技术影响更大,比如在第22.2节中开发的FunctionPtr模板,它是std::function实现的一种常见优化,以消除虚函数的使用。Boost的any类型([BoostAny])是另一种开放可辨别联合类型。C++17的标准库中,引入了std::any。

后来,Boost库([Boost])引入了几种可辨别联合类型,包括一种变体类型([BoostVariant]),它影响了本章中开发的联合类型。Boost.Variant([BoostVariant])的设计文档,包含了关于变量赋值异常安全的问题(称为“永不为空的约定”)和各种不完全令人满意的解决方案讨论记录。当标准库在C++17引入std::variant时,放弃了永不空的约定:通过允许std::variant状态可以变成valueless\_by\_exception,从而消除了为备份分配堆存储的需要,而std::variant状态会赋值给它抛出的新值,我们用空变量对这一行为进行了建模。

与我们的Variant模板不同,std::variant允许多个相同的模板参数(例如,std::variant<int, int>)。在Variant中启用该功能需要在设计方面进行大量修改,包括添加一个方法来消除VariantChoice基类的歧义,以及在26.2节中描述的嵌套包扩展的替代方法。

本章描述的visit()操作变体在结构上与Andrei Alexandrescu在[AlexandrescuAdHocVisitor]中描述的临时访问器模式相同。Alexandrescu的特别访问器旨在简化针对一组已知派生类(描述为类型列表)检查某个公共基类指针的过程。该实现使用dynamic\_cast来针对类型列表中的每个派生类测试指针,当发现匹配时,使用派生类指针调用访问器。



