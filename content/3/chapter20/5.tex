EnableIf本质是只在模板参数满足某些特定条件时,启用特化或偏特化模板。advanceIter()算法有效的形式是检查迭代器参数类别是否可转换为std::random\_access\_iterator\_tag,从而算法可用于各种随机访问迭代器。

若把这个概念发挥到极致,模板对其模板参数执行的每个操作都编码为EnableIf条件的一部分,会怎么样呢?模板实例化永远不会失败,因为不提供所需操作的模板参数推导将失败(通过EnableIf),而不是允许实例化继续进行。我们将这种模板称为“实例化安全”模板,并在这里介绍这种模板的实现。

我们从一个非常基本的模板min()开始,计算两个值的最小值。通常会这样实现一个模板:

\begin{cpp}
template<typename T>
T const& min(T const& x, T const& y)
{
	if (y < x) {
		return y;
	}
	return x;
}
\end{cpp}

该模板要求类型T具有能够使用小于操作符比较两个T值(特别是两个T const左值),然后隐式地将比较结果转换为bool,以便在if语句中使用。检查小于操作符,并计算结果类型的特征类似于在第19.4.4节讨论的SFINAE友好的PlusResultT特征,但为了方便起见,在这里先展示一下LessResultT特征:

\filename{typeoverload/lessresult.hpp}
\begin{cpp}
#include <utility> // for declval()
#include <type_traits> // for true_type and false_type

template<typename T1, typename T2>
class HasLess {
	template<typename T> struct Identity;
	template<typename U1, typename U2> static std::true_type
		test(Identity<decltype(std::declval<U1>() < std::declval<U2>())>*);
	template<typename U1, typename U2> static std::false_type
		test(...);
	public:
	static constexpr bool value = decltype(test<T1, T2>(nullptr))::value;
};

template<typename T1, typename T2, bool HasLess>
class LessResultImpl {
	public:
	using Type = decltype(std::declval<T1>() < std::declval<T2>());
};

template<typename T1, typename T2>
class LessResultImpl<T1, T2, false> {
};

template<typename T1, typename T2>
class LessResultT
: public LessResultImpl<T1, T2, HasLess<T1, T2>::value> {
};

template<typename T1, typename T2>
using LessResult = typename LessResultT<T1, T2>::Type;
\end{cpp}

这个特征可以和IsConvertible特征组合在一起,使min()实例化安全:

\filename{typeoverload/min2.hpp}
\begin{cpp}
#include "isconvertible.hpp"
#include "lessresult.hpp"

template<typename T>
EnableIf<IsConvertible<LessResult<T const&, T const&>, bool>,
		T const&>
min(T const& x, T const& y)
{
	if (y < x) {
		return y;
	}
	return x;
}
\end{cpp}

尝试使用不同类型的小于操作符(或完全忽略操作符)来调用这个min()函数:

\filename{typeoverload/min.cpp}
\begin{cpp}
#include "min.hpp"

struct X1 { };
bool operator< (X1 const&, X1 const&) { return true; }

struct X2 { };
bool operator<(X2, X2) { return true; }

struct X3 { };
bool operator<(X3&, X3&) { return true; }

struct X4 { };
struct BoolConvertible {
	operator bool() const { return true; } // implicit conversion to bool
};

struct X5 { };
BoolConvertible operator< (X5 const&, X5 const&)
{
	return BoolConvertible();
}

struct NotBoolConvertible { // no conversion to bool
};

struct X6 { };
NotBoolConvertible operator< (X6 const&, X6 const&)
{
	return NotBoolConvertible();
}

struct BoolLike {
	explicit operator bool() const { return true; } // explicit conversion to bool
};
struct X7 { };
BoolLike operator< (X7 const&, X7 const&) { return BoolLike(); }

int main()
{
	min(X1(), X1()); // X1 can be passed to min()
	min(X2(), X2()); // X2 can be passed to min()
	min(X3(), X3()); // ERROR: X3 cannot be passed to min()
	min(X4(), X4()); // ERROR: X4 cannot be passed to min()
	min(X5(), X5()); // X5 can be passed to min()
	min(X6(), X6()); // ERROR: X6 cannot be passed to min()
	min(X7(), X7()); // UNEXPECTED ERROR: X7 cannot be passed to min()
}
\end{cpp}

编译时虽然对X3、X4、X6和X7的4个不同min()调用存在错误,但这些错误并不来自min()的主体。相反,而是抱怨没有合适的min()函数,因为SFINAE已经取消了匹配的选项。Clang会输出以下诊断信息:

\begin{shell}
min.cpp:41:3: error: no matching function for call to ’min’
min(X3(), X3()); // ERROR: X3 cannot be passed to min
^~~
./min.hpp:8:1: note: candidate template ignored: substitution failure
[with T = X3]: no type named ’Type’ in
’LessResultT<const X3 &, const X3 &>’
min(T const& x, T const& y)
\end{shell}

因此,EnableIf只允许实例化那些满足模板要求的模板参数(X1, X2,和X5),所以从min()函数体中不会得到错误。此外,若有一些可能适用于这些类型的min()的其他重载,重载解析可以选择其中一个。

示例中的最后一个类型X7,演示了实现实例化安全模板的一些微妙之处。特别是,若将X7传递给非实例化安全的min(),则实例化成功。因为BoolLike不能隐式转换为bool,所以实例化安全的min()不能使用。这里的区别特别微妙:显式转换为bool可以在某些上下文中隐式使用,包括控制流语句的布尔条件(if、while、for和do)、内置的!、\&\&和||操作符以及三元操作符。这些上下文中,值可以转换为bool。

\begin{notice}C++11引入了到bool的上下文转换概念,以及显式转换操作符。它们取代了“安全bool”的习惯用法([KarlssonSafeBool]),这通常涉及到(隐式)用户定义的数据指针转换。使用指向数据的指针的原因是可以将其视为bool值,而不需要其他多余的转换,例如bool作为算术操作的一部分可以提升为int。BoolConvertible() + 5(不幸的是)是定义良好的代码。
\end{notice}

然而,我们坚持使用一般的、隐式的bool转换,结果导致实例化安全模板过度约束;指定的需求(在EnableIf中)比实际需求(模板需要正确实例化的东西)更强。另一方面,若完全忘记了转换到bool的要求,min()模板就会约束不足,会允许一些可能导致实例化失败的模板参数参与实例化(例如X6)。

为了修复实例化安全的min(),需要特征来确定类型T是否在上下文中可转换为bool。控制流语句对定义特征没有帮助,因为语句不能在SFINAE中出现(逻辑操作也不能),所以可以为任意类型重载逻辑操作。三元操作符是一个表达式,并且没有重载,因此可以使用它来测试类型是否在上下文条件下可以转换为bool类型:

\filename{typeoverload/iscontextualbool.hpp}
\begin{cpp}
#include <utility> // for declval()
#include <type_traits> // for true_type and false_type

template<typename T>
class IsContextualBoolT {
	private:
	template<typename T> struct Identity;
	template<typename U> static std::true_type
		test(Identity<decltype(declval<U>()? 0 : 1)>*);
	template<typename U> static std::false_type
		test(...);
	public:
	static constexpr bool value = decltype(test<T>(nullptr))::value;
};

template<typename T>
constexpr bool IsContextualBool = IsContextualBoolT<T>::value;
\end{cpp}

有了这个新特征,可以在EnableIf中提供一个实例化安全的min(),其包含正确的需求集:

\filename{typeoverload/min3.hpp}
\begin{cpp}
#include "iscontextualbool.hpp"
#include "lessresult.hpp"

template<typename T>
EnableIf<IsContextualBool<LessResult<T const&, T const&>>,
		T const&>
min(T const& x, T const& y)
{
	if (y < x) {
		return y;
	}
	return x;
}
\end{cpp}

这里用于使min()实例化安全的技术,可以扩展为描述非简单模板的需求,方法是将各种需求检查组合到描述某些类型类的特征中,比如前向迭代器,并在EnableIf中组合这些特征。这样做有两个好处,一是可以更好地重载行为,二是可以消除编译器在嵌套模板实例化中深入输出错误时,产生新的错误。另一方面,提供的错误消息往往缺乏关于哪个特定操作失败的特性。此外,正如用min()所展示的那样,准确地确定和编码模板的需求,可能是一项令人生畏的任务。我们将在28.2节中探讨利用这些特性进行调试的方法。





