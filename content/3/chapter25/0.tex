本书中,我们经常使用同构容器和类数组类型来说明模板的强大功能。这种同构结构扩展了C/C++数组的概念,在大多数应用程序中普遍存在。C++(和C)也有非同构的容器设施:类(或结构体),本章探讨元组类似于类和结构的方式聚合数据,包含int、double和std::string的元组类似于包含int、double和std::string成员的结构体,只是元组的元素是按位置引用的(如0、1、2),而不是通过名称引用。位置接口和从类型列表轻松构造元组的能力,使元组比结构体更适合与模板元编程技术一起使用。

元组的另一种方式是在可执行程序中显示类型列表,Typelist<int, double, std::string>描述了可以在编译时操作的包含int, double和std::string的类型序列,Tuple<int, double, std::string>描述了可以在运行时操作的int, double和std::string的存储。例如,下面的程序创建这样一个元组的实例:

\begin{cpp}
template<typename... Types>
class Tuple {
	... // implementation discussed below
};

Tuple<int, double, std::string> t(17, 3.14, "Hello, World!");
\end{cpp}

通常使用模板元编程和类型列表来生成可用于存储数据的元组,即使在上面的例子中选择了int、double和std::string作为元素类型,也可以用元程序创建元组中存储的类型集。

本章,我们将探索Tuple类模板的实现和操作,是std::Tuple类模板的简化版本。






























