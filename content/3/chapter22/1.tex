
函数对象对于为模板提供可定制的行为,下面的函数模板枚举从0到某个值的整数值,并将每个值提供给给定的函数对象f:

\filename{bridge/forupto1.cpp}
\begin{cpp}
#include <vector>
#include <iostream>

template<typename F>
void forUpTo(int n, F f)
{
	for (int i = 0; i != n; ++i)
	{
		f(i); // call passed function f for i
	}
}

void printInt(int i)
{
	std::cout << i << ’ ’;
}

int main()
{
	std::vector<int> values;
	
	// insert values from 0 to 4:
	forUpTo(5,
	[&values](int i) {
		values.push_back(i);
	});

	// print elements:
	forUpTo(5,
			printInt); // prints 0 1 2 3 4
	std::cout << ’\n’;
}
\end{cpp}

forUpTo()函数模板可以与函数对象一起使用,包括Lambda、函数指针、函数操作符或转换为函数指针/引用的类,每次使用forUpTo()都可能产生不同的实例化。示例函数模板非常小,但若模板很大,这些实例化可能会增加代码量。

限制代码量增加的方法,是将函数模板转换为不需要实例化的非模板。可以尝试使用函数指针:

\filename{bridge/forupto2.hpp}
\begin{cpp}
void forUpTo(int n, void (*f)(int))
{
	for (int i = 0; i != n; ++i)
	{
		f(i); // call passed function f for i
	}
}
\end{cpp}

虽然这个实现在传递printInt()时正常工作,但在传递Lambda时将产生错误:

\begin{cpp}
forUpTo(5,
printInt); // OK: prints 0 1 2 3 4

forUpTo(5,
		[&values](int i) { // ERROR: lambda not convertible to a function pointer
			values.push_back(i);
		});
\end{cpp}

标准库的类模板std::function<>允许使用forUpTo()的替代表达式:

\filename{bridge/forupto3.hpp}
\begin{cpp}
#include <functional>
void forUpTo(int n, std::function<void(int)> f)
{
	for (int i = 0; i != n; ++i)
	{
		f(i); // call passed function f for i
	}
}
\end{cpp}

std::function<>的模板参数是一个函数类型,描述了函数对象将接收的参数类型和应该产生的返回类型,就像函数指针描述参数和结果类型一样。

forUpTo()提供了静态多态的一些方面——使用合适的函数操作符处理一组无界类型的能力,包括函数指针、Lambda和类——而其本身仍然是非模板函数,只有一个实现。可以使用称为类型擦除的技术来实现,这种技术弥合了静态和动态多态之间的差异。






























