
另一个模式是奇异递归模板模式(CRTP)。这个命名奇怪的模式指的是一种通用的技术类,包括将一个派生类作为模板参数传递给自己的基类。这种模式最简单的C++代码如下所示:

\begin{cpp}
template<typename Derived>
class CuriousBase {
	...
};

class Curious : public CuriousBase<Curious> {
	...
};
\end{cpp}

CRTP的第一个概要展示了一个非依赖性基类:类Curious不是模板,因此不受依赖基类的一些名称可见性问题的影响。然而,这并不是CRTP的内在特征。完全可以使用以下备选方式:

\begin{cpp}
template<typename Derived>
class CuriousBase {
	...
};

template<typename T>
class CuriousTemplate : public CuriousBase<CuriousTemplate<T>> {
	...
};
\end{cpp}

通过模板参数将派生类向下传递给基类,基类可以自定义派生类的行为,而不需要使用虚函数。这使得CRTP可以分解出只能是成员函数(例如,构造函数、析构函数和下标操作符)或依赖于派生类标识的实现。

CRTP的简单应用包括跟踪创建了多少特定类类型的对象,这很容易通过在每个构造函数中递增一个整型静态数据成员,并在析构函数中递减来实现。但必须在每个类中提供这样的代码就很烦,并且通过单个(非CRTP)基类实现此功能会混淆不同派生类的对象计数。不过,可以这样编写模板:

\filename{inherit/objectcounter.hpp}
\begin{cpp}
#include <cstddef>
template<typename CountedType>
class ObjectCounter {
private:
	inline static std::size_t count = 0; // number of existing objects

protected:
	
	// default constructor
	ObjectCounter() {
		++count;
	}

	// copy constructor
	ObjectCounter (ObjectCounter<CountedType> const&) {
		++count;
	}

	// move constructor
	ObjectCounter (ObjectCounter<CountedType> &&) {
		++count;
	}

	// destructor
	~ObjectCounter() {
		--count;
	}

public:
	// return number of existing objects:
	static std::size_t live() {
		return count;
	}
};
\end{cpp}

使用内联是为了能够在类结构内部,定义和初始化count成员。C++17前,必须在类模板外定义:

\begin{cpp}
template<typename CountedType>
class ObjectCounter {
	private:
	static std::size_t count; // number of existing objects
	...
};

// initialize counter with zero:
template<typename CountedType>
std::size_t ObjectCounter<CountedType>::count = 0;
\end{cpp}

若想计算某个类类型的活(即未销毁)对象的数量,从ObjectCounter模板派生类就足够了。可以这样定义和使用用于计数的字符串类:

\filename{inherit/countertest.cpp}
\begin{cpp}
#include "objectcounter.hpp"
#include <iostream>

template<typename CharT>
class MyString : public ObjectCounter<MyString<CharT>> {
	...
};

int main()
{
	MyString<char> s1, s2;
	MyString<wchar_t> ws;
	std::cout << "num of MyString<char>: "
				<< MyString<char>::live() << ’\n’;
	std::cout << "num of MyString<wchar_t>: "
				<< ws.live() << ’\n’;
}
\end{cpp}

\subsection{Barton-Nackman技巧(友元工厂)}

1994年,John J. Barton和Lee R. Nackman提出了一种模板技术,他们称之为限制模板展开(参见[BartonNackman])。这种技术的部分动机是由于函数模板的重载在当时非常受限

\begin{notice}为了了解函数模板重载在现代C++中的工作原理,有必要阅读第16.2节。
\end{notice}

并且在大多数编译器中无法使用命名空间。

假设一个类模板Array,希望为其定义相等operator==。一种可能是将操作符声明为类模板的成员,但这不是很好的实践方式,因为第一个参数(绑定到this指针)要遵守与第二个参数不同的转换规则。因为operator==参数对称,所以最好将其声明为命名空间作用域的函数。其实现方法大致如下所示:

\begin{cpp}
template<typename T>
class Array {
	public:
	...
};

template<typename T>
bool operator== (Array<T> const& a, Array<T> const& b)
{
	...
}
\end{cpp}

但若函数模板不能重载,就会出现一个问题:不能在该作用域中声明其他operator==模板,而其他类模板很可能需要这样的模板。Barton和Nackman将类中的操作符定义为友元函数,解决了这个问题:

\begin{cpp}
template<typename T>
class Array {
	static bool areEqual(Array<T> const& a, Array<T> const& b);
	
	public:
	...
	friend bool operator== (Array<T> const& a, Array<T> const& b) {
		return areEqual(a, b);
	}
};
\end{cpp}

假设这个版本的Array实例化为float类型,友元操作符函数声明为该实例化的结果,但该函数本身不是函数模板的实例化,其只是普通的非模板函数,作为实例化过程的副作用注入到全局作用域中。因为是非模板函数,所以即使在向语言中添加函数模板重载之前,也可以使用operator==的其他声明重载。Barton和Nackman将此称为限制模板展开,因为它避免使用适用于所有类型T(换句话说,不受限制的展开)的模板operator==(T, T)。

因为

\begin{cpp}
operator== (Array<T> const&, Array<T> const&)
\end{cpp}

是在类定义内部定义的,其隐式地认为是一个内联函数,因此我们决定将实现委托给一个不需要内联的静态成员函数areEqual。

自1994年以来,友元函数的名称查找已经发生了变化,因此Barton-Nackman技巧在标准C++中并不那么有用。在它发明的时候,友元声明在类模板的外围作用域中可见,这个模板是通过一个名为友元名称注入的过程进行实例化。标准C++通过依赖于参数的查找来查找友元函数声明(具体细节请参阅第13.2.2节),从而函数调用的至少一个参数,必须已含友元函数作为关联类的类。若参数类的类型不相关,可以转换为包含友元的类,则无法找到友元函数。例如:

\filename{inherit/wrapper.cpp}
\begin{cpp}
class S {
};

template<typename T>
class Wrapper {
	private:
	T object;
	public:
	Wrapper(T obj) : object(obj) { // implicit conversion from T to Wrapper<T>
	}
	friend void foo(Wrapper<T> const&) {
	}
};

int main()
{
	S s;
	Wrapper<S> w(s);
	foo(w); // OK: Wrapper<S> is a class associated with w
	foo(s); // ERROR: Wrapper<S> is not associated with s
}
\end{cpp}

这里,foo(w)是有效的,因为函数foo()是在Wrapper<S>中声明的友元,它是一个与参数w相关联的类。

\begin{notice}S也是一个与w相关的类,因为它是w类型的模板参数。ADL的具体规则在第13.2.1节中讨论。
\end{notice}

在调用foo(s)中,因为定义它的类Wrapper<S>与类型S的参数s不相关,所以函数foo(Wrapper<S> const\&)的友元声明不可见。即使存在有效的从类型s到类型Wrapper<S>的隐式转换(通过Wrapper<S>的构造函数),因为候选函数foo()在第一个地方没有找到,所以这种转换从未进行考虑。在Barton和Nackman发明他们的技巧时,友元名称注入将使友元函数foo()可见,并且调用foo(s)会成功。

现代C++中,在类模板中定义友元函数比简单地定义普通函数模板的优势是语法:友元函数定义可以访问其外围类的私有和受保护成员,并且不需要重述外围类模板的所有模板参数。但友元函数定义在与CRPT结合使用时可能会更好,如下面一节描述的操作符实现所示。

\subsection{实现操作符}

实现提供重载操作符的类时,通常会为许多不同(但相关)的操作符提供重载。实现相等操作符(==)的类,可能还会实现不等操作符(!=),而实现小于操作符(<)的类,也可能实现其他关系操作符(>,<=,>=)。这些操作符中只需要定义一个,而其他操作符可以使用这个操作符定义。类X的不等操作符可能定义为相等操作符:

\begin{cpp}
bool operator!= (X const& x1, X const& x2) {
	return !(x1 == x2);
}
\end{cpp}

考虑到有大量具有类似!=定义的类型,很容易将其推广到模板中:

\begin{cpp}
template<typename T>
bool operator!= (T const& x1, T const& x2) {
	return !(x1 == x2);
}
\end{cpp}

C++标准库中也包含类似的定义,作为<utility>头文件的一部分。当确定它们在命名空间std中可用时会导致问题时,这些定义(对于!=、>、<=和>=)在标准化期间降级到命名空间std::rel\_ops中。实际上,让这些定义可见会使类型看起来都有一个!=操作符(可能无法实例化),并且该操作符对其两个参数总是精确匹配。虽然第一个问题可以通过使用SFINAE技术(请参阅第19.4节)来克服,这样!=定义将只对具有适当的==操作符的类型进行实例化,但第二个问题仍然存在:上面的!=定义将优于用户提供的定义,这些定义需要进行派生到基的转换,这就很诡异了。

基于CRTP的这些操作符模板的另一种表达方式允许类选择通用操作符定义,提供了增加代码重用的优势,而没有过度通用操作符的副作用:

\filename{inherit/equalitycomparable.cpp}
\begin{cpp}
template<typename Derived>
class EqualityComparable
{
	public:
	friend bool operator!= (Derived const& x1, Derived const& x2) {
		return !(x1 == x2);
	}
};

class X : public EqualityComparable<X>
{
	public:
	friend bool operator== (X const& x1, X const& x2) {
		// implement logic for comparing two objects of type X
	}
};

int main()
{
	X x1, x2;
	if (x1 != x2) { }
}
\end{cpp}

这里,结合了CRTP和Barton-Nackman技巧!EqualityComparable<>使用CRTP提供一个操作符!根据派生类对operator==的定义,为其指定派生类。实际上,通过友元函数定义(Barton-Nackman技巧)提供了该定义,将两个参数提供给operator!=转换的等价行为。

当将行为分解到基类中同时,保留最终派生类的标识时,CRTP可能很有用。除了Barton-Nackman技巧之外,CRTP还可以基于一些规范操作符提供一些操作符的一般定义。这些特性使Barton-Nackman技巧的CRTP,成为C++模板库作者最喜欢的技术之一。

\subsection{门面模式}

使用CRTP和Barton-Nackman技巧来定义一些操作符是一种方便的快捷方式。可以进一步采用这种思想,这样CRTP基类就可以用CRTP派生类公开的一个小得多(但更容易实现)的接口来定义类的大部分或所有公共接口。这种模式称为门面模式,可以定义需要满足某些现有接口数字类型、迭代器、容器等要求的新类型。

为了说明门面模式,我们将为迭代器实现一个门面模式,这将简化编写(符合标准库要求的)迭代器的过程。迭代器类型(特别是随机访问迭代器)所需的接口相当大。下面是类模板IteratorFacade的框架,演示了迭代器接口的需求:

\filename{inherit/iteratorfacadeskel.hpp}
\begin{cpp}
template<typename Derived, typename Value, typename Category,
		typename Reference = Value&, typename Distance = std::ptrdiff_t>
class IteratorFacade
{
	public:
	using value_type = typename std::remove_const<Value>::type;
	using reference = Reference;
	using pointer = Value*;
	using difference_type = Distance;
	using iterator_category = Category;
	
	// input iterator interface:
	reference operator *() const { ... }
	pointer operator ->() const { ... }
	Derived& operator ++() { ... }
	Derived operator ++(int) { ... }
	friend bool operator== (IteratorFacade const& lhs,
						    IteratorFacade const& rhs) { ... }
	...
	
	// bidirectional iterator interface:
	Derived& operator --() { ... }
	Derived operator --(int) { ... }
	
	// random access iterator interface:
	reference operator [](difference_type n) const { ... }
	Derived& operator +=(difference_type n) { ... }
	...
	friend difference_type operator -(IteratorFacade const& lhs,
									  IteratorFacade const& rhs) { ... }
	friend bool operator <(IteratorFacade const& lhs,
						   IteratorFacade const& rhs) { ... }
	...
};
\end{cpp}

简洁起见,省略了一些声明,即使实现每个新迭代器列出的所有声明,也相当繁琐。幸运的是,该接口可以总结为几个核心操作:

\begin{itemize}
\item 
对于所有迭代器:
\begin{itemize}
\item [-]
dereference():访问迭代器指向的值(通常通过操作符*和\texttt{->}使用)。

\item [-]
increment():移动迭代器指向序列中的下一项。

\item [-]
equals():确定两个迭代器是否指向序列中的同一项。
\end{itemize}

\item 
双向迭代器:
\begin{itemize}
\item [-]
decrement():移动迭代器以指向列表中的前一项。
\end{itemize}

\item 
随机访问迭代器:
\begin{itemize}
\item [-]
advance():将迭代器向前(或向后)移动n步。

\item [-]
measureDistance():确定序列中从一个迭代器到另一个迭代器的步数。
\end{itemize}
\end{itemize}

门面是调整一个只实现那些核心操作的类型,以提供完整的迭代器接口。IteratorFacade的实现主要涉及将迭代器语法映射到最小的接口。下面的例子中,使用成员函数asDerived()来访问CRTP派生类:

\begin{cpp}
Derived& asDerived() { return *static_cast<Derived*>(this); }
Derived const& asDerived() const {
	return *static_cast<Derived const*>(this);
}
\end{cpp}

根据该定义,门面的大部分实现都很简单。

\begin{notice}为了简化表示,忽略代理迭代器的存在,因为代理迭代器的解引用操作不会返回真正的引用。迭代器外观的完整实现(如[BoostIterator]中的实现)将调整操作符的结果类型\texttt{—>}和操作符[]以考虑代理。
\end{notice}

这里,只举例说明输入迭代器的一些要求的定义,其他迭代器的情况也相同。

\begin{cpp}
reference operator*() const {
	return asDerived().dereference();
}
Derived& operator++() {
	asDerived().increment();
	return asDerived();
}
Derived operator++(int) {
	Derived result(asDerived());
	asDerived().increment();
	return result;
}
friend bool operator== (IteratorFacade const& lhs,
IteratorFacade const& rhs) {
	return lhs.asDerived().equals(rhs.asDerived());
}
\end{cpp}

\subsubsection{定义链表迭代器}

通过IteratorFacade的定义,可以将迭代器定义为简单的链表类。假设这样定义链表中的节点:

\filename{inherit/listnode.hpp}
\begin{cpp}
template<typename T>
class ListNode
{
	public:
	T value;
	ListNode<T>* next = nullptr;
	~ListNode() { delete next; }
};
\end{cpp}

使用IteratorFacade,可以以一种简单的方式定义链表迭代器:

\filename{inherit/listnodeiterator0.hpp}
\begin{cpp}
template<typename T>
class ListNodeIterator
: public IteratorFacade<ListNodeIterator<T>, T,
std::forward_iterator_tag>
{
	ListNode<T>* current = nullptr;
	public:
	T& dereference() const {
		return current->value;
	}
	void increment() {
		current = current->next;
	}
	bool equals(ListNodeIterator const& other) const {
		return current == other.current;
	}
	ListNodeIterator(ListNode<T>* current = nullptr) : current(current) { }
};
\end{cpp}

ListNodeIterator提供了充当前向迭代器所需的所有正确操作符和嵌套类型,并且只需要很少的代码来实现。定义更复杂的迭代器(例如,随机访问迭代器)只需要少量的工作。

\subsubsection{隐藏接口}

ListNodeIterator实现的缺点是,需要作为公共接口公开操作dereference()、advance()和equals()。为了消除这个需求,可以重写IteratorFacade,通过单独的访问类IteratorFacadeAccess,在派生的CRTP类上执行所有操作:

\filename{inherit/iteratorfacadeaccessskel.hpp}
\begin{cpp}
// ‘friend’ this class to allow IteratorFacade access to core iterator operations:
class IteratorFacadeAccess
{
	// only IteratorFacade can use these definitions
	template<typename Derived, typename Value, typename Category,
			typename Reference, typename Distance>
	friend class IteratorFacade;
	
	// required of all iterators:
	template<typename Reference, typename Iterator>
	static Reference dereference(Iterator const& i) {
		return i.dereference();
	}
	...
	// required of bidirectional iterators:
	template<typename Iterator>
	static void decrement(Iterator& i) {
		return i.decrement();
	}

	// required of random-access iterators:
	template<typename Iterator, typename Distance>
	static void advance(Iterator& i, Distance n) {
		return i.advance(n);
	}
	...
};
\end{cpp}

该类为每个核心迭代器操作提供静态成员函数,调用所提供的迭代器对应的(非静态)成员函数。所有的静态成员函数都是私有的,只授予IteratorFacade本身访问权。因此,ListNodeIterator可以将IteratorFacadeAccess设为友元,并对门面所需的接口保持私有:

\begin{cpp}
friend class IteratorFacadeAccess;
\end{cpp}

\subsubsection{迭代器适配器}

IteratorFacade可以很容易地构建一个迭代器适配器,可接受一个现有的迭代器,并公开一个新的迭代器,以提供底层序列的转换视图。例如,可能有一个Person值的容器:

\filename{inherit/person.hpp}
\begin{cpp}
struct Person {
	std::string firstName;
	std::string lastName;
	
	friend std::ostream& operator<<(std::ostream& strm, Person const& p) {
		return strm << p.lastName << ", " << p.firstName;
	}
};
\end{cpp}

但不需要遍历容器中的所有Person值,而只希望看到名字。本节中,将开发一个名为ProjectionIterator的迭代器适配器,会将底层(基)迭代器的值“投射”到一些指向数据的成员,例如Person::firstName。

ProjectionIterator是根据基本迭代器(iterator)和将由迭代器(T)公开值类型定义的迭代器:

\filename{inherit/projectioniteratorskel.hpp}
\begin{cpp}
template<typename Iterator, typename T>
class ProjectionIterator
	: public IteratorFacade<
			ProjectionIterator<Iterator, T>,
			T,
			typename std::iterator_traits<Iterator>::iterator_category,
			T&,
			typename std::iterator_traits<Iterator>::difference_type>
{
	using Base = typename std::iterator_traits<Iterator>::value_type;
	using Distance =
		typename std::iterator_traits<Iterator>::difference_type;
	
	Iterator iter;
	T Base::* member;
	
	friend class IteratorFacadeAccess;
	... // implement core iterator operations for IteratorFacade
	public:
	ProjectionIterator(Iterator iter, T Base::* member)
		: iter(iter), member(member) { }
};

template<typename Iterator, typename Base, typename T>
auto project(Iterator iter, T Base::* member) {
	return ProjectionIterator<Iterator, T>(iter, member);
}
\end{cpp}

每个投射迭代器存储两个值:iter和member,前者是底层序列(Base值)的迭代器,后者是指向数据的成员,描述投射哪个成员。可以考虑为IteratorFacade的基类提供模板参数:第一个是ProjectionIterator(用于启用CRTP),第二个(T)和第四个(T\&)参数是投射迭代器的值和引用类型,将其定义为一个由T值组成的序列。

\begin{notice}为了简化示例,假定底层迭代器返回的是引用,而不是代理。
\end{notice}

第三和第五个参数仅传递底层迭代器的类别和不同类型。当迭代器是输入迭代器时,投射迭代器将是输入迭代器;当iterator是双向迭代器时,投射迭代器将是双向迭代器,以此类推。project()函数可以方便地构建投射迭代器。

唯一缺少的部分是IteratorFacade核心需求的实现。最有趣的是dereference(),它对底层迭代器进行解引用,然后通过指向数据的指针成员进行投射:

\begin{cpp}
T& dereference() const {
	return (*iter).*member;
}
\end{cpp}

剩下的操作由底层迭代器实现:

\begin{cpp}
void increment() {
	++iter;
}
bool equals(ProjectionIterator const& other) const {
	return iter == other.iter;
}
void decrement() {
	--iter;
}
\end{cpp}

简单起见,这里省略了随机访问迭代器的定义,其类似于下面的定义。

就是这样!使用投射迭代器,可以打印出vector中Person对象的名字:

\filename{inherit/projectioniterator.cpp}
\begin{cpp}
#include <vector>
#include <algorithm>
#include <iterator>

int main()
{
	std::vector<Person> authors = { {"David", "Vandevoorde"},
		{"Nicolai", "Josuttis"},
		{"Douglas", "Gregor"} };
	std::copy(project(authors.begin(), &Person::firstName),
	project(authors.end(), &Person::firstName),
	std::ostream_iterator<std::string>(std::cout, "\n"));
}
\end{cpp}

这个程序会输出:

\begin{shell}
David
Nicolai
Douglas
\end{shell}

门面模式对于创建符合某些特定接口的新类型特别有用。新类型只需要向门面公开少量的核心操作(门面迭代器的核心操作在3到6个之间),门面负责使用CRTP和Barton-Nackman技巧的组合,并提供完整而正确的公共接口即可。












