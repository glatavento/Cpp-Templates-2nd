\section{零值初始化}
对于基本类型,如int、double或指针类型,没有默认构造函数可以初始化。相反,任何未初始化的局部变量都有一个未定义的值:

\begin{cpp}
void foo()
{
	int x; // x has undefined value
	int* ptr; // ptr points to anywhere (instead of nowhere)
}
\end{cpp}

若编写模板,并想让模板类型的变量用默认值初始化,就会遇到一个问题,简单的定义对内置类型并没有进行初始化:

\begin{cpp}
template<typename T>
void foo()
{
	T x; // x has undefined value if T is built-in type
}
\end{cpp}

因此,可以显式调用内置类型的默认构造函数,该构造函数用0初始化内置类型(bool为false,指针为nullptr)。因此,即使是内置类型,也可以通过编写以下代码来确保正确的初始化:

\begin{cpp}
template<typename T>
void foo()
{
	T x{}; // x is zero (or false or nullptr) if T is a built-in type
}
\end{cpp}

这种初始化方式称为值初始化,要么调用提供的构造函数,要么对对象进行零初始化。即使构造函数是显式的,也可以这样做。

C++11前,正确初始化的语法是

\begin{cpp}
T x = T(); // x is zero (or false or nullptr) if T is a built-in type
\end{cpp}

C++17前,这种机制(现在仍受支持)只有在为复制初始化选择的构造函数不是显式的情况下才有效。在C++17中,省略了强制复制,从而避免了这种限制,而且两种语法都有效。但若没有默认构造函数,带大括号的初始化表示法可以使用初始化列表构造函数。

\begin{notice}
对于某种类型X,有参数类型为std::initializer\_list<X>的构造函数。
\end{notice}

为了确保将类型参数化的类模板成员初始化,可以定义默认构造函数,使用带大括号的初始化式来初始化成员:

\begin{cpp}
template<typename T>
class MyClass {
private:
	T x;
public:
	MyClass() : x{} { // ensures that x is initialized even for built-in types
	}
	...
};
\end{cpp}

C++11前的语法

\begin{cpp}
MyClass() : x() { // ensures that x is initialized even for built-in types
}
\end{cpp}

在后续标准中使用,仍然有效。

C++11后,还可以为非静态成员提供默认初始化,这样也可以实现以下操作:

\begin{cpp}
template<typename T>
class MyClass {
private:
	T x{}; // zero-initialize x unless otherwise specified
	...
};
\end{cpp}

但是,请注意默认参数不能使用该语法。例如,

\begin{cpp}
template<typename T>
void foo(T p{}) { // ERROR
	...
}
\end{cpp}

必须这样写:

\begin{cpp}
template<typename T>
void foo(T p = T{}) { // OK (must use T() before C++11)
	...
}
\end{cpp}





















