\section{模板参数类型auto}
C++17后,可以定义非类型模板参数。使用这个特性,可以实现一个确定大小的堆栈类:

\filename{basics/stackauto.hpp}
\begin{cpp}
#include <array>
#include <cassert>

template<typename T, auto Maxsize>
class Stack {
	public:
	using size_type = decltype(Maxsize);
private:
	std::array<T,Maxsize> elems; // elements
	size_type numElems; // current number of elements
public:
	Stack(); // constructor
	void push(T const& elem); // push element
	void pop(); // pop element
	T const& top() const; // return top element
	bool empty() const { // return whether the stack is empty
		return numElems == 0;
	}
	size_type size() const { // return current number of elements
		return numElems;
	}
};

// constructor
template<typename T, auto Maxsize>
Stack<T,Maxsize>::Stack ()
: numElems(0) { // start with no elements
	// nothing else to do
}

template<typename T, auto Maxsize>
void Stack<T,Maxsize>::push (T const& elem) {
	assert(numElems < Maxsize);
	elems[numElems] = elem; // append element
	++numElems; // increment number of elements
}

template<typename T, auto Maxsize>
void Stack<T,Maxsize>::pop () {
	assert(!empty());
	--numElems; // decrement number of elements
}

template<typename T, auto Maxsize>
T const& Stack<T,Maxsize>::top () const {
	assert(!empty());
	return elems[numElems-1]; // return last element
}
\end{cpp}

定义

\begin{cpp}
template<typename T, auto Maxsize>
class Stack {
	...
};
\end{cpp}

通过使用占位符类型auto,可以将Maxsize定义为尚未指定类型的值,可以是任何允许为非类型模板参数的类型。

内部可以同时使用这两个值:

\begin{cpp}
std::array<T,Maxsize> elems; // elements
\end{cpp}

其类型为:

\begin{cpp}
using size_type = decltype(Maxsize);
\end{cpp}

例如,用作size()成员函数的返回类型:

\begin{cpp}
size_type size() const { // return current number of elements
	return numElems;
}
\end{cpp}

C++14后,也可以在这里使用auto作为返回类型,让编译器确定返回类型:

\begin{cpp}
auto size() const { // return current number of elements
	return numElems;
}
\end{cpp}

通过这个类声明,当使用堆栈时,元素数量由使用类型定义:

\filename{basics/stackauto.cpp}
\begin{cpp}
#include <iostream>
#include <string>
#include "stackauto.hpp"

int main() {
	Stack<int,20u> int20Stack; // stack of up to 20 ints
	Stack<std::string,40> stringStack; // stack of up to 40 strings
	
	// manipulate stack of up to 20 ints
	int20Stack.push(7);
	std::cout << int20Stack.top() << '\n';
	auto size1 = int20Stack.size();
	
	// manipulate stack of up to 40 strings
	stringStack.push("hello");
	std::cout << stringStack.top() << '\n';
	auto size2 = stringStack.size();
	
	if (!std::is_same_v<decltype(size1), decltype(size2)>) {
		std::cout << "size types differ" << '\n';
	}
}
\end{cpp}


\begin{cpp}
Stack<int,20u> int20Stack; // stack of up to 20 ints
\end{cpp}

因为传递了20u,所以内部大小类型是unsigned int。

\begin{cpp}
Stack<std::string,40> stringStack; // stack of up to 40 strings
\end{cpp}

因为传递了40,内部大小类型是int。

两个堆栈的size()将有不同的返回类型,因此

\begin{cpp}
auto size1 = int20Stack.size();
...
auto size2 = stringStack.size();
\end{cpp}

size1和size2的类型不同。可以使用标准类型特征std::is\_same(参见D.3.3节)和decltype,进行检查:

\begin{cpp}
if (!std::is_same<decltype(size1), decltype(size2)>::value) {
	std::cout << "size types differ" << '\n';
}
\end{cpp}

因此,输出是:

\begin{shell}
size types differ
\end{shell}

C++17后,对于返回值可以使用后缀\_v并省略::value(详见5.6节):

\begin{cpp}
if (!std::is_same_v<decltype(size1), decltype(size2)>) {
	std::cout << "size types differ" << '\n';
}
\end{cpp}

对非类型模板参数类型的其他约束仍然有效。特别是,第3.3节中讨论的非类型模板参数类型的限制仍然适用。例如:

\begin{cpp}
Stack<int,3.14> sd; // ERROR: Floating-point nontype argument
\end{cpp}

并且,可以将字符串作为常量数组传递(C++17中甚至是静态的局部声明;见第3.3节):

\filename{basics/message.cpp}
\begin{cpp}
#include <iostream>

template<auto T> // take value of any possible nontype parameter (since C++17)
class Message {
	public:
	void print() {
		std::cout << T << '\n';
	}
};

int main() {
	Message<42> msg1;
	msg1.print(); // initialize with int 42 and print that value
	
	static char const s[] = "hello";
	Message<s> msg2; // initialize with char const[6] "hello"
	msg2.print(); // and print that value
}
\end{cpp}

还要注意的是,template<decltype(auto) N>也可以,其允许将N实例化为引用:

\begin{cpp}
template<decltype(auto) N>
class C {
	...
};

int i;
C<(i)> x; // N is int&
\end{cpp}

详情参见15.10.1。




