
来讨论一下引用传递的不同风格。所有情况下,都不会创建副本(因为参数只引用传递的实参)。另外,传递参数永不衰变。然而,有些传递是不可行的,若传递对这种参数进行了,在某些情况下,参数的结果类型可能会出现问题。

\subsection{传递常量引用}

为了避免(不必要的)复制,传递非临时对象时,可以使用常量引用。例如:

\begin{cpp}
template<typename T>
void printR (T const& arg) {
	...
}
\end{cpp}

通过这样的声明,传递的对象永远不会创建副本:

\begin{cpp}
std::string returnString();
std::string s = "hi";
printR(s); // no copy
printR(std::string("hi")); // no copy
printR(returnString()); // no copy
printR(std::move(s)); // no copy
\end{cpp}

即使是int也是通过引用传递的,这有点适得其反,但并不重要。

因此:

\begin{cpp}
int i = 42;
printR(i); // passes reference instead of just copying i
\end{cpp}

printR()会实例化为:

\begin{cpp}
void printR(int const& arg) {
	...
}
\end{cpp}

底层实现中,通过引用传递参数是通过传递参数地址实现的。地址编码紧凑时,将地址从调用方传递给被调用方的效率很高。然而,在编译代码时,传递地址会给编译器带来不确定性:使用该地址做什么?理论上,可以对该地址“可达”的值进行修改。因此,编译器必须假设其可能缓存的值(通常在机器寄存器中)在调用之后都无效。重新加载这些值的成本可能非常高。有些读者可能会认为我们传递的是常量引用:难道编译器不能推断出这样的数值其实不会发生更改的吗?不幸的是,情况并非如此,因为调用者可以通过自己的非常量引用修改引用的对象。

\begin{notice}使用const\_cast是另一种显式修改引用对象的方法。
\end{notice}

内联可以缓和一下:如果编译器可以内联展开调用,就可以推断调用和被调用在一起,并且在许多情况下“看到”地址,除了传递底层值外,没有其他用途。函数模板通常很短,因此可以用于内联扩展。但若模板封装了更复杂的算法,就不太可能使用内联了。

\subsubsection{传递引用类型不会衰变}

当通过引用将参数进行传递时,就不会衰变。从而不会将数组转换为指针,并且不删除const和volatile等限定符。因为调用参数声明为T const\&,所以模板参数T本身并没有推导为const。例如:

\begin{cpp}
template<typename T>
void printR (T const& arg) {
	...
}

std::string const c = "hi";
printR(c); // T deduced as std::string, arg is std::string const&

printR("hi"); // T deduced as char[3], arg is char const(&)[3]

int arr[4];
printR(arr); // T deduced as int[4], arg is int const(&)[4]
\end{cpp}

因此,printR()中使用T类型声明的局部对象不是常量。

\subsection{传递非常量引用}

当通过传递的参数作为返回值时(例如,使用out或inout参数时),必须使用非常量引用(除非通过指针传递)。在传递参数时,不会创建副本,被调用函数模板的参数只能直接访问传递的参数。

来看看以下代码:

\begin{cpp}
template<typename T>
void outR (T& arg) {
	...
}
\end{cpp}

通常不允许对一个临时的(prvalue)或一个通过std::move()(xvalue)传递的现有对象调用outR():

\begin{cpp}
std::string returnString();
std::string s = "hi";
outR(s); // OK: T deduced as std::string, arg is std::string&
outR(std::string("hi")); // ERROR: not allowed to pass a temporary (prvalue)
outR(returnString()); // ERROR: not allowed to pass a temporary (prvalue)
outR(std::move(s)); // ERROR: not allowed to pass an xvalue
\end{cpp}

可以传递非常量类型的数组,也不会衰变:

\begin{cpp}
int arr[4];
outR(arr); // OK: T deduced as int[4], arg is int(&)[4]
\end{cpp}

因此,可以修改元素(处理数组的大小)。例如:

\begin{cpp}
template<typename T>
void outR (T& arg) {
	if (std::is_array<T>::value) {
		std::cout << "got array of " << std::extent<T>::value << " elems\n";
	}
	...
}
\end{cpp}

然而,模板在这里有点棘手。若传递const参数,可能导致arg变成一个常量引用的声明,这意味着允许传递右值,但这里需要左值:

\begin{cpp}
std::string const c = "hi";
outR(c); // OK: T deduced as std::string const
outR(returnConstString()); // OK: same if returnConstString() returns const string
outR(std::move(c)); // OK: T deduced as std::string const
outR("hi"); // OK: T deduced as char const[3]
\end{cpp}

\begin{notice}当传递std::move(c)时,std::move()首先将c转换为std::string const\&\&,然后将T推导为std::string const。
\end{notice}

这种情况下,修改函数模板中传递的参数是错误的。在表达式中传递常量对象是可能的,但当函数完全实例化时(这可能发生在编译的后期),修改该值的尝试都将触发错误(然而,这可能发生在被调用模板的内部;见9.4节)。

如果想禁用将常量对象传递给非常量引用,可以这样做:

\begin{itemize}
\item 
使用静态断言触发编译时错误:

\begin{cpp}
template<typename T>
void outR (T& arg) {
	static_assert(!std::is_const<T>::value,
				  "out parameter of foo<T>(T&) is const");
	...
}
\end{cpp}

\item 
使用std::enable\_if<>(参见章节6.3)禁用此模板:

\begin{cpp}
template<typename T,
		 typename = std::enable_if_t<!std::is_const<T>::value>
void outR (T& arg) {
	...
}
\end{cpp}

或概念,若支持(见章节6.5和附录E):

\begin{cpp}
template<typename T>
requires !std::is_const_v<T>
void outR (T& arg) {
	...
}
\end{cpp}

\end{itemize}

\subsection{通过转发引用进行传递}

使用引用调用的原因是能够完美地转发参数(参见第6.1节)。但当使用转发引用(定义为模板形参的右值引用)时,需要使用特殊的规则。考虑以下代码:

\begin{cpp}
template<typename T>
void passR (T&& arg) { // arg declared as forwarding reference
	...
}
\end{cpp}

可以把所有的东西都传递给转发引用,和通过引用传递一样,不会创建副本:

\begin{cpp}
std::string s = "hi";
passR(s); // OK: T deduced as std::string& (also the type of arg)
passR(std::string("hi")); // OK: T deduced as std::string, arg is std::string&&
passR(returnString()); // OK: T deduced as std::string, arg is std::string&&
passR(std::move(s)); // OK: T deduced as std::string, arg is std::string&&
passR(arr); // OK: T deduced as int(&)[4] (also the type of arg)
\end{cpp}

但是,类型推导的特殊规则可能会导致一些意外发生:

\begin{cpp}
std::string const c = "hi";
passR(c); // OK: T deduced as std::string const&
passR("hi"); // OK: T deduced as char const(&)[3] (also the type of arg)
int arr[4];
passR(arr); // OK: T deduced as int (&)[4] (also the type of arg)
\end{cpp}

每一种情况下,passR()内部的参数arg有一个类型,“知道”传递的是右值(使用移动语义)还是常量/非常量左值。这是传递参数的唯一方法,这样就可以用来区分这三种情况的行为。

这给人的印象是,将参数声明为转发引用几乎完美。但注意,天下没有免费的午餐。

例如,模板参数T隐式地成为引用类型的唯一情况。因此,使用T声明局部对象而不进行初始化可能会出错:

\begin{cpp}
template<typename T>
void passR(T&& arg) { // arg is a forwarding reference
	T x; // for passed lvalues, x is a reference, which requires an initializer
	...
}

passR(42); // OK: T deduced as int
int i;
passR(i); // ERROR: T deduced as int&, which makes the declaration of x in passR() invalid
\end{cpp}

有关如何处理这种情况的详细信息,请参阅第15.6.2节。




















