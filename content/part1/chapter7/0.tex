从一开始,C++就提供了按值和按引用调用,但要决定选择哪一种并不那么容易:通常按引用调用对于重要的对象来说成本更低,但更复杂。C++11添加了移动语义,现在有了不同的通过引用传递的方式:

\begin{tcolorbox}[colback=webgreen!5!white,colframe=webgreen!75!black]
\hspace*{0.75cm}常量右值引用X const\&\&也可以,但没有既定的语义。
\end{tcolorbox}

\begin{enumerate}
\item 
X const\& (常量左值引用):

参数引用传递的对象,但不能修改。

\item 
X\& (非常数的左值引用):

参数引用传递的对象,并能够修改。

\item 
X\&\& (右值引用):

参数引用传递的对象,带有移动语义,可以修改或“窃取”值。
\end{enumerate}

决定如何声明已知具体类型的参数已经够复杂的了。模板中类型是未知的,因此很难决定哪种传递机制更为合适。

第1.6.1节中,我们确实建议在函数模板中按值传递参数,除非有很好的理由:

\begin{itemize}
\item 
不能复制

\begin{tcolorbox}[colback=webgreen!5!white,colframe=webgreen!75!black]
\hspace*{0.75cm}由于C++17,即使没有可用的复制或移动构造函数,也可以通过值传递临时实体(右值)(参见B.2.1节)。因此,由于C++17的约束,不可能复制左值。
\end{tcolorbox}

\item 
参数用于返回数据。

\item 
模板只是通过保留原始参数的所有属性,将参数转发到其他地方

\item 
有显著的性能改进
\end{itemize}

本章讨论了在模板中声明参数的不同方法,提出了通过值传递参数的建议,并为不这样做的原因提供了参数。本文还讨论了在处理字符串字面值和其他数组时遇到的棘手问题。

阅读本章时,熟悉值类别(lvalue、rvalue、prvalue、xvalue等)的术语对理解本章内容会有帮助,这些在附录B中都有解释。





























