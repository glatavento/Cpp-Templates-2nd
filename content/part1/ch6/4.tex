\section{使用enable\_if<>}
可以使用enable\_if<>来解决在6.2节中引入的构造函数模板的问题。

必须解决的问题是,禁用模板构造函数的声明

\begin{cpp}
template<typename STR>
Person(STR&& n);
\end{cpp}

若传递的参数STR具有正确的类型(即std::string,或可转换为std::string的类型)。

为此,使用另一个标准类型std::is\_convertible<FROM,TO>。C++17中,相应的声明如下所示:

\begin{cpp}
template<typename STR,
	typename = std::enable_if_t<
		std::is_convertible_v<STR, std::string>>>
Person(STR&& n);
\end{cpp}

如果STR类型可转换为std::string类型,则整个声明展开为

\begin{cpp}
template<typename STR,
	typename = void>
Person(STR&& n);
\end{cpp}

如果STR类型不能转换为std::string类型,将忽略整个函数模板。

\begin{notice}
为什么不检查STR是否“不可转换为Person”?我们正在定义一个函数,该函数可能允许将字符串转换为Person。因此构造函数必须知道它是否启用,这取决于是否可转换,而可转换取决于是否启用,循环往复。永远不要在影响enable\_if使用的条件的地方使用enable\_if。这是编译器不一定能检测到的逻辑错误。
\end{notice}

同样,可以使用别名模板定义名称:

\begin{cpp}
template<typename T>
using EnableIfString = std::enable_if_t<
						std::is_convertible_v<T, std::string>>;
...
template<typename STR, typename = EnableIfString<STR>>
Person(STR&& n);
\end{cpp}

因此,Person类如下所示:

\filename{basics/specialmemtmpl3.hpp}
\begin{cpp}
#include <utility>
#include <string>
#include <iostream>
#include <type_traits>

template<typename T>
using EnableIfString = std::enable_if_t<
						std::is_convertible_v<T,std::string>>;

class Person
{
private:
	std::string name;
public:
	// generic constructor for passed initial name:
	template<typename STR, typename = EnableIfString<STR>>
	explicit Person(STR&& n)
	: name(std::forward<STR>(n)) {
		std::cout << "TMPL-CONSTR for '" << name << "'\n";
	}

	// copy and move constructor:
	Person (Person const& p) : name(p.name) {
		std::cout << "COPY-CONSTR Person '" << name << "'\n";
	}
	Person (Person&& p) : name(std::move(p.name)) {
		std::cout << "MOVE-CONSTR Person '" << name << "'\n";
	}
};
\end{cpp}

现在,所有行为都符合预期:

\filename{basics/specialmemtmpl3.cpp}
\begin{cpp}
#include "specialmemtmpl3.hpp"

int main()
{
	std::string s = "sname";
	Person p1(s); // init with string object => calls TMPL-CONSTR
	Person p2("tmp"); // init with string literal => calls TMPL-CONSTR
	Person p3(p1); // OK => calls COPY-CONSTR
	Person p4(std::move(p1)); // OK => calls MOVE-CONST
}
\end{cpp}

C++14中,必须如下声明别名模板,因为\_v版本没有为类型定义value:

\begin{cpp}
template<typename T>
using EnableIfString = std::enable_if_t<
						std::is_convertible<T, std::string>::value>;
\end{cpp}

C++11中,必须声明特殊成员模板,因为\_t版本并没有定义type:

\begin{cpp}
template<typename T>
using EnableIfString
= typename std::enable_if<std::is_convertible<T, std::string>::value
						>::type;
\end{cpp}

但这些现在都隐藏在EnableIfString<>的定义中。

因为它要求类型是隐式可转换的,所以使用std::is\_convertible<>还有另一种选择。通过使用std::is\_constructible<>,允许显式转换用于初始化。然而,在这种情况下,参数的顺序是反的:

\begin{cpp}
template<typename T>
using EnableIfString = std::enable_if_t<
						std::is_constructible_v<std::string, T>>;
\end{cpp}

关于std::is\_constructible<>和std::is\_convertible<>的详细信息请参见D.3.2节。关于在可变参数模板上应用enable\_if<>的详细信息和示例,请参见D.6节。

\subsubsection{禁用特殊成员函数}

注意,通常不能使用enable\_if<>来禁用预定义的复制/移动构造函数和/或赋值操作符。原因是成员函数模板永远不会算作特殊成员函数,并且在需要复制构造函数等情况下会忽略。因此,在此声明:

\begin{cpp}
class C {
public:
	template<typename T>
	C (T const&) {
		std::cout << "tmpl copy constructor\n";
	}
	...
};
\end{cpp}

当请求一个C的副本时,预定义的复制构造函数仍然可以使用:

\begin{cpp}
C x;
C y{x}; // still uses the predefined copy constructor (not the member template)
\end{cpp}

(实际上没有办法使用成员模板,因为没有办法指定或推导它的模板参数T。)

删除预定义的复制构造函数不是解决方案,因为复制C的尝试将导致错误。

不过,有一个解决方案:

可以为const volatile参数声明复制构造函数,并将其标记为“已删除”(即用= delete修饰)。这样做可以防止隐式声明另一个复制构造函数。有了这些,就可以定义一个构造函数模板,对于非volatile类型,该构造函数将优先于(已删除的)复制构造函数:

\begin{cpp}
class C
{
public:
	...
	// user-define the predefined copy constructor as deleted
	// (with conversion to volatile to enable better matches)
	C(C const volatile&) = delete;
	
	// implement copy constructor template with better match:
	template<typename T>
	C (T const&) {
		std::cout << "tmpl copy constructor\n";
	}
...
};
\end{cpp}

现在模板构造函数会用于“通常”的复制:

\begin{cpp}
C x;
C y{x}; // uses the member template
\end{cpp}

在模板构造函数中,可以使用enable\_if<>进行约束。例如,若模板参数是整型,为了防止复制类模板C<>的对象,可以实现以下方法:

\begin{cpp}
template<typename T>
class C
{
public:
	...
	// user-define the predefined copy constructor as deleted
	// (with conversion to volatile to enable better matches)
	C(C const volatile&) = delete;
	// if T is no integral type, provide copy constructor template with better match:
	template<typename U,
	typename = std::enable_if_t<!std::is_integral<U>::value>>
	C (C<U> const&) {
		...
	}
	...
};
\end{cpp}












