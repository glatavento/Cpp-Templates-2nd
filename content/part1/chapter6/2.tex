成员函数模板也可以用作特殊的成员函数,包括用作构造函数,但这可能会导致意想不到的结果。

看看下面的例子:

\filename{basics/specialmemtmpl1.cpp}
\begin{lstlisting}[style=styleCXX]
#include <utility>
#include <string>
#include <iostream>

class Person
{
private:
	std::string name;
public:
	// constructor for passed initial name:
	explicit Person(std::string const& n) : name(n) {
		std::cout << "copying string-CONSTR for '" << name << "'\n";
	}
	explicit Person(std::string&& n) : name(std::move(n)) {
		std::cout << "moving string-CONSTR for '" << name << "'\n";
	}
	// copy and move constructor:
	Person (Person const& p) : name(p.name) {
		std::cout << "COPY-CONSTR Person '" << name << "'\n";
	}
	Person (Person&& p) : name(std::move(p.name)) {
		std::cout << "MOVE-CONSTR Person '" << name << "'\n";
	}
};

int main()
{
	std::string s = "sname";
	Person p1(s); // init with string object => calls copying string-CONSTR
	Person p2("tmp"); // init with string literal => calls moving string-CONSTR
	Person p3(p1); // copy Person => calls COPY-CONSTR
	Person p4(std::move(p1)); // move Person => calls MOVE-CONST
}
\end{lstlisting}

这里,有一个具有string成员名的类Person,我们为其提供了初始化构造函数。为了支持移动语义,重载了接受std::string参数的构造函数:

\begin{itemize}
\item 
提供了string的构造函数,其名称由传递参数的副本进行初始化:

\begin{lstlisting}[style=styleCXX]
Person(std::string const& n) : name(n) {
	std::cout << "copying string-CONSTR for '" << name << "'\n";
}
\end{lstlisting}

\item 
提供了一个可移动string对象的构造函数,使用std::move()来“窃取”下面的值:

\begin{lstlisting}[style=styleCXX]
Person(std::string&& n) : name(std::move(n)) {
	std::cout << "moving string-CONSTR for '" << name << "'\n";
}
\end{lstlisting}
\end{itemize}

正如预期的那样,对于传入的字符串对象(lvalues)调用第一种方法,而对于可移动的对象(rvalues)调用后者:

\begin{lstlisting}[style=styleCXX]
std::string s = "sname";
Person p1(s); // init with string object => calls copying string-CONSTR
Person p2("tmp"); // init with string literal => calls moving string-CONSTR
\end{lstlisting}

除了这些构造函数之外,示例还对复制和移动构造函数有特定的实现,以确定Person作为一个整体何时复制/移动:

\begin{lstlisting}[style=styleCXX]
Person p3(p1); // copy Person => calls COPY-CONSTR
Person p4(std::move(p1)); // move Person => calls MOVE-CONSTR
\end{lstlisting}

现在用一个泛型构造函数替换两个需要传入string的构造函数,完美转发的参数将传递给成员name:

\filename{basics/specialmemtmpl2.hpp}
\begin{lstlisting}[style=styleCXX]
#include <utility>
#include <string>
#include <iostream>

class Person
{
private:
	std::string name;
public:
	// generic constructor for passed initial name:
	template<typename STR>
	explicit Person(STR&& n) : name(std::forward<STR>(n)) {
		std::cout << "TMPL-CONSTR for '" << name << "'\n";
	}

	// copy and move constructor:
	Person (Person const& p) : name(p.name) {
		std::cout << "COPY-CONSTR Person '" << name << "'\n";
	}
	Person (Person&& p) : name(std::move(p.name)) {
		std::cout << "MOVE-CONSTR Person '" << name << "'\n";
	}
};
\end{lstlisting}

传递string的构造函数工作正常:

\begin{lstlisting}[style=styleCXX]
std::string s = "sname";
Person p1(s); // init with string object => calls TMPL-CONSTR
Person p2("tmp"); // init with string literal => calls TMPL-CONSTR
\end{lstlisting}

在这种情况下p2的构造并没有创建临时字符串,参数STR推导为char const[4]。将std::forward<STR>应用于构造函数的指针参数没有太大的效果,因此name成员是由一个以null结尾的字符串(进行构造)。

但当我们试图调用复制构造函数时,会得到一个错误:

\begin{lstlisting}[style=styleCXX]
Person p3(p1); // ERROR
\end{lstlisting}

当用一个可移动对象初始化一个新的Person时,仍然可以正常工作:

\begin{lstlisting}[style=styleCXX]
Person p4(std::move(p1)); // OK: move Person => calls MOVE-CONST
\end{lstlisting}

复制一个常量Person也可以正常工作:

\begin{lstlisting}[style=styleCXX]
Person const p2c("ctmp"); // init constant object with string literal
Person p3c(p2c); // OK: copy constant Person => calls COPY-CONSTR
\end{lstlisting}

问题是,根据C++的重载解析规则(参见16.2.4节),对于一个非常量左值Person p,成员模板

\begin{lstlisting}[style=styleCXX]
template<typename STR>
Person(STR&& n)
\end{lstlisting}

要比(通常是预定义的)复制构造函数匹配的更好:

\begin{lstlisting}[style=styleCXX]
Person (Person const& p)
\end{lstlisting}

STR只是替换为Person\&,而对于复制构造函数,需要转换为const。

可以考虑通过提供一个非常量复制构造函数来解决这个问题:

\begin{lstlisting}[style=styleCXX]
Person (Person& p)
\end{lstlisting}

这只是部分的解决方案,对于派生类的对象,成员模板仍然是更好的匹配。当传递的参数是Person或可以转换为Person的表达式时,这里需要的是禁用成员模板。这可以通过使用std::enable\_if<>来禁用,这将在下一节中介绍。
































