从C++11开始,标准库提供了辅助模板std::enable\_if<>,以在特定的编译时条件下忽略函数模板。

例如,函数模板foo<>()有如下定义:

\begin{lstlisting}[style=styleCXX]
template<typename T>
typename std::enable_if<(sizeof(T) > 4)>::type
foo() {
}
\end{lstlisting}

如果sizeof(T) > 4生成false,则忽略foo<>()的定义。

\begin{tcolorbox}[colback=webgreen!5!white,colframe=webgreen!75!black]
\hspace*{0.75cm}不要忘记将条件放入圆括号中,否则条件中的>将视为模板参数列表的结束。
\end{tcolorbox}

如果sizeof(T) > 4的结果为true,则函数模板实例展开为

\begin{lstlisting}[style=styleCXX]
void foo() {
}
\end{lstlisting}

也就是说,std::enable\_if<>是一种类型特征,计算作为其(第一个)模板参数传递的给定编译时表达式:

\begin{itemize}
\item 
若表达式结果为true,其类型成员类型将产生一个类型:

\begin{itemize}
\item[-]
若没有传递第二个模板参数,则该类型为void。
	
\item[-]
否则,该类型就是第二个模板参数类型。
\end{itemize}

\item 
若表达式结果为false,则没有定义成员类型。由于名为SFINAE的模板特性(替换失败不为过)(请参阅8.4节),这将忽略使用enable\_if表达式的函数模板。
\end{itemize}

对于自C++14产生类型的类型特征,有一个对应的别名模板std::enable\_if\_t<>,允许跳过typename和::type(请参阅第2.8节了解详细信息)。因此,从C++14起就使用

\begin{lstlisting}[style=styleCXX]
template<typename T>
std::enable_if_t<(sizeof(T) > 4)>
foo() {
}
\end{lstlisting}

若第二个参数传递至enable\_if<>或enable\_if\_t<>:

\begin{lstlisting}[style=styleCXX]
template<typename T>
std::enable_if_t<(sizeof(T) > 4), T>
foo() {
	return T();
}
\end{lstlisting}

如果表达式为true,则enable\_if构造展开为第二个参数。因此,若MyType是传递或推导为T的具体类型,其大小大于4,则效果为

\begin{lstlisting}[style=styleCXX]
MyType foo();
\end{lstlisting}

在声明中间使用enable\_if表达式非常笨拙。由于这个原因,使用std::enable\_if<>的常见方法是使用带有默认值的函数模板参数: 

\begin{lstlisting}[style=styleCXX]
template<typename T,
		typename = std::enable_if_t<(sizeof(T) > 4)>>
void foo() {
}
\end{lstlisting}

其会扩展为

\begin{lstlisting}[style=styleCXX]
template<typename T,
		typename = void>
void foo() {
}
\end{lstlisting}

当sizeof(T) > 4时。

若感觉还是太笨拙,并且想让需求/约束更明确,可以使用别名模板为它命名:

\begin{lstlisting}[style=styleCXX]
template<typename T>
using EnableIfSizeGreater4 = std::enable_if_t<(sizeof(T) > 4)>;

template<typename T,
	typename = EnableIfSizeGreater4<T>>
void foo() {
}
\end{lstlisting}

请参阅第20.3节,以了解如何实现std::enable\_if。










