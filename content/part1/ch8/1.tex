\section{模板元编程}
模板在编译时实例化(与动态语言相反,动态语言在运行时处理泛型)。C++模板的一些特性可以与实例化过程结合,成为一种递归“编程语言”。

\begin{notice}
Erwin Unruh通过在编译时计算素数,成为第一个发现编译时计算的人。详见第23.7节。
\end{notice}

因此,模板可以用来“计算”。第23章将详细介绍所有功能,这里只是举一个简单的例子。

下面的代码,在编译时找出给定的数字是否为素数:

\filename{basics/isprime.hpp}
\begin{cpp}
template<unsigned p, unsigned d> // p: number to check, d: current divisor
struct DoIsPrime {
	static constexpr bool value = (p%d != 0) && DoIsPrime<p,d-1>::value;
};

template<unsigned p> // end recursion if divisor is 2
struct DoIsPrime<p,2> {
	static constexpr bool value = (p%2 != 0);
};

template<unsigned p> // primary template
struct IsPrime {
	// start recursion with divisor from p/2:
	static constexpr bool value = DoIsPrime<p,p/2>::value;
};

// special cases (to avoid endless recursion with template instantiation):
template<>
struct IsPrime<0> { static constexpr bool value = false; };
template<>
struct IsPrime<1> { static constexpr bool value = false; };
template<>
struct IsPrime<2> { static constexpr bool value = true; };
template<>
struct IsPrime<3> { static constexpr bool value = true; };
\end{cpp}

IsPrime<>模板返回成员值,无论传递的模板参数p是否是一个素数。为了实现,实例化DoIsPrime<>,其递归地展开为一个表达式,检查p/2和2之间的每个除数d是否能整除p。

例如,表达式为

\begin{cpp}
IsPrime<9>::value
\end{cpp}

扩展为

\begin{cpp}
DoIsPrime<9,4>::value
\end{cpp}

继续扩展

\begin{cpp}
9%4!=0 && DoIsPrime<9,3>::value
\end{cpp}

继续扩展

\begin{cpp}
9%4!=0 && 9%3!=0 && DoIsPrime<9,2>::value
\end{cpp}

继续扩展

\begin{cpp}
9%4!=0 && 9%3!=0 && 9%2!=0
\end{cpp}

计算结果为false,因为9\%3是0。

正如这个实例链所示:

\begin{itemize}
\item 
使用递归展开DoIsPrime<>来遍历从p/2到2的所有除数,以确定这些除数是否能整除给定整数。

\item 
当d等于2时,DoIsPrime<>的偏特化作为结束递归。
\end{itemize}

注意,所有这些都在编译时完成。也就是说,

\begin{cpp}
IsPrime<9>::value
\end{cpp}

在编译时展开为false。

模板语法可以说是笨拙的,但类似的代码自C++98(或更早)就一直有效,并且可以提高运行库的效率。

\begin{notice}
C++11前,通常将值成员声明为枚举数常量,而不是静态数据成员,以避免静态数据成员需要在类外定义(参见第23.6节了解详细信息)。例如:
\begin{cpp}
enum f value = (p%d != 0) && DoIsPrime<p,d-1>::value g;
\end{cpp}
\end{notice}

详见第23章。


























