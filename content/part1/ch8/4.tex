\section{SFINAE(替换失败不为过)}

C++中,以各种参数类型重载的函数很常见。因此,当编译器看到对重载函数的调用时,必须考虑每个候选函数,评估调用参数,并选择最匹配的候选函数(有关此过程的细节,请参阅附录C)。

候选集包括函数模板的情况下,编译器首先必须确定为该候选对象使用哪些模板参数,然后在函数参数列表及其返回类型中替换这些参数,然后评估匹配程度(就像普通函数一样)。但替换过程可能会遇到问题:可能产生毫无意义的构造。语言规则并不认为这种无意义的替换会导致错误,而具有这种问题的候选则会直接忽略。

这里,称这个原则为SFINAE(发音类似于sfee-nay),“替换失败不为过”。

这里描述的替换过程不同于按需实例化过程(请参阅第2.2节):即使是不需要实例化,也可以进行替换(编译器评估是否需要),其会直接替换函数声明(但不是函数体)中的内容。

考虑下面的例子:

\filename{basics/len1.hpp}
\begin{cpp}
// number of elements in a raw array:
template<typename T, unsigned N>
std::size_t len (T(&)[N])
{
	return N;
}

// number of elements for a type having size_type:
template<typename T>
typename T::size_type len (T const& t)
{
	return t.size();
}
\end{cpp}

这里,定义了两个函数模板len(),接受一个泛型参数:

\begin{notice}
没有将该函数命名为size(),避免与C++标准库的命名冲突,C++17引入了标准函数模板std::size()。
\end{notice}

\begin{enumerate}
\item 
第一个函数模板将参数声明为T(\&)[N],从而参数必须是由N个T类型元素组成的数组。

\item 
第二个函数模板将参数声明为T,没有对参数施加任何约束,而是返回类型T::size\_type,这要求传递的参数类型具有size\_type成员变量。
\end{enumerate}

当传递数组或字符串字面量时,只有数组的函数模板匹配:

\begin{cpp}
int a[10];
std::cout << len(a); // OK: only len() for array matches
std::cout << len("tmp"); // OK: only len() for array matches
\end{cpp}

根据签名,第二个函数模板在(分别)用int[10]和char const[4]替换T时也会匹配,这些替换会导致返回类型T::size\_type中出现错误。因此,对于这些调用,第二个模板将会直接忽略。

当传递std::vector<>时,只有第二个函数模板匹配:

\begin{cpp}
std::vector<int> v;
std::cout << len(v); // OK: only len() for a type with size_type matches
\end{cpp}

当传递指针时,两个模板都不匹配(不会失败)。结果,编译器会抱怨没有找到匹配的len()函数:

\begin{cpp}
int* p;
std::cout << len(p); // ERROR: no matching len() function found
\end{cpp}

注意,这与传递一个具有size\_type成员,但没有size()成员函数的类型对象不同,例如:std::allocator<>:

\begin{cpp}
std::allocator<int> x;
std::cout << len(x); // ERROR: len() function found, but can't size()
\end{cpp}

当传递这种类型的对象时,编译器会找到第二个函数模板作为匹配的函数模板。因此,这将导致编译错误,即对std::allocator<int>的调用size()无效,而不是没有找到匹配的len()函数。这一次,没有忽略第二个函数模板。

当替换候选对象的返回类型没有意义时,忽略它会导致编译器选择另一个参数匹配更差的候选对象。例如:

\filename{basics/len2.hpp}
\begin{cpp}
// number of elements in a raw array:
template<typename T, unsigned N>
std::size_t len (T(&)[N])
{
	return N;
}

// number of elements for a type having size_type:
template<typename T>
typename T::size_type len (T const& t)
{
	return t.size();
}

// fallback for all other types:
std::size_t len (...)
{
	return 0;
}
\end{cpp}

这里,还提供了一个通用的len()函数,该函数总是匹配,但其匹配度也是最糟糕的那个(重载解析中使用省略号匹配(…),参见C.2节)。

对于数组和vector,这里有两个匹配,其中特化的匹配更好。对于指针,只有保底候选可以匹配,这样编译器就不会再抱怨这个调用缺少len()了。

\begin{notice}
实践中,保底候选函数通常会提供更有用的默认值、抛出异常或包含静态断言输出更有用的错误消息。
\end{notice}

但对于分配器,第二个和第三个函数模板匹配,第二个函数模板会进行更好的匹配。因此,这会导致无法调用size()成员函数的错误:

\begin{cpp}
int a[10];
std::cout << len(a); // OK: len() for array is best match
std::cout << len("tmp"); // OK: len() for array is best match

std::vector<int> v;
std::cout << len(v); // OK: len() for a type with size_type is best match

int* p;
std::cout << len(p); // OK: only fallback len() matches

std::allocator<int> x;
std::cout << len(x); // ERROR: 2nd len() function matches best,
					 // but can't call size() for x
\end{cpp}

关于SFINAE的更多细节,请参见15.7节和SFINAE应用的章节19.4。

\subsubsection{SFINAE和重载解析}

随着时间的推移,SFINAE原则在模板设计者中变得重要和普遍,以至于它的缩写已经成为一个动词。若打算应用SFINAE机制来确保函数模板在某些约束条件下进行忽略,可以说“SFINAE出了一个函数”,方法是对模板代码进行测试,导致这些约束条件的代码无效。当在C++标准中读到函数模板“不应该参与重载解析,除非…”,这意味着SFINAE去除了这个函数模板。

例如,std::thread类声明了一个构造函数:

\begin{cpp}
namespace std {
class thread {
public:
	...
	template<typename F, typename... Args>
	explicit thread(F&& f, Args&&... args);
	...
};
}
\end{cpp}

附注如下:

备注:如果decay\_t<F>与std::thread类型相同,此构造函数将不参与重载解析。

这意味着如果使用std::thread作为第一个也是唯一的参数调用模板构造函数,将忽略。原因是成员模板有时可能比预定义的复制或移动构造函数更匹配(请参阅6.2节和16.2.4节了解详细信息)。当调用线程时,通过SFINAE输出构造函数模板,确保另一个线程构造一个线程时,总是使用预定义的复制或移动构造函数。

\begin{notice}
由于删除了类线程的复制构造函数,这也禁止了复制。
\end{notice}

这种技术可能很麻烦。幸运的是,标准库提供了更容易禁用模板的工具。这种特性是std::enable\_if<>,在6.3节中介绍。允许通过使用包含禁用条件的构造替换类型来禁用模板。

因此,std::thread的实际声明如下所示:

\begin{cpp}
namespace std {
	class thread {
		public:
		...
		template<typename F, typename... Args,
		typename = enable_if_t<!is_same_v<decay_t<F>,
		thread>>>
		explicit thread(F&& f, Args&&... args);
		...
	};
}
\end{cpp}

请参阅第20.3节了解如何使用偏特化和SFINAE实现std::enable\_if<>的详细信息。


\subsection{SFINAE与decltype}

对于某些条件,找出并制定正确的表达式来SFINAE出函数模板并不容易。

例如,想确保函数模板len()对于具有size\_type成员,但没有size()成员函数的参数类型就会忽略。函数声明中没有对size()成员函数的要求,最终会在实例化时出错:

\begin{cpp}
template<typename T>
typename T::size_type len (T const& t)
{
	return t.size();
}

std::allocator<int> x;
std::cout << len(x) << '\n'; // ERROR: len() selected, but x has no size()
\end{cpp}

有一种常见的模式或习语可以用来处理这种情况:

\begin{itemize}
\item 
使用尾部返回类型语法指定返回类型(前面使用auto,在末尾返回类型之前使用\texttt{->})。

\item 
使用decltype和逗号操作符定义返回类型。

\item 
给出以逗号操作符开头的表达式(在重载逗号操作符时转换为void)。

\item 
在逗号操作符的末尾定义一个实际返回类型的对象。
\end{itemize}

例如:

\begin{cpp}
template<typename T>
auto len (T const& t) -> decltype( (void)(t.size()), T::size_type() )
{
	return t.size();
}
\end{cpp}

返回类型是

\begin{cpp}
decltype( (void)(t.size)(), T::size_type() )
\end{cpp}

decltype构造的操作数是一个逗号分隔的表达式列表,因此最后一个表达式T::size\_type()会产生期望的返回类型的值(decltype使用该值转换为返回类型)。在(最后一个)逗号之前,必须有有效的表达式,本例中就是t.size()。将表达式强制转换为void,是为了避免表达式使用了用户重载定义的逗号操作符。

注意,decltype的参数是一个未计算的操作数,所以可以创建“虚拟对象”而不调用构造函数,这将在第11.2.3节中讨论。























