\section{处理返回值}


对于返回值,还可以决定是通过值返回,还是通过引用返回。但返回引用可能是麻烦的来源,因为引用的东西超出了控制范围。一些情况下,返回引用是常见的编程实践:

\begin{itemize}
\item 
返回容器或字符串元素(例如,通过operator[]或front())

\item 
授予类成员写访问权限

\item 
返回链式调用的对象(流的operator<{}<和operator>{}>,类对象的operator=)
\end{itemize}

此外,通过返回const引用来授予成员读权限。

若使用不当,可能会产生麻烦。例如:

\begin{cpp}
std::string* s = new std::string("whatever");
auto& c = (*s)[0];
delete s;
std::cout << c; // run-time ERROR
\end{cpp}

这里,获得了一个字符串元素的引用,但是当使用这个引用时,底层字符串已经不存在了(创建了一个悬空引用),并且出现了未定义行为。这个例子有些做作(有经验的程序员可能马上就会注意到问题),但是事情可能会变得不那么明显。例如:

\begin{cpp}
auto s = std::make_shared<std::string>("whatever");
auto& c = (*s)[0];
s.reset();
std::cout << c; // run-time ERROR
\end{cpp}

因此,应该确保函数模板按值返回结果。正如本章所讨论的,使用模板参数T并不能保证它不是引用,因为T有时可能会隐式推导为引用:

\begin{cpp}
template<typename T>
T retR(T&& p) // p is a forwarding reference
{
	return T{...}; // OOPS: returns by reference when called for lvalues
}
\end{cpp}

即使T是由按值调用推导而来的模板参数,当显式指定模板参数为引用时,也可能成为引用类型:

\begin{cpp}
template<typename T>
T retV(T p) // Note: T might become a reference
{
	return T{...}; // OOPS: returns a reference if T is a reference
}

int x;
retV<int&>(x); // retT() instantiated for T as int&
\end{cpp}

安全起见,这里有两个选择:

\begin{itemize}
\item 
使用类型特征std::remove\_reference<>(参见D.4节)将类型T转换为非引用:

\begin{cpp}
template<typename T>
typename std::remove_reference<T>::type retV(T p)
{
	return T{...}; // always returns by value
}
\end{cpp}

其他特征,如std::decay<>(参见D.4节),因为隐式地移除引用,在这里也可能会有用。

\item 
编译器通过声明返回类型auto来推断返回类型(C++14起;参见第1.3.2节),因为auto总会衰变:

\begin{cpp}
template<typename T>
auto retV(T p) // by-value return type deduced by compiler
{
	return T{...}; // always returns by value
}
\end{cpp}

\end{itemize}


















