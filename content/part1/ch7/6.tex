\section{推荐的模板参数声明}
正如前几节所述,声明依赖于模板参数类型有不同的方式:

\begin{itemize}
\item 
通过值传递参数:

这种方法很简单,衰变字符串字面值和数组,但不能为大型对象提供最佳性能。调用者可以决定使用std::cref()和std::ref()通过引用传递,但是必须确定这样做的必要性。

\item 
通过引用传递参数:

这种方法通常可以为大型对象提供更好的性能,特别是在传递参数时

\begin{itemize}
\item[-]
现有对象(lvalue)到左值引用,

\item[-]
临时对象(prvalue)或标记为可移动(xvalue)的对象将引用右值,

\item[-]
或者两者都为转发引用。
\end{itemize}

这些情况下,参数都不会衰变,所以在传递字符串字面值和其他数组时,可能需要特别注意。对于转发引用,还必须注意使用模板参数隐式推导出引用类型的方法。
\end{itemize}

\subsubsection{不要过于泛化}

实践中,函数模板通常不支持任意类型的参数,可以进行了一些约束。例如,可能知道只传递某种类型的vector。这种情况下,最好不要太泛化地声明这样的函数。如前所述,这可能会出现令人惊讶的副作用,可以使用以下声明:

\begin{cpp}
template<typename T>
void printVector (std::vector<T> const& v)
{
	...
}
\end{cpp}

通过在printVector()中声明参数v,可以确定传递的T不能成为引用,因为vector不能使用引用作为元素类型。另外,因为std::vector<>的复制构造函数会创建元素副本,所以按值传递vector的成本很高。出于这个原因,将这样的vector参数声明为按值传递可能不合适。若将参数v的声明交由类型T来决定,那么按值调用和按引用调用之间的区别就不那么明显了。

\subsubsection{std::make\_pair()的实例}

std::make\_pair<>()是一个很好的例子,演示了决定参数传递机制的缺陷。C++标准库中,可以使用其进行类型推导,并创建std::pair<>对象(一个方便的函数模板)。它的声明在不同版本的C++标准中有所不同:

\begin{itemize}
\item
C++98中,make\_pair<>()在命名空间std中声明,使用引用调用来避免不必要的复制:

\begin{cpp}
template<typename T1, typename T2>
pair<T1,T2> make_pair (T1 const& a, T2 const& b)
{
	return pair<T1,T2>(a,b);
}
\end{cpp}

然而,使用字符串字面值对或不同大小的数组时,这会导致严重的问题。

\begin{notice}
请参见C++库的181号issue[LibIssue181]
\end{notice}

\item
C++03中,函数定义改为使用按值调用:

\begin{cpp}
template<typename T1, typename T2>
pair<T1,T2> make_pair (T1 a, T2 b)
{
	return pair<T1,T2>(a,b);
}
\end{cpp}

如同在问题解决方案的基本原理中所了解到的那样,“与其他两个建议相比,这似乎是对标准的一个小修改,而且任何效率方面的担忧都由该解决方案的优点所抵消。”

\item[-]
C++11中,make\_pair()必须支持移动语义,因此参数必须成为转发引用。因此,该定义又发生了如下变化:

\begin{cpp}
template<typename T1, typename T2>
constexpr pair<typename decay<T1>::type, typename decay<T2>::type>
make_pair (T1&& a, T2&& b)
{
	return pair<typename decay<T1>::type,
				typename decay<T2>::type>(forward<T1>(a),
										  forward<T2>(b));
}
\end{cpp}

完整的实现甚至更加复杂:为了支持std::ref()和std::cref(),该函数还将std::reference\_wrapper的实例展开为实际的引用。
\end{itemize}

C++标准库现在在许多地方以类似的方式完美地转发传递的参数,并与std::decay<>一起使用。


















