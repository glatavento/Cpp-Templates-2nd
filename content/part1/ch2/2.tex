\section{使用栈类模板}

C++17前,要使用类模板的对象,必须显式指定模板参数。

\begin{notice}
C++17引入了类参数模板推导,若模板参数可以从构造函数派生,则可以跳过这些参数。这会在第2.9节中进行讨论。
\end{notice}

下面的例子展示了如何使用类模板Stack<>:

\filename{basics/stack1test.cpp}
\begin{cpp}
#include "stack1.hpp"
#include <iostream>
#include <string>

int main()
{
	Stack<int> intStack; // stack of ints
	Stack<std::string> stringStack; // stack of strings
	
	// manipulate int stack
	intStack.push(7);
	std::cout << intStack.top() << '\n';
	
	// manipulate string stack
	stringStack.push("hello");
	std::cout << stringStack.top() << '\n';
	stringStack.pop();
}
\end{cpp}

通过声明类型Stack<int>,int在类模板中作为类型T。因此,intStack是一个对象,使用的是vector<int>类型,并且调用相应的成员函数。类似地,通过声明和使用Stack<std::string>,将创建使用vector<std::string>的对象,使用相应的成员函数。

注意,代码只对调用的模板(成员)函数实例化。对于类模板,只有在使用成员函数时才实例化。当然,这节省了时间和空间,并且只允许使用部分地类模板,这会在2.3节中详细讨论。

例子中,默认构造函数push()和top()为int和string实例化,但pop()只对string实例化。如果类模板具有静态成员,则对于使用类模板的每个类型实例,这些成员也会实例化一次。

可以像使用其他类型一样使用实例化的类模板类型。可以使用const、volatile或从中派生数组和引用类型对其进行限定。也可以将其作为类型定义的一部分,使用typedef或using(请参阅第2.8节了解关于类型定义的详细信息),或者在构建另一个模板类型时将其用作类型参数。例如:

\begin{cpp}
void foo(Stack<int> const& s) // parameter s is int stack
{
	using IntStack = Stack<int>; // IntStack is another name for Stack<int>
	Stack<int> istack[10]; // istack is array of 10 int stacks
	IntStack istack2[10]; // istack2 is also an array of 10 int stacks (same type)
	...
}
\end{cpp}

模板参数可以是任何类型,例如float指针,甚至是Stack<int>:

\begin{cpp}
Stack<float*> floatPtrStack; // stack of float pointers
Stack<Stack<int>> intStackStack; // stack of stack of ints
\end{cpp}

这里的要求是,需要这种类型支持所使用的操作。

注意,在C++11之前,必须在两个模板右括号之间放空格:

\begin{cpp}
Stack<Stack<int> > intStackStack; // 所有C++版本都可以用
\end{cpp}

如果没有这样做,就使用>{}>,会导致语法错误:

\begin{cpp}
Stack<Stack<int>> intStackStack; // C++11之前会报错
\end{cpp}

旧行为可以帮助C++编译器对独立于代码语义源码进行标记。然而,由于缺少空格是一个错误,这需要相应的错误消息,因此无论如何都必须考虑代码的语义。因此,在C++11中,在两个模板右括号之间放一个空格的规则被“尖括号黑客”删除了(详见13.3.1节)。





















