
参数包还可以出现在其他地方,例如表达式、类模板、using声明,甚至推导策略。第12.4.2节有一个完整的列表。

\subsection{表达式}

不仅仅是转发所有参数,可以使用它们进行计算。也就是,可以使用参数包中的参数进行计算。

例如,下面的函数将参数包args的每个参数都加倍,并将每个加倍的参数传递给print():

\begin{lstlisting}[style=styleCXX]
template<typename... T>
void printDoubled (T const&... args)
{
	print (args + args...);
}
\end{lstlisting}

例如,

\begin{lstlisting}[style=styleCXX]
printDoubled(7.5, std::string("hello"), std::complex<float>(4,2));
\end{lstlisting}

该函数具有以下效果(除了构造函数的副作用):

\begin{lstlisting}[style=styleCXX]
print(7.5 + 7.5,
	std::string("hello") + std::string("hello"),
	std::complex<float>(4,2) + std::complex<float>(4,2));
\end{lstlisting}

如果想给每个参数加1,请注意省略号中的点不能直接跟在数字后面:

\begin{lstlisting}[style=styleCXX]
template<typename... T>
void addOne (T const&... args)
{
	print (args + 1...); // ERROR: 1... is a literal with too many decimal points
	print (args + 1 ...); // OK
	print ((args + 1)...); // OK
}
\end{lstlisting}

编译时表达式可以以同样的方式包含模板参数包。例如,下面的函数模板返回所有参数的类型是否相同:

\begin{lstlisting}[style=styleCXX]
template<typename T1, typename... TN>
constexpr bool isHomogeneous (T1, TN...)
{
	return (std::is_same<T1,TN>::value && ...); // since C++17
}
\end{lstlisting}

这是折叠表达式的一种应用(参见第4.2节)

\begin{lstlisting}[style=styleCXX]
isHomogeneous(43, -1, "hello")
\end{lstlisting}

返回值的表达式展开为

\begin{lstlisting}[style=styleCXX]
std::is_same<int,int>::value && std::is_same<int,char const*>::value
\end{lstlisting}

结果是false

\begin{lstlisting}[style=styleCXX]
isHomogeneous("hello", " ", "world", "!")
\end{lstlisting}

会产生true,因为传递的参数都推导为char const*(因为调用参数按值传递,所以参数类型衰变)。

\subsection{索引}

下面的函数使用可变索引列表来访问传递给第一个参数的对应元素:

\begin{lstlisting}[style=styleCXX]
template<typename C, typename... Idx>
void printElems (C const& coll, Idx... idx)
{
	print (coll[idx]...);
}
\end{lstlisting}

当使用

\begin{lstlisting}[style=styleCXX]
std::vector<std::string> coll = {"good", "times", "say", "bye"};
printElems(coll,2,0,3);
\end{lstlisting}

则调用

\begin{tcblisting}{commandshell={}}
print (coll[2], coll[0], coll[3]);
\end{tcblisting}

还可以将非类型模板参数声明为参数包。例如:

\begin{lstlisting}[style=styleCXX]
template<std::size_t... Idx, typename C>
void printIdx (C const& coll)
{
	print(coll[Idx]...);
}
\end{lstlisting}

也可以使用

\begin{lstlisting}[style=styleCXX]
std::vector<std::string> coll = {"good", "times", "say", "bye"};
printIdx<2,0,3>(coll);
\end{lstlisting}

这与前面的例子有相同的效果。

\subsection{类模板}

可变参数模板也可以是类模板,其中任意数量的模板参数指定了相应成员的类型:

\begin{lstlisting}[style=styleCXX]
emplate<typename... Elements>
class Tuple;

Tuple<int, std::string, char> t; // t can hold integer, string, and character
\end{lstlisting}

这将在第25章中讨论。

另一个例子是能够指定对象可能的类型:

\begin{lstlisting}[style=styleCXX]
template<typename... Types>
class Variant;

Variant<int, std::string, char> v; // v can hold integer, string, or character
\end{lstlisting}

这将在第26章中讨论。

也可以定义类,作为类型来表示一组索引:

\begin{lstlisting}[style=styleCXX]
// type for arbitrary number of indices:
template<std::size_t...>
struct Indices {
};
\end{lstlisting}

可以用来定义函数,该函数在编译时对给定索引使用get<>()访问std::array或std::tuple的元素调用print():

\begin{lstlisting}[style=styleCXX]
template<typename T, std::size_t... Idx>
void printByIdx(T t, Indices<Idx...>)
{
	print(std::get<Idx>(t)...);
}
\end{lstlisting}

该模板的使用方法如下:

\begin{lstlisting}[style=styleCXX]
std::array<std::string, 5> arr = {"Hello", "my", "new", "!", "World"};
printByIdx(arr, Indices<0, 4, 3>());
\end{lstlisting}

或者这样:

\begin{lstlisting}[style=styleCXX]
auto t = std::make_tuple(12, "monkeys", 2.0);
printByIdx(t, Indices<0, 1, 2>());
\end{lstlisting}

这只元编程的开始,在第8.1节和第23章中会更加详细的讨论元编程。

\subsection{推导策略}

推导策略(参见第2.9节)也可以应用于可变参数。例如,C++标准库定义了以下std::arrays的推导策略:

\begin{lstlisting}[style=styleCXX]
namespace std {
	template<typename T, typename... U> array(T, U...)
	-> array<enable_if_t<(is_same_v<T, U> && ...), T>,
		(1 + sizeof...(U))>;
}
\end{lstlisting}

初始化,

\begin{lstlisting}[style=styleCXX]
std::array a{42,45,77};
\end{lstlisting}

策略中推导出元素类型的T,以及各种U...类型转换为后续元素的类型。因此,元素总数为1 + sizeof...(U):

\begin{lstlisting}[style=styleCXX]
std::array<int, 3> a{42,45,77};
\end{lstlisting}

第一个数组参数的std::enable\_if<>表达式是一个折叠表达式(4.4.1节中引入的isHomogeneous())展开为:

\begin{lstlisting}[style=styleCXX]
is_same_v<T, U1> && is_same_v<T, U2> && is_same_v<T, U3> ...
\end{lstlisting}

若结果不为true(即并非所有元素类型都相同),则丢弃推导策略,从而整体推导失败。标准库需要所有元素具有相同的类型时,推导策略才能成功执行。

\subsection{基类和using}

看看下面的例子:

\filename{basics/varusing.cpp}
\begin{lstlisting}[style=styleCXX]
#include <string>
#include <unordered_set>

class Customer
{
	private:
	std::string name;
	public:
	Customer(std::string const& n) : name(n) { }
	std::string getName() const { return name; }
};

struct CustomerEq {
	bool operator() (Customer const& c1, Customer const& c2) const {
		return c1.getName() == c2.getName();
	}
};

struct CustomerHash {
	std::size_t operator() (Customer const& c) const {
		return std::hash<std::string>()(c.getName());
	}
};

// define class that combines operator() for variadic base classes:
template<typename... Bases>
struct Overloader : Bases...
{
	using Bases::operator()...; // OK since C++17
};

int main()
{
	// combine hasher and equality for customers in one type:
	using CustomerOP = Overloader<CustomerHash,CustomerEq>;
	
	std::unordered_set<Customer,CustomerHash,CustomerEq> coll1;
	std::unordered_set<Customer,CustomerOP,CustomerOP> coll2;
	...
}
\end{lstlisting}

首先定义了一个类Customer和一些独立的函数对象,用于计算哈希值和比较Customer对象。

\begin{lstlisting}[style=styleCXX]
template<typename... Bases>
struct Overloader : Bases...
{
	using Bases::operator()...; // OK since C++17
};
\end{lstlisting}

可以定义一个派生自可变数量基类的类,该类从每个基类引入函数操作符的声明。

\begin{lstlisting}[style=styleCXX]
using CustomerOP = Overloader<CustomerHash,CustomerEq>;
\end{lstlisting}

使用这个特性从CustomerHash和CustomerEq派生CustomerOP,并在派生类中启用函数操作符的实现。

参见第26.4节了解该技术的另一种应用场景。


























