C++11引入了一个新特性constexpr,简化了各种形式的编译时计算。特别是,如果有输入适当,可以在编译时对constexpr函数求值。虽然在C++11中引入constexpr函数有严格限制(每个constexpr函数本质上都需要包含一个return语句),但在C++14中,这些限制大部分都取消了。当然,成功地计算constexpr函数仍然需要所有的计算步骤,在编译时是可行和有效的:这排除了堆分配或抛出异常的情况。

我们测试一个数字是否是素数的例子,可以使用C++11进行如下实现:

\filename{basics/isprime11.hpp}
\begin{cpp}
constexpr bool
doIsPrime (unsigned p, unsigned d) // p: number to check, d: current divisor
{
	return d!=2 ? (p%d!=0) && doIsPrime(p,d-1) // check this and smaller divisors
	: (p%2!=0); // end recursion if divisor is 2
}

constexpr bool isPrime (unsigned p)
{
	return p < 4 ? !(p<2) // handle special cases
	: doIsPrime(p,p/2); // start recursion with divisor from p/2
}
\end{cpp}

由于只有一条语句的限制,只能使用条件操作符作为选择机制,并且需要递归来遍历元素。但其语法是普通的C++函数代码,这使得它比依赖于模板实例化的第一个版本更容易使用。

C++14中,constexpr函数可以使用通用C++代码中的控制结构。因此,不用编写笨拙的模板代码或有些“奇怪的”单行程序,现在只使用普通的for循环:

\filename{basics/isprime14.hpp}
\begin{cpp}
constexpr bool isPrime (unsigned int p)
{
	for (unsigned int d=2; d<=p/2; ++d) {
		if (p % d == 0) {
			return false; // found divisor without remainder
		}
	}
	return p > 1; // no divisor without remainder found
}
\end{cpp}

使用C++11和C++14版本的constexpr isPrime()实现,可以直接调用

\begin{cpp}
isPrime(9)
\end{cpp}

找出9是否为质数。可以在编译时这样做,但不一定要这样做。在需要编译时值的上下文中(例如,数组长度或非类型模板参数),编译器将尝试在编译时计算对constexpr函数的调用。若无法计算,则会产生错误(因为最后必须生成一个常量)。其他上下文中,编译器在编译时可能尝试或不尝试求值,但若这样的求值失败,是不会产生错误信息,而是将问题留给运行时。

\begin{notice}2017年写这本书的时候,编译器似乎确实在尝试编译时求值,即使不严格的情况下。
\end{notice}

例如:

\begin{cpp}
constexpr bool b1 = isPrime(9); // evaluated at compile time
\end{cpp}

编译时计算该值,也适用于

\begin{cpp}
const bool b2 = isPrime(9); // evaluated at compile time if in namespace scope
\end{cpp}

假设b2是全局定义的或在命名空间中定义的。块作用域中,编译器可以决定是在编译时计算,还是在运行时计算。

\begin{notice}理论上,即使使用了constexpr,编译器也可以决定在运行时计算b的初始值。编译器只需要在编译时检查它是否可计算即可。
\end{notice}

例如,这样:

\begin{cpp}
bool fiftySevenIsPrime() {
	return isPrime(57); // evaluated at compile or running time
}
\end{cpp}

编译器可能会在编译时计算对isPrime的调用。

另外:

\begin{cpp}
int x;
...
std::cout << isPrime(x); // evaluated at run time
\end{cpp}

将生成在运行时计算x是否为素数的代码。

