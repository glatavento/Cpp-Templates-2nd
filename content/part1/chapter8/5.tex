偏特化、SFINAE和std::enable\_if允许启用或禁用模板。C++17引入了编译时if语句,允许我们根据编译时条件启用或禁用特定语句。使用if constexpr(…)语法,编译器使用编译时表达式来决定是应用then部分,还是else部分(如果有的话)。

作为第一个例子,4.1.1节中介绍的可变参数函数模板print(),使用递归打印参数(任意类型)。与提供一个单独的函数来结束递归不同,constexpr if特性决定是否继续递归:

\begin{tcolorbox}[colback=webgreen!5!white,colframe=webgreen!75!black]
\hspace*{0.75cm}虽然代码读取if constexpr,但该特性称为constexpr if,因为它是if的“constexpr”形式(由于历史原因)。
\end{tcolorbox}

\begin{lstlisting}[style=styleCXX]
template<typename T, typename... Types>
void print (T const& firstArg, Types const&... args)
{
	std::cout << firstArg << ’\n’;
	if constexpr(sizeof...(args) > 0) {
		print(args...); // code only available if sizeof...(args)>0 (since C++17)
	}
}
\end{lstlisting}

这里,若只对一个参数调用print(), args将成为空的参数包,因此sizeof…(args)将变为0。结果,print()的递归调用变成了一个丢弃的语句,代码没有实例化。因此,相应的函数不需要存在,递归结束。

代码没有实例化意味着只执行第一个翻译阶段,检查正确的语法和不依赖于模板参数的名称(参见第1.1.3节)。

\begin{lstlisting}[style=styleCXX]
template<typename T>
void foo(T t)
{
	if constexpr(std::is_integral_v<T>) {
		if (t > 0) {
			foo(t-1); // OK
		}
	}
	else {
		undeclared(t); // error if not declared and not discarded (i.e. T is not integral)
		undeclared(); // error if not declared (even if discarded)
		static_assert(false, "no integral"); // always asserts (even if discarded)
		static_assert(!std::is_integral_v<T>, "no integral"); // OK
	}
}
\end{lstlisting}

若constexpr可以用于任何函数,而不仅在模板中。只需要一个产生布尔值的编译时表达式。例如:

\begin{lstlisting}[style=styleCXX]
int main()
{
	if constexpr(std::numeric_limits<char>::is_signed) {
		foo(42); // OK
	}
	else {
		undeclared(42); // error if undeclared() not declared
		static_assert(false, "unsigned"); // always asserts (even if discarded)
		static_assert(!std::numeric_limits<char>::is_signed,
		"char is unsigned"); // OK
	}
}
\end{lstlisting}

有了这个特性,若给定的大小不是质数,可以使用第8.2节中的isPrime()在编译时执行:

\begin{lstlisting}[style=styleCXX]
template<typename T, std::size_t SZ>
void foo (std::array<T,SZ> const& coll)
{
	if constexpr(!isPrime(SZ)) {
		... // special additional handling if the passed array has no prime number as size
	}
	...
}
\end{lstlisting}

详见14.6节。



























