\section{类模板偏特化}

模板全特化很有用,但有时会希望为一组模板参数特化类模板或变量模板,而不是只特化一组特定的模板参数。例如,有一个链表的类模板:

\begin{cpp}
template<typename T>
class List { // #1
	public:
	...
	void append(T const&);
	inline std::size_t length() const;
	...
};
\end{cpp}

使用此模板的大型项目可以实例化多种类型成员。对于未内联展开的成员函数(例如:List<T>::append()),可能会导致代码的大幅度增长。从底层来看,List<int*>::append()和List<void*>::append()的代码相同。我们希望指定所有指针List共享一个实现。虽然不能用C++来表达,可以指定不同的模板定义实例化来实现:

\begin{cpp}
template<typename T>
class List<T*> { // #2
	private:
	List<void*> impl;
	...
	public:
	...
	inline void append(T* p) {
		impl.append(p);
	}
	inline std::size_t length() const {
		return impl.length();
	}
	...
};
\end{cpp}

此上下文中,位于\#1的原始模板称为主模板,而后的定义称为偏特化(因为使用此模板定义的模板参数仅部分指定)。描述偏特化的语法是模板参数列表声明(template<…>)和一组在类模板名称上显式指定的模板参数(示例中是<T*>)的组合。

我们的代码有一个问题,因为List<void*>递归地包含了同一个List<void*>类型的成员。为了打破这个循环,可以在之前的偏特化前使用全特化:

\begin{cpp}
template<>
class List<void*> { // #3
	...
	void append (void* p);
	inline std::size_t length() const;
	...
};
\end{cpp}

因为全特化比偏特化更匹配,所以可行。因此,指针List的所有成员函数都会转发(通过容易内联的函数)到List<void*>。这是对抗代码膨胀(C++模板经常被诟病)的有效方法。

偏特化声明的形参和实参列表存在一些限制。其中一些如下:

\begin{enumerate}
\item 
偏特化的实参必须在类型(类型、非类型或模板)上与主模板的相应形参匹配。

\item 
偏特化的形参列表不能有默认实参,取而代之的是主类模板的默认实参。

\item 
偏特化的非类型参数应该是非依赖值或普通非类型模板参数。不能是更复杂的依赖表达式,如2*N(其中N是模板参数)。

\item 
偏特化的模板实参列表不应与主模板的形参列表相同(忽略重命名)

\item 
若其中一个模板参数是包扩展,这个包必须放在模板参数列表的最后。
\end{enumerate}

下面的例子说明了这些限制:

\begin{cpp}
template<typename T, int I = 3>
class S; // primary template

template<typename T>
class S<int, T>; // ERROR: parameter kind mismatch

template<typename T = int>
class S<T, 10>; // ERROR: no default arguments

template<int I>
class S<int, I*2>; // ERROR: no nontype expressions

template<typename U, int K>
class S<U, K>; // ERROR: no significant difference from primary template

template<typename... Ts>
class Tuple;

template<typename Tail, typename... Ts>
class Tuple<Ts..., Tail>; // ERROR: pack expansion not at the end

template<typename Tail, typename... Ts>
class Tuple<Tuple<Ts...>, Tail>; // OK: pack expansion is at the end of a
								// nested template argument list
\end{cpp}

每个偏特化都与主模板相关联。使用模板时,总是要查找主模板,但随后参数也要与相关特化的参数进行匹配(使用模板推导指引,如第15章所述),以确定选择哪个模板实现。就像函数模板参数推导一样,SFINAE原则在这里也适用:当试图匹配偏特化时,若形成了无效构造,则放弃该特化,并检查另一个候选是否可用。若没有找到匹配的特化,则选择主模板。若找到多个匹配的特化,则选择最特化的那个;若没有最特化的,则会出现歧义性错误。

最后,类模板偏特化有可能比主模板拥有更多或更少的参数。考虑泛型模板List,再次在点\#1声明。已经讨论了如何优化指针列表的情况,但这里可能希望对某些指向成员的指针类型进行同样的优化。下面的代码实现了指向成员指针的指针:

\begin{cpp}
// partial specialization for any pointer-to-void* member
template<typename C>
class List<void* C::*> { // #4
	public:
	using ElementType = void* C::*;
	...
	void append(ElementType pm);
	inline std::size_t length() const;
	...
};

// partial specialization for any pointer-to-member-pointer type except
// pointer-to-void* member, which is handled earlier
// (note that this partial specialization has two template parameters,
// whereas the primary template only has one parameter)
// this specialization makes use of the prior one to achieve the
// desired optimization
template<typename T, typename C>
class List<T* C::*> { // #5
	private:
	List<void* C::*> impl;
	...
	public:
	using ElementType = T* C::*;
	...
	inline void append(ElementType pm) {
		impl.append(static_cast<void* C::*>(pm));
	}
	inline std::size_t length() const {
		return impl.length();
	}
	...
};
\end{cpp}

除了对模板参数的数量外,在\#4定义的实现是一个偏特化(对于简单的指针情况,是一个全特化),所有其他定义都会使用\#5点的声明转发。然而,\#4的特化比\#5更特化;因此,不应出现歧义。

而且,显式编写的模板参数数量,甚至可能与主模板中的模板参数数量不同。这既可以在有默认模板参数的确情况下发生,也可以使用可变参数模板:

\begin{cpp}
template<typename... Elements>
class Tuple; // primary template

template<typename T1>
class Tuple<T>; // one-element tuple

template<typename T1, typename T2, typename... Rest>
class Tuple<T1, T2, Rest...>; // tuple with two or more elements
\end{cpp}


















