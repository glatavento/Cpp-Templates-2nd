\section{推导上下文}
比“T”复杂得多的参数化类型可以匹配到给定的参数类型。下面是一些例子:

\begin{cpp}
template<typename T>
void f1(T*);

template<typename E, int N>
void f2(E(&)[N]);

template<typename T1, typename T2, typename T3>
void f3(T1 (T2::*)(T3*));

class S {
	public:
	void f(double*);
};

void g (int*** ppp) {
	bool b[42];
	f1(ppp); // deduces T to be int**
	f2(b); // deduces E to be bool and N to be 42
	f3(&S::f); // deduces T1 = void, T2 = S, and T3 = double
}
\end{cpp} 

复杂类型声明有基本的构造(指针、引用、数组和函数声明符;指向成员的声明符;模板标识;等等),匹配过程从顶层构造开始,递归地遍历各个元素。大多数类型声明构造都可以用这种方式匹配,这称为推导上下文。然而,有一些结构不可推导上下文。例如:

\begin{itemize}
\item 
限定类型名称。例如,永远不会使用像Q<T>::X这样的类型名来推导模板参数T。

\item 
非类型参数不只有非类型表达式,像S<I+1>这样的类型名永远不会用于推断I,也不会通过匹配类型为int(\&)[sizeof(S<T>)]的参数来推断T的类型。
\end{itemize}

这些限制并不奇怪,因为推导过程通常不唯一(甚至是有限的),尽管限定类型名的这种限制有时容易忽略。非推导上下文不会表示程序出错,甚至不会说明分析的参数不能参与类型推导。为了说明这一点,看看下面这个更复杂的例子:

\filename{basics/fppm.cpp}
\begin{cpp}
template<int N>
class X {
	public:
	using I = int;
	void f(int) {
	}
};

template<int N>
void fppm(void (X<N>::*p)(typename X<N>::I));

int main() {
	fppm(&X<33>::f); // fine: N deduced to be 33
}
\end{cpp}

函数模板fppm()中,子构造X<N>::I不用推导上下文。然而,成员指针类型的成员类组件X<N>可推导上下文,当它推导出的参数N放入到非推导上下文时,将获得与实际参数类型兼容的类型\&X<33>::f。因此,参数匹配,推导成功。

相反,对于完全由推导上下文构建的参数类型,推导也可能出现矛盾。例如,声明的类模板X和Y:

\begin{cpp}
template<typename T>
void f(X<Y<T>, Y<T>>);

void g() {
	f(X<Y<int>, Y<int>>()); // OK
	f(X<Y<int>, Y<char>>()); // ERROR: deduction fails
}
\end{cpp}

第二次调用函数模板f()的问题是,两个参数T会推导出不同的类型,这是无效的(两种情况下,函数调用参数是通过调用类模板X的默认构造函数,获得的临时对象)。
































