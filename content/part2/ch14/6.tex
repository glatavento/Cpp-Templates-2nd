\section{编译时if语句}
正如第8.5节中介绍的,C++17添加了一种新的语句类型,编写模板时非常有用:编译时if。不过,这在实例化过程中引入了一个新的问题。

下面的例子演示了其基本操作:

\begin{cpp}
template<typename T> bool f(T p) {
	if constexpr (sizeof(T) <= sizeof(long long)) {
		return p>0;
	} else {
		return p.compare(0) > 0;
	}
}

bool g(int n) {
	return f(n); // OK
}
\end{cpp}

编译时if是一个if语句,其中if关键字紧跟着constexpr关键字(如本例所示)。

\begin{notice}
虽然代码读取if constexpr,但该特性称为constexpr if,因为它是if的“constexpr”形式。
\end{notice}

后面的圆括号条件必须有一个确定的布尔值(包含了对bool的隐式转换)。因此,编译器知道将选择哪个分支;另一个分支为丢弃分支。特别值得注意的是,在模板(包括泛型Lambda)的实例化期间,不会实例化丢弃分支。我们使用T = int实例化f(T),将丢弃else分支。如果没有丢弃,将会实例化,并且会遇到表达式p.compare(0)的错误信息(当p是一个简单整数时,这个表达式是错误是)。

C++17及其constexpr if语句可用之前,为了避免此类错误,需要显式的模板特化或重载(参见第16章)来达到类似的效果。

上面的例子中,在C++14可以这样实现:

\begin{cpp}
template<bool b> struct Dispatch { // only to be instantiated when b is false
	static bool f(T p) { // (due to next specialization for true)
		return p.compare(0) > 0;
	}
};

template<> struct Dispatch<true> {
	static bool f(T p) {
		return p > 0;
	}
};

template<typename T> bool f(T p) {
	return Dispatch<sizeof(T) <= sizeof(long long)>::f(p);
}

bool g(int n) {
	return f(n); // OK
}
\end{cpp}

显然,constexpr if替代方案使我们的意图,更加清楚和简洁。然而,需要实现来细化实例化的单元:以前的函数定义总是作为一个整体实例化,现在必须抑制部分实例化。

constexpr if的另一个用法是表示处理函数参数包所需的递归。为了推广这个例子,在第8.5节中进行了介绍:

\begin{cpp}
template<typename Head, typename... Remainder>
void f(Head&& h, Remainder&&... r) 
	doSomething(std::forward<Head>(h));
	if constexpr (sizeof...(r) != 0) {
		// handle the remainder recursively (perfectly forwarding the arguments):
		f(std::forward<Remainder>(r)...);
	}
}
\end{cpp}

如果没有constexpr if语句,就需要对f()模板进行重载,以确保递归终止。

即使在非模板上下文中,constexpr if语句也有些独特的效果:

\begin{cpp}
void h();
void g() {
	if constexpr (sizeof(int) == 1) {
		h();
	}
}
\end{cpp}

在大多数平台上,g()中的条件为false,因此会丢弃h()的调用。因此,h()不需要定义(当然,除非在其他地方使用)。如果本例中省略了关键字constexpr,那么缺少h()的定义通常会在链接时引起错误。

\begin{notice}
优化可能会掩盖错误。如果保证不存在问题,则可以使用constexpr。
\end{notice}























