\section{延迟实例化}

到目前为止所展示的这些例子,和使用非模板类相比并没有本质上的区别,例如:需要完整的类类型(参见第10.3.1节)。对于模板的情况,编译器将根据类模板定义生成这个完整的定义。

那模板实例化程度如何?一个模糊的答案是:达到需要的程度即可。换句话说,编译器在实例化模板时是“延迟的”。来看看这种懒惰到底意味着什么。

\subsection{部分实例化和完全实例化}

正如所看到的,编译器有时不需要替换类或函数模板的完整定义。例如:

\begin{cpp}
template<typename T> T f (T p) { return 2*p; }
decltype(f(2)) x = 2;
\end{cpp}

本例中,由decltype(f(2))表示的类型不需要函数模板f()的完整实例化。因此,编译器允许替换f()的声明,而不允许替换它的“主体”。这有时称为部分实例化。

如果引用类模板的实例,而不需要该实例是完整类型,编译器不执行该类模板实例的完整实例化。看看下面的例子:

\begin{cpp}
template<typename T> class Q {
	using Type = typename T::Type;
};

Q<int>* p = 0; // OK: the body of Q<int> is not substituted
\end{cpp}

这里,因为T::Type在T为int时没有意义,所以Q<int>的实例化将触发一个错误。在这个例子中Q<int>不需要完整定义,所以没有执行完整的实例化。

变量模板也有“完整”和“部分”实例化的区别。下面的例子说明了这一点:

\begin{cpp}
template<typename T> T v = T::default_value();
decltype(v<int>) s; // OK: initializer of v<int> not instantiated
\end{cpp}

v<int>的完整实例化会引起错误,但只是需要变量模板实例类型的话,不需要进行完整实例化。

有趣的是,别名模板没有这种区别。

C++谈到“模板实例化”而不确定是完整还是部分实例化时,通常说的是“完整实例化”。也就是说,实例化在默认情况下是完全实例化。

\subsection{实例化组件}

当类模板隐式(完全)实例化时,其成员的声明也会实例化,但相应的定义不会(即成员部分实例化),当然也有一些例外。首先,若类模板包含匿名联合,则该联合定义的成员也会实例化。

\begin{notice}
匿名联合在这方面总是特别的:其成员可以认为是外围类的成员。匿名联合是一种表示某些类成员共享相同存储空间的结构。
\end{notice}

另一个异常发生在虚成员函数中,其定义可以作为类模板实例化的结果而实例化,也可以不实例化。许多实现将实例化该定义,因为支持虚函数调用机制的内部结构,要求虚函数实际上作为可链接实体存在。 

实例化模板时,默认函数调用参数会单独处理。除非调用了实际使用默认参数的函数(或成员函数),否则不会对其实例化。另外,如果函数调用时带有覆盖默认参数的显式参数,不会实例化默认参数。

类似地,通常不会实例化异常规范和默认成员初始化器。

结合一些例子来说明这些原则:

\filename{details/lazy1.hpp}
\begin{cpp}
template<typename T>
class Safe {};

template<int N>
class Danger {
	int arr[N]; // OK here, although would fail for N<=0
};

template<typename T, int N>
class Tricky {
	public:
	void noBodyHere(Safe<T> = 3); // OK until usage of default value results in an error
	void inclass() {
		Danger<N> noBoomYet; // OK until inclass() is used with N<=0
	}
	struct Nested {
		Danger<N> pfew; // OK until Nested is used with N<=0
	};
	union { // due anonymous union:
		Danger<N> anonymous; // OK until Tricky is instantiated with N<=0
		int align;
	};
	void unsafe(T (*p)[N]); // OK until Tricky is instantiated with N<=0
	void error() {
		Danger<-1> boom; // always ERROR (which not all compilers detect)
	}
};
\end{cpp}

标准C++编译器会检查这些模板定义,来检查语法和语义约束。在当检查涉及模板参数的约束时,将其“认为是最佳情况”。例如,成员Danger::arr中的参数N可以是零或负的(这是无效的),但假定情况并非如此。

\begin{notice}
一些编译器,如GCC,允许零长度数组作为扩展,因此即使N为0,也没问题。
\end{notice}

因此,inclass()、嵌套struct和匿名联合的定义都不是问题。

只要N是一个未替换的模板参数,成员的不安全声明(T (*p)[N])也没有问题。

因为模板Safe<>不能用整数初始化,所以成员noBodyHere()声明的默认参数specification(=3)看起来很诡异,但假设要么默认参数不需要用Safe<T>的泛型定义,要么Safe<T>将特化(参见第16章)以使用整数值初始化。然而,即使模板没有实例化,成员函数error()也会为其定义一个错误。因为使用Danger<-1>需要完整地定义Danger<-1>类,而生成该类会导致定义一个负大小的数组。有趣的是,虽然标准明确声明此代码无效,但也允许编译器在没有实际使用模板实例时不报错误。由于Tricky<T,N>::error()不用于具体的T和N,因此编译器不需要为此发出错误。在编写本文时,GCC和Visual C++都不会对此进行报错。

现在来分析一下当添加以下定义时会发生什么:

\begin{cpp}
Tricky<int, -1> inst;
\end{cpp}

这会导致编译器在模板Tricky<>的定义中用int替换T,用-1替换N,从而(完全)实例化Tricky<int, -1>。不是所有成员定义都需要,但默认构造函数和析构函数(隐式声明)肯定会调用,因此它们的定义必须以某种方式可用。如上所述,Tricky<int, -1>的成员会部分实例化(也就是声明被替换):这个过程可能会导致错误。例如,不安全的(T (*p)[N])声明会创建一个负数的数组类型(这是错误的)。类似地,因为类型Danger<-1>无法完成,成员anonymous会触发错误。相比之下,成员class()和struct Nested的定义还没有实例化,因此在完整类型Danger<-1>(包含前面讨论的无效数组)实例化时不会发生错误。

实例化模板时,实际上也应该提供虚成员的定义。否则,可能会发生链接错误。例如:

\filename{details/lazy2.cpp}
\begin{cpp}
template<typename T>
class VirtualClass {
	public:
	virtual ~VirtualClass() {}
	virtual T vmem(); // Likely ERROR if instantiated without definition
};

int main() {
	VirtualClass<int> inst;
}
\end{cpp}

最后,注意operator\inlcpp{->}:

\begin{cpp}
template<typename T>
class C {
	public:
	T operator-> ();
};
\end{cpp}

通常,operator\inlcpp{->} 必须返回指针类型或其他类类型。这样,因为声明了operator\inlcpp{->} 的返回类型为int,C<int>的补全会触发一个错误。但某些类模板定义触发了这些类型(返回类型为T或者T*)的定义,所以语言规则更灵活。

\begin{notice}
典型的例子是智能指针模板(例如,std::unique\_ptr<T>)
\end{notice}

只有在重载解析规则确实选择了用户自定义的operator\inlcpp{->} 时,才要求自定义operator\inlcpp{->} 只能返回应用其他(内置)operator\inlcpp{->} 的类型。这甚至对模板之外的代码也同样有效(这种松弛法则(relaxed behavior)在上下文中用处不大)。因此,尽管int会替代该返回类型,但这里的声明不会触发错误。






































