\section{按需实例化}
当C++编译器遇到模板特化时,将通过替换模板参数来创建该特化。

\begin{notice}
术语特化用于一般意义上的实体,是模板的一个特定实例(参见第10章),不涉及第16章中描述的显式特化机制。
\end{notice}

特化是自动完成的,不需要特殊代码(或者模板定义)的指导。这种按需变化的实例化特性将C++模板与其他早期编译语言不同(如Ada或Eiffel;一些语言需要显式的实例化指令,而其他语言则使用运行时分发机制来避免实例化)。这有时也称为隐式或自动实例化。

按需实例化要求编译器在使用时,需要经常访问模板及其部分成员的完整定义(不仅仅是声明)。看看下面的示例:

\begin{cpp}
template<typename T> class C; // #1 declaration only

C<int>* p = 0; // #2 fine: definition of C<int> not needed
template<typename T>

class C {
	public:
	void f(); // #3 member declaration
}; // #4 class template definition completed

void g (C<int>& c) { // #5 use class template declaration only
	c.f(); // #6 use class template definition;
} // will need definition of C::f()

// in this translation unit
template<typename T>
void C<T>::f() // required definition due to #6
{ }
\end{cpp}

\#1处只有模板的声明是可用的,而不是定义(这种声明称为前置声明)。与普通类的情况一样,不需要类模板的定义来声明指向该类型的指针或引用,就像\#2那样。例如,函数g()的参数类型不需要模板C的完整定义。但当组件需要知道模板特化的大小时,或者访问类的特化成员时,整个类模板定义就必须可见。这解释了为什么在源代码中\#6,必须看到类模板定义;否则,编译器无法验证该成员是否存在,以及是否可访问(不是private或protected)。此外,也需要成员函数的定义,因为\#6的调用需要C<int>::f()。

下面是另一个表达式,因为需要C<void>的大小,所以需要实例化之前的类模板:

\begin{cpp}
C<void>* p = new C<void>;
\end{cpp}

这种情况下需要实例化,以便编译器可以确定C<void>的大小,new表达式需要确定分配多少存储空间。对于这个特定的模板,替代T的参数X的类型不会影响模板的大小,因为C<X>是一个空类。但编译器不需要分析模板定义来避免实例化(编译器都会执行实例化)。此外,本例中还需要实例化来确定C<void>是否具有可访问的默认构造函数,并确保C<void>不会声明成员操作符new或delete。

访问类模板成员在源代码中并不总是显式可见,例如:C++重载解析要求对候选函数的参数类型可见:

\begin{cpp}
template<typename T>
class C {
public:
	C(int); // a constructor that can be called with a single parameter
}; // may be used for implicit conversions

void candidate(C<double>); // #1
void candidate(int) { } // #2

int main() {
	candidate(42); // both previous function declarations can be called
}
\end{cpp}

调用candidate(42)时,将解析为在\#2处的重载声明。然而,\#1的声明也可以实例化,以检查是否是匹配的候选(因为单参数构造函数可以隐式地将42转换为类型C<double>的右值)。如果编译器可以在没有实例化的情况下解析调用,则允许(但不是必需)执行此实例化(本例可能就是这种情况,因为在精确匹配上不会选择隐式转换)。还要注意,C<double>的实例化可能会触发错误。




























