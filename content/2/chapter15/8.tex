

模板参数推导是一个强大的特性,消除了在大多数函数模板调用中显式指定模板参数的需要,并且启用了函数模板重载(参见第1.5节)和部分类模板特化(参见第16.4节)。

但是,开发者在使用模板时可能会遇到一些限制,本节来讨论一下这些限制。

\subsection{合法的参数转换}

通常,模板推导试图找到函数模板参数的替换类型,使参数化的类型P与类型A相同。然而,当这不可能时,当P在推导上下文中包含模板参数时,以下差异合理的:

\begin{itemize}
\item 
若原始参数是用引用声明符声明的,那么替换的P类型可能比A类型更符合const/volatile限定。

\item 
如果A类型是指针或成员指针类型,则可以通过限定转换转换为替换的P类型(换句话说,就是添加const和/或volatile限定符的转换)。

\item 
除非对转换操作符模板进行推导,否则替换的P类型可以是A类型的基类类型,或者是指向A为基类类型的指针。例如:

\begin{cpp}
template<typename T>
class B {
};

template<typename T>
class D : public B<T> {
};

template<typename T> void f(B<T>*);
void g(D<long> dl)
{
	f(&dl); // deduction succeeds with T substituted with long
}
\end{cpp}
\end{itemize}

若P在推导的上下文中不包含模板参数,那么所有隐式转换都是可以进行。例如:

\begin{cpp}
template<typename T> int f(T, typename T::X);
struct V {
	V();
	struct X {
		X(double);
	};
} v;
int r = f(v, 7.0); // OK: T is deduced to V through the first parameter,
					// which causes the second parameter to have type V::X
					// which can be constructed from a double value
\end{cpp}

只有在不能精确匹配的情况下才考虑宽松匹配要求。即便如此,只有找到一个替换使A类型与添加了这些转换的P类型相匹配,推导才会成功。

这些规则的作用范围非常窄,忽略了(例如)可以应用于函数参数以使调用成功的各种转换。例如,下面对max()函数模板的调用,如15.1节所示:

\begin{cpp}
std::string maxWithHello(std::string s)
{
	return ::max(s, "hello");
}
\end{cpp}

这里由第一个参数推导的模板参数,将T推导为std::string,而由第二个参数推导的模板参数T推导为char[6],因此模板参数推导失败,因为两个参数使用相同的模板参数。这个失败可能令人惊讶,因为字符串字面量“hello”可以隐式转换为std::string

\begin{cpp}
::max<std::string>(s, "hello")
\end{cpp}

这样就成功了。

当两个参数从一个公共基类派生出不同的类类型时,推导并不认为该公共基类是推导出的类型的候选者。参见1.2节对此问题的讨论和可能的解决方案。

\subsection{类模板参数}

C++17前,模板参数推导只适用于函数和成员函数模板。特别地,类模板的参数并没有从参数导出到对其构造函数的调用。例如:

\begin{cpp}
template<typename T>
class S {
	public:
	S(T b) : a(b) {
	}
	private:
	T a;
};

S x(12); // ERROR before C++17: the class template parameter T was not deduced from
// the constructor call argument 12
\end{cpp}

这个限制在C++17中取消了,参见第15.12节。

\subsection{默认调用参数}

默认函数调用参数可以在函数模板中指定,就像普通函数一样:

\begin{cpp}
template<typename T>
void init (T* loc, T const& val = T())
{
	*loc = val;
}
\end{cpp}

这个例子所示,默认函数调用参数可以依赖于模板参数。这种依赖的默认参数只有在没有提供显式参数的情况下,才会实例化——这一原则使得下面的例子有效:

\begin{cpp}
class S {
	public:
	S(int, int);
};

S s(0, 0);

int main()

{
	init(&s, S(7, 42)); // T() is invalid for T = S, but the default
	// call argument T() needs no instantiation
	// because an explicit argument is given
}
\end{cpp}

即使不依赖默认调用参数,也不能用它来推导模板参数。所以,以下代码无效:

\begin{cpp}
template<typename T>
void f (T x = 42)
{ }

int main()
{
	f<int>(); // OK: T = int
	f(); // ERROR: cannot deduce T from default call argument
}
\end{cpp}

\subsection{异常规范}

与默认调用参数一样,异常规范只在需要时才实例化,不参与模板参数推导。例如:

\begin{cpp}
template<typename T>
void f(T, int) noexcept(nonexistent(T())); // #1

template<typename T>
void f(T, ...); // #2 (C-style vararg function)
void test(int i)
{
	f(i, i); // ERROR: chooses #1 , but the expression nonexistent(T()) is ill-formed
}
\end{cpp}

\#1的函数中的noexcept规范试图调用一个不存在的函数。通常,在函数模板声明中直接出现这样的错误,会触发模板参数推导失败(SFINAE),通过选择\#2的匹配较小的函数(与省略号参数匹配从重载解析的角度来看,是最糟糕的匹配类型;详见附录C)但是,由于异常说明不参与模板参数推导,所以重载解析选择\#1,当noexcept实例化时,程序就会呈现出一种病态的形式。

相同的规则适用于潜在异常类型的异常规范:

\begin{cpp}
template<typename T>
void g(T, int) throw(typename T::Nonexistent); // #1

template<typename T>
void g(T, ...); // #2

void test(int i)
{
	g(i, i); // ERROR: chooses #1 , but the type T::Nonexistent is ill-formed
}
\end{cpp}

然而,这些“动态”异常规范自C++11以来已弃用,并在C++17中删除。


















