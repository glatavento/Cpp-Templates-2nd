真正用于解析模板定义的第一个编译器是在20世纪90年代中期,由一家名为Taligent的公司开发的。在此之前——甚至几年后——大多数编译器都将模板视为在实例化时才处理的标记。因此,除了找到模板定义结束位置等少许操作以外,没有进行任何解析。撰写本文时,Microsoft Visual C++编译器仍然以这种方式工作。爱迪生设计集团(EDG)的编译器前端使用了一种混合技术,模板在内部视为一个带注释的标记序列,在需要的模式中执行“通用解析”来验证语法(EDG的产品模拟多个其他编译器,可以很好地模拟微软编译器的行为)。

Bill Gibbons是Taligent在C++委员会的代表,其极力主张让模板可以无二义性地进行解析。然而,直到惠普公司完成第一个完整的编译器之后,Taligent公司的努力才真正产品化,也才有了一个真正编译模板的C++编译器。和其他具有竞争性优点的产品一样,这个C++编译器很快就由于高质量的错误信息而得到业界的认可。模板的错误信息不会总延迟到实例化时才发出,也要归功于这个编译器。

模板的早期开发过程中,Tom Pennello(Metaware公司的一位著名解析专家)就意识到了尖括号所带来的一些问题。Stroustrup也对这个话题进行了讨论[StroustrupDnE],而且认为人们更喜欢阅读尖括号,而不是圆括号。然而,除了尖括号和圆括号,还存在其他的一些可能性:Pennello在1991年的C++标准大会(在达拉斯举办)上特别地提议使用大括号,例如(List{::X})。

\begin{notice}使用括号也不是完全没有问题。具体来说,特化类模板的语法需要进行重大的调整。
\end{notice}

那时,因为嵌入在其他模板内部的模板(也称为成员模板)还是不合法的,所以问题的影响范围也非常有限,也就不会涉及到13.3.3节的问题。最后,委员会拒绝了这个取代尖括号的提议。

13.4.2节中描述的非依赖型名称和依赖型基类的名称查找规则,是在1993年C++标准中引入的。在1994年,Bjarne Stroustrup的[StroustrupDnE]首次公开描述了这一规则。然而直到1997年惠普才把这一规则引入C++编译器,自那以后出现了大量的派生自依赖型基类的类模板代码。当惠普工程师开始测试该实现时,发现大部分以特殊方式使用模板的代码都无法编译成功了。

\begin{notice}幸运的是,在发布新功能之前就发现了。
\end{notice}

特别地,STL的所有实现都打破了这一规则。

\begin{notice}具有讽刺意味的是,第一个实现也是由惠普开发的。
\end{notice}

考虑到客户的转换代码的成本,对于那些“假定非依赖型名称可以在依赖型基类中进行查找”的代码,惠普弱化了相关的错误信息。例如,对于位于类模板作用域的非依赖型名称,若利用标准原则不能找到该名称,C++就会在依赖型基类中进行查找。若仍然找不到,才会给出一个错误而编译失败。然而,若在依赖型基类中找到了该名称,那么就会给出一个警告,对该名称进行标记,并且当成是依赖型名称,然后在实例化时再次查找。

查找过程中,“非依赖型基类中的名称,会隐藏相同名称的模板参数(13.4.1节)”这一规则显然是一个疏忽,但修改这一规则的建议还没被C++标准委员会所认可。最好的办法就是避免使用非依赖型基类中的名称作为模板参数名称。命名转换对这一类问题都是一种良好的解决方式。

友元注入一度认为是有害的,会使得程序的合法性与实例出现的顺序紧密相关。Bill Gibbons(此时他还在Taligent公司开发编译器)就是解决这一问题的最大支持者,因为消除实例顺序依赖性激活了一个新的、有趣的C++开发领域(传闻Taligent正在做)。然而,Barton-Nackman技巧(21.2.1节)需要友元注入的形式,正是这种特殊的技术使它以基于ADL的(弱化)形式保留在语言中。

Andrew Koenig首次为操作符函数提出了ADL查找(这就是为什么有时候ADL也称为Koenig查找),动机主要是考虑美观性:“用外围命名空间显式地限定操作符名称”看起来很拖沓(例如,对于a+b,需要这样编写:N::operator+(a,b)),而为每个操作符使用using声明又会让代码看起来非常笨重。因此,才决定操作符可以在参数关联的命名空间中查找。ADL随后扩展到普通函数名称的查找,得以容纳有限种类的友元名称注入,并为模板及其实例支持两阶段查找模型(第14章)。泛化的ADL规则也称作扩展Koenig查找。

David Vandevoorde通过他的论文N1757在C++11中添加了“尖括号黑客”规范。还通过核心问题1104的解决方案添加了“有向图黑客”,以解决美国对C++11标准草案的审查要求。