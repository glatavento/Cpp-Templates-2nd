\section{定义原则的细节}

本附录的介绍中所述,ODR有很多细节。我们根据规则的范围,逐个介绍相应约束。

\subsection{程序的约束}

每个程序最多只能有以下的一个定义:

\begin{itemize}
\item 
非内联函数和非内联成员函数(包括函数模板的全特化)

\item 
非内联变量(本质上是在命名空间作用域或全局作用域中声明的变量,没有静态说明符)

\item 
非内联静态数据成员
\end{itemize}

例如,由以下两个翻译单元组成的C++程序无效:

\begin{cpp}
// translation unit 1:
int counter;

// translation unit 2:
int counter; // ERROR: defined twice (ODR violation)
\end{cpp}

此规则不适用于具有内部链接的实体(本质上是在全局作用域或命名空间作用域中,使用静态说明符声明的实体),即使两个这样的实体具有相同的名称,也会认为是不同的。同样,在匿名命名空间中声明的实体,若出现在不同的翻译单元中,则认为是不同的;C++11及以后的版本中,这些实体默认也具有内部链接,但在C++11前,默认具有外部链接。以下两个翻译单元可以组合成一个有效的C++程序:

\begin{cpp}
// translation unit 1:
static int counter = 2; // unrelated to other translation units

namespace {
	void unique() // unrelated to other translation units
	{ }
}

// translation unit 2:
static int counter = 0; // unrelated to other translation units
namespace {
	void unique() // unrelated to other translation units
	{
		++counter;
	}
}

int main()
{
	unique();
}
\end{cpp}

此外,若在conexpr if语句的丢弃分支之外的上下文中使用,则程序中必须只有前面提到的一项(该特性仅在C++17中可用;参见第14.6节)。在上下文中使用的术语具有确切的含义,表明在程序的某个地方存在对实体的某种引用,所以在直接的代码生成中需要实体。

\begin{notice}
各种优化技术可能会对实体进行删除,但该语言没有假设这种优化。
\end{notice}

此引用可以是对变量值的访问、对函数的调用或此类实体的地址。这个引用可以在源文件中显式引用,也可以隐式引用。new表达式可以隐式地创建对delete操作符的调用,以处理当构造函数抛出异常要求清理未使用(但已分配)内存时的情况。另一个例子包括复制构造函数,即使最终优化掉了,也必须定义(除非语言要求优化,C++17中经常出现这种情况)。虚函数也是隐式使用(支持虚函数调用的内部结构使用),除非是纯虚函数。还有其他几种隐式用法,但为了简明起见,这里暂时忽略它们。

有些引用不构成使用:出现在未求值操作数中的引用(sizeof或decltype操作符的操作数)。typeid操作符(参见第9.1.1节)的操作数只在某些情况下不计算,若引用作为typeid操作符的一部分出现就不使用,除非typeid操作符的参数最终指定了一个多态对象(可能继承了虚函数的对象)。例如,以下单文件程序:

\begin{cpp}
#include <typeinfo>

class Decider {
	#if defined(DYNAMIC)
	virtual ~Decider() {
	}
	#endif
};

extern Decider d;

int main()
{
	char const* name = typeid(d).name();
	return (int)sizeof(d);
}
\end{cpp}

当预处理器符号DYNAMIC未定义时,这是一个有效的程序。变量d没有定义,但是sizeof(d)中对d的引用并不构成使用,而typeid(d)中的引用,只有在d是多态类型对象时才构成使用(在运行时之前,不总确定多态类型id操作的结果)。

根据C++标准,本节描述的约束不需要从C++实现进行诊断,这些符号就是链接器报告为重复或缺失的定义。

\subsection{翻译单元的限制}

翻译单元中,实体都不能定义超过一次。所以下面的例子无效:

\begin{cpp}
inline void f() {}
inline void f() {} // ERROR: duplicate definition
\end{cpp}

这也是在头文件中使用守卫包围代码的主要原因之一:

\begin{cpp}
// guarddemo.hpp:
#ifndef GUARDDEMO_HPP
#define GUARDDEMO_HPP
...
#endif // GUARDDEMO_HPP
\end{cpp}

这样的保护确保头文件第二次\#include时,丢弃内容,从而避免其类、内联实体、模板等的重复定义。

ODR还指定必须在某些情况下定义某些实体。这可能是类类型、内联函数和内联变量的情况。接下来,我们将回顾详细的规则。

类类型X(包括结构体和联合体)必须在翻译单元中定义,才能在该翻译单元中进行下列类型的使用:

\begin{itemize}
\item 
创建类型为X的对象(例如,通过变量声明或通过new表达式)。当创建本身包含X类型对象的对象时,可以间接创建。

\item 
类型X的数据成员声明。

\item 
对X类型的对象应用sizeof或typeid操作符。

\item 
显式或隐式地访问类型X的成员。

\item 
使用类型的转换将表达式转换为或从类型X转换,或者使用隐式强制转换、static\_cast或dynamic\_cast换将表达式转换为指向X的指针或引用(void*除外)。

\item 
将值赋给X类型的对象。

\item 
定义或调用带有参数或返回类型为X的函数,但声明这样的函数并不需要定义其类型。
\end{itemize}

类型规则也适用于从类模板生成的类型X,需要在定义类型X的情况下定义相应的模板。这些情况会创建实例化点或POI(参见第14.3.2节)。

内联函数必须使用每个翻译单元中定义(这些翻译单元中调用它们或获取它们的地址),但与类类型不同,它们的定义可以遵循以下使用要点:

\begin{cpp}
inline int notSoFast();

int main()
{
	notSoFast();
}

inline int notSoFast()
{ }
\end{cpp}

虽然这是有效的C++代码,但基于旧技术的编译器实际上并不会“内联”调用尚未看到主体的函数;因此,预期的效果可能无法达到。

与类模板一样,使用由参数化函数声明(函数或成员函数模板,或类模板的成员函数)生成的函数会创建一个实例化点。但与类模板不同,相应的定义可以出现在实例化点之后。

本小节中解释的ODR的各个方面通常很容易通过C++编译器验证;因此,C++标准要求编译器在违反这些规则时发出某种诊断。异常是缺少参数化函数的定义,这种情况通常没有诊断信息。

\subsection{交叉翻译单元的等价约束}

多个翻译单元中定义某些类型可能会带来新的错误:多个定义不匹配。但传统的编译器每次只处理一个翻译单元,所以很难检测到这些错误。因此,C++标准并不要求检测或诊断多个定义之间的差异,但若违反交叉转换单元的约束,C++标准将其限定为未定义行为,从而任何合理或不合理的事情都可能发生。通常,这种未诊断的错误可能导致程序崩溃或错误的结果,但也可能导致其他更直接的损害(例如,文件损坏)。

\begin{notice}
gcc编译器的第1版实际上没有做到了这一点,只是某些情况下启动了“肉鸽游戏”。
\end{notice}

交叉翻译单元约束指定,实体在两个不同的地方定义时,这两个地方必须包含完全相同的标记序列(预处理之后保留的关键字、操作符、标识符等)。此外,这些标记必须在它们各自的上下文中的表示相同(例如,标识符可能需要引用相同的变量)。

考虑下面的例子:

\begin{cpp}
// translation unit 1:
static int counter = 0;
inline void increaseCounter()
{
	++counter;
}

int main()
{ }

// translation unit 2:
static int counter = 0;
inline void increaseCounter()
{
	++counter;
}
\end{cpp}

这个例子是错误的,因为即使内联函数incrementounter()的标记序列在两个翻译单元中看起来相同,从而包含一个指向两个不同实体的标记计数器。不过,因为名为counter的两个变量具有内部链接(静态说明符),所以尽管名称相同,但不相关。即使没有实际使用内联函数,这也是一个错误。

\#included头文件中可以放置在多个翻译单元中的实体定义,无论何时需要定义,都要包含这些定义,确保在所有情况下标记序列都相同。

\begin{notice}
条件编译指令在不同的翻译单元中会有不同的计算结果,所以小心使用这些指令。其他更不常见差异也可能存在。
\end{notice}

使用这种方法,两个相同的标记引用不同事物的情况将变得相当罕见。发生时,产生的错误往往是隐秘的,难以查找。

交叉翻译单元约束不仅可以在多个地方定义的实体,还适用于声明中的默认参数。换句话说,下面的程序有未定义行为:

\begin{cpp}
// translation unit 1:
void unused(int = 3);
int main()
{ }

// translation unit 2:
void unused(int = 4);
\end{cpp}

这里,标记流的等价有时会涉及微妙的隐式影响。下面的例子是从C++标准中提取出来的(稍微修改了一下):

\begin{cpp}
// translation unit 1:
class X {
	public:
	X(int, int);
	X(int, int, int);
};

X::X(int, int = 0)
{ }

class D {
	X x = 0;
};

D d1; // X(int, int) called by D()

// translation unit 2:
class X {
	public:
	X(int, int);
	X(int, int, int);
};

X::X(int, int = 0, int = 0)
{ }

class D : public X {
	X x = 0;
};

D d2; // X(int, int, int) called by D()
\end{cpp}

本例中,出现问题是因为类D隐式生成的默认构造函数在两个翻译单元中不同。一个调用带两个参数的X构造函数,另一个调用带三个参数的X构造函数。如果有什么区别的话,就是将默认参数限制在程序中的一个位置(可能的话,这个位置应该在头文件中)的动机。幸运的是,在类外定义上放置默认参数是一种少见的做法。

对于相同的标记必须引用相同的实体规则,当然也有例外。如果相同的标记引用了具有相同值的不相关常量,并且没有使用结果表达式的地址(甚至通过将引用绑定到产生该常量的变量,也没有隐式地使用),那么这些标记等价。该异常允许的程序结构如下所示:

\begin{cpp}
// header.hpp:
#ifndef HEADER_HPP
#define HEADER_HPP

int const length = 10;

class MiniBuffer {
	char buf[length];
	...
};

#endif // HEADER_HPP
\end{cpp}

原则上,当头文件包含在两个不同的翻译单元中时,将创建两个名为length的不同常量。因为在此上下文中,const意味着static。然而,这种常量变量通常用于定义编译时常量值,而不是运行时的特定存储位置。因此,若不强制存在这样一个存储位置(通过引用变量的地址),两个常量有相同的值就够了。

最后,关于模板的说明。模板中的名称绑定分为两个阶段,非依赖名称在模板定义点绑定。对于这些规则,等价规则的处理类似于其他非模板定义。对于在实例化点绑定的名称,必须在该点应用等价规则,绑定必须等价。





