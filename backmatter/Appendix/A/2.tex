\section{声明和定义}
术语声明和定义在常见的“开发者对话”中经常交替使用。在ODR的范围内,这些词的含义很重要。

\begin{notice}
交换关于C和C++的观点时,仔细处理术语是一个好习惯。我们在整本书中都是这样做的。
\end{notice}

声明是一种C++构造,(通常)

\begin{notice}
有些结构(如static\_assert)不引入名称,但在语法上视为声明。
\end{notice}

程序中引入或重新引入一个名称。声明也可以是定义,这取决于它引入了哪个,以及如何引入:

\begin{itemize}
\item 
\textbf{命名空间和命名空间别名:}
命名空间及其别名的声明也需要定义,尽管术语定义在此上下文中并不常见,因为命名空间的成员列表可以在以后进行“扩展”(例如,与类和枚举类型不同)。

\item 
\textbf{类、类模板、函数、函数模板、成员函数和成员函数模板: }
当声明包含与名称相关联的大括号正文时,声明就是定义。该规则包括联合、操作符、成员操作符、静态成员函数、构造函数和析构函数,以及这些东西的模板版本的显式特化(即任何类实体和函数实体)。

\item 
\textbf{枚举:}
当声明包含用大括号括起来的枚举数列表时,该声明是定义。

\item 
\textbf{局部变量和非静态数据成员:}
这些可以视为定义(尽管区别很少),函数定义中的函数参数声明本身就是一个定义,因为它表示一个局部变量,但函数声明中的函数参数不是定义。

\item 
\textbf{全局变量:}
若声明之前没有使用关键字extern,或者有初始化式,那么全局变量的声明也是该变量的定义。否则,就不是定义。

\item 
\textbf{静态数据成员: }
当声明出现在成员的类或类模板外部,或者在类或类模板中内联或constexpr声明时,声明就是定义。

\item 
\textbf{显式和偏特化:}
若template<>或template<…>本身是一个定义,只是静态数据成员或静态数据成员模板的显式特化,只有在包含初始化式时才是定义。

\end{itemize}

其他声明不是定义。这包括类型别名(带有typedef或using)、using声明、using指令、模板参数声明、显式实例化指令、static\_assert声明等。






