\section{使用概念}

首先研究一下如何在用户代码中使用概念(即,定义模板的代码不必定义应用于模板参数的概念)。

\subsubsection{处理需求}

下面是我们的双参数max()模板,其有一个约束条件:

\begin{cpp}
template<typename T> requires LessThanComparable<T>
T max(T a, T b) {
	return b < a ? a : b;
}
\end{cpp}

增加了一条requires子句

\begin{cpp}
requires LessThanComparable<T>
\end{cpp}

假设之前已经进行了声明——通过头文件包含——LessThanComparable的概念。

这样的概念是布尔谓词(即产生bool类型值的表达式),其计算结果为常量表达式。因为约束在编译时计算,因此就生成的代码而言,不会产生多余的开销:这个受约束的模板生成的代码,与之前讨论的不受约束的版本性能一样。

当尝试使用该模板时,其不会实例化,直到对requires子句进行了计算,并生成了true。若产生的是false,则可能会产生一个错误,解释需求的哪一部分失败了(或者,可能会选择一个匹配的重载模板,但是没有满足需求)。

require子句不必用概念来表示(这样做是很好的实践,会产生更好的诊断信息):可以使用布尔常量表达式。如6.5节所讨论的,下面的代码确保模板构造函数不能用作复制构造函数:

\begin{cpp}
class Person
{
	private:
	std::string name;
	public:
	template<typename STR>
	requires std::is_convertible_v<STR,std::string>
	explicit Person(STR&& n)
	: name(std::forward<STR>(n)) {
		std::cout << "TMPL-CONSTR for ’" << name << "’\n";
	}
	...
};
\end{cpp}

因为特别的布尔表达式(在这种情况下使用类型特征),所以这里不使用命名概念(参见第E.2节)可能更合理

\begin{cpp}
std::is_convertible_v<STR,std::string>
\end{cpp}

用于修复可能使用的模板构造函数,而不是复制构造函数的问题。如何组织概念和约束的细节仍然是C++社区探索的领域,并且可能随着时间的推移而演变,但似乎有一个共识,即概念应该反映代码的含义,而不是它是否可以编译。

\subsubsection{处理多种需求}

上例中,只有一个requires,但是有多个requires的情况并不少见。可以想象一个描述元素值序列的Sequence概念(与标准中相同的概念相匹配)和一个模板find(),给定一个序列和一个值,返回一个指向该序列中第一个出现的值的迭代器(如果有的话)。该模板可以定义如下:

\begin{cpp}
template<typename Seq>
	requires Sequence<Seq> &&
			EqualityComparable<typename Seq::value_type>
	typename Seq::iterator find(Seq const& seq,
								typename Seq::value_type const& val)
{
	return std::find(seq.begin(), seq.end(), val);
}
\end{cpp}

对该模板的调用都将首先依次检查每个需求,并且只有当所有需求产生true时,才能为调用选择模板,并实例化模板(若重载解析不会因为其他原因丢弃模板)。

也可以使用||来表达“替代”需求。很少需要这样做,在require子句中过度使用||操作符可能会占用编译资源(使编译速度明显变慢)。在某些情况下,这会非常方便。例如:

\begin{cpp}
template<typename T>
	requires Integral<T> ||
			FloatingPoint<T>
T power(T b, T p);
\end{cpp}

单需求也可以涉及多个模板参数,而单概念可以表达多个模板参数上的谓词。例如:

\begin{cpp}
template<typename T, typename U>
	requires SomeConcept<T, U>
auto f(T x, U y) -> decltype(x+y)
\end{cpp}

因此,概念可以在类型参数之间建立关系。

\subsubsection{单个需求的快捷方式}

为了减少requires子句的符号开销,当约束只涉及一个参数时,可以使用一种语法快捷方式。可以通过使用上面max()模板声明的简写:

\begin{cpp}
template<LessThanComparable T>
T max(T a, T b) {
	return b < a ? a : b;
}
\end{cpp}

功能上等价于前面的max()定义。然而,当重新声明一个受约束的模板时,必须使用与原始声明的形式相同(从这个意义上说,它们只在功能上等价)。

可以在find()模板中对这两个需求之一使用相同的简写方式:

\begin{cpp}
template<Sequence Seq>
	requires EqualityComparable<typename Seq::value_type>
typename Seq::iterator find(Seq const& seq,
							typename Seq::value_type const& val)
{
	return std::find(seq.begin(), seq.end(), val);
}
\end{cpp}

这等价于前面为序列类型定义的find()模板。





