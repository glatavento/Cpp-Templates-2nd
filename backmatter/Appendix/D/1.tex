
使用类型特征时,必须包含头文件<type\_traits>:

\begin{cpp}
#include <type_traits>
\end{cpp}

然后,该用法取决于特征是否产生类型或值:

\begin{itemize}
\item 
对于产生类型的特征,使用如下方式访问该类型:
\begin{cpp}
typename std::trait<...>::type
std::trait_t<...> // since C++14
\end{cpp}

\item 
对于产生值的特征,可以通过如下方式访问该值:
\begin{cpp}
std::trait<...>::value
std::trait<...>() // implicit conversion to its type
std::trait_v<...> // since C++17
\end{cpp}
\end{itemize}

例如:

\filename{utils/traits1.cpp}
\begin{cpp}
#include <type_traits>
#include <iostream>

int main()
{
	int i = 42;
	std::add_const<int>::type c = i; // c is int const
	std::add_const_t<int> c14 = i; // since C++14
	static_assert(std::is_const<decltype(c)>::value, "c should be const");
	
	std::cout << std::boolalpha;
	std::cout << std::is_same<decltype(c), int const>::value // true
			<< ’\n’;
	std::cout << std::is_same_v<decltype(c), int const> // since C++17
			<< ’\n’;
	if (std::is_same<decltype(c), int const>{}) { // implicit conversion to bool
		std::cout << "same \n";
	}
}
\end{cpp}

参见第2.8节,了解traits的\_t版本的定义方式。关于特征的\_v版本的定义方式,请参见5.6节。

\subsection{D.1.1\hspace{0.2cm}std::integral\_constant和std::bool\_constant}

所有产生值的标准类型特征都派生自辅助类模板实例std::integral\_constant:

\begin{cpp}
namespace std {
	template<typename T, T val>
	struct integral_constant {
		static constexpr T value = val; // value of the trait
		using value_type = T; // type of the value
		using type = integral_constant<T,val>;
		constexpr operator value_type() const noexcept {
			return value;
		}
		constexpr value_type operator() () const noexcept { // since C++14
			return value;
		}
	};
}
\end{cpp}

就是说:

\begin{itemize}
\item 
可以使用value\_type成员来查询结果的类型。由于产生值的许多特征是谓词,value\_type通常是bool。

\item 
特征类型的对象具有隐式类型转换,转换为类型特征产生值的类型。

\item 
C++14(以及以后的版本)中,类型特征的对象也可为函数对象(函子),其中“函数调用”会产生值。

\item 
类型成员只生成底层integral\_constant实例。
\end{itemize}

若特征产生布尔值,也可以这样

\begin{cpp}
	namespace std {
		template<bool B>
		using bool_constant = integral_constant<bool, B>; // since C++17
		using true_type = bool_constant<true>;
		using false_type = bool_constant<false>;
	}
\end{cpp}

\begin{notice}C++17前,该标准不包括别名模板bool\_constant<>。std::true\_type和std::false\_type在C++11和C++14中确实存在,但分别指定为integral\_constant<bool,true>和integral\_constant<bool,false>。
\end{notice}

以便这些布尔特征在特定属性适用时继承std::true\_type,否则继承std::false\_type,所以它们对应的值成员等于true和false。为结果值设置不同的类型true和false,可以基于类型特征的结果进行标签调度(参见第19.3.3节和第20.2节)。

例如:

\filename{utils/traits2.cpp}
\begin{cpp}
#include <type_traits>
#include <iostream>

int main()
{
	using namespace std;
	cout << boolalpha;
	
	using MyType = int;
	cout << is_const<MyType>::value << ’\n’; // prints false
	
	using VT = is_const<MyType>::value_type; // bool
	using T = is_const<MyType>::type; // integral_constant<bool, false>
	cout << is_same<VT,bool>::value << ’\n’; // prints true
	cout << is_same<T, integral_constant<bool, false>>::value
		<< ’\n’; // prints true
	cout << is_same<T, bool_constant<false>>::value
		<< ’\n’; // prints true (not valid	
				// prior to C++17)
				
	auto ic = is_const<MyType>(); // object of trait type
	cout << is_same<decltype(ic), is_const<int>>::value << ’\n’; // true
	cout << ic() << ’\n’; // function call (prints false)
	
	static constexpr auto mytypeIsConst = is_const<MyType>{};
	if constexpr(mytypeIsConst) { // compile-time check since C++17 => false
		... // discarded statement
	}
	static_assert(!std::is_const<MyType>{}, "MyType should not be const");
}
\end{cpp}

各种元编程上下文中,具有不同的非布尔integral\_constant特化类型。参见第24.3节对类似类型CTValue的讨论,以及第25.6节对元组元素访问的使用。

\subsection{D.1.2\hspace{0.2cm}使用特征}

使用性格特征时需要注意以下几点:

\begin{itemize}
\item 
类型特征应用于类型,但decltype允许测试表达式、变量和函数的属性,decltype只在变量或函数的命名没有附加括号的情况下才会产生类型;对于任何其他表达式,生成的类型也反映表达式的类型类别。例如:

\begin{cpp}
void foo (std::string&& s)
{
	// check the type of s:
	std::is_lvalue_reference<decltype(s)>::value // false
	std::is_rvalue_reference<decltype(s)>::value // true, as declared
	// check the value category of s used as expression:
	std::is_lvalue_reference<decltype((s))>::value // true, s used as lvalue
	std::is_rvalue_reference<decltype((s))>::value // false
}
\end{cpp}

详见第15.10.2节。

\item 
对于菜鸟开发者来说,有些特征可能具有反直觉的行为。参见第11.2.1节中的示例。

\item 
有些特征是有要求或前提条件,违反这些前提条件会导致未定义行为。

\begin{notice}C++标准委员会考虑了C++17的提议,要求违反类型特征的前提条件会导致编译时错误。然而,因为一些类型特征目前的需求比严格的必要条件更强(比如总是需要完整的类型),所以这个改变推迟了。
\end{notice}

参见第11.2.1节中的一些示例。

\item 
许多特征需要完整的类型(参见第10.3.1节)。为了能够在不完整类型中使用,有时可以引入模板来延迟计算(详见第11.5节)。

\item 
有时逻辑操作符\&\&、||和!不能用于定义基于其他类型特征的新类型特征。此外,处理可能失败的特征可能会有问题,至少会有一些缺陷。出于这个原因,我们提供了一些特殊的特性,能够在逻辑上组合布尔特性。参见第D.6部分了解详细信息。

\item 
尽管标准别名模板(以\_t或\_v结尾),但也有缺点,并在某些元编程上下文中不可用。详见第19.7.3节。
\end{itemize}



