
\documentclass[11pt, onecolumn, oneside]{ctexbook}


\title{C++ Templates 2nd Edition} 
\author{David Vandevoorde \and Nicolai M. Josuttis \and Douglas Gregor \and 陈晓伟}
\date{\today}

\setCJKmainfont{HYXuanSong}[BoldFont=HYXuanSong 75S, ItalicFont=FZNewKai-Z03]
\setmainfont{Libertinus Serif}
\setsansfont{Libertinus Sans}
\setmonofont{Iosevka Fixed Slab}

\usepackage{tabularx,booktabs,graphicx,subfigure,hyperref,url,subfiles,enumitem,minted}

\usepackage[a5paper, scale=0.8]{geometry}
\usepackage[most]{tcolorbox}

\hypersetup{breaklinks=true}

\setenumerate{nosep}
\setitemize{nosep}
\setdescription{nosep}

\definecolor{mygreen}{rgb}{0,0.6,0}
\definecolor{mygray}{rgb}{0.5,0.5,0.5}
\definecolor{mymauve}{rgb}{0.58,0,0.82}
\definecolor{keywordcolor}{rgb}{0.8,0.1,0.5}
\definecolor{webgreen}{rgb}{0,.5,0}
\definecolor{bgcolor}{rgb}{0.92,0.92,0.92}

% URL 正确换行
% https://liam.page/2017/05/17/help-the-url-command-from-hyperref-to-break-at-line-wrapping-point/
\makeatletter
\def\UrlAlphabet{%
	\do\a\do\b\do\c\do\d\do\e\do\f\do\g\do\h\do\i\do\j%
	\do\k\do\l\do\m\do\n\do\o\do\p\do\q\do\r\do\s\do\t%
	\do\u\do\v\do\w\do\x\do\y\do\z\do\A\do\B\do\C\do\D%
	\do\E\do\F\do\G\do\H\do\I\do\J\do\K\do\L\do\M\do\N%
	\do\O\do\P\do\Q\do\R\do\S\do\T\do\U\do\V\do\W\do\X%
	\do\Y\do\Z}
\def\UrlDigits{\do\1\do\2\do\3\do\4\do\5\do\6\do\7\do\8\do\9\do\0}
\g@addto@macro{\UrlBreaks}{\UrlOrds}
\g@addto@macro{\UrlBreaks}{\UrlAlphabet}
\g@addto@macro{\UrlBreaks}{\UrlDigits}
\makeatother

\ExplSyntaxOn
\NewDocumentCommand { \filename } { m }
{ \noindent \textit { #1 } }

\setminted { obeytabs, tabsize=2, breaklines=true, fontsize=\footnotesize, frame=single }

\NewDocumentEnvironment { cpp } { }
{ 
  \VerbatimEnvironment
  \begin { minted } [ linenos=true ] { cpp } 
} 
{ \end   { minted } }

\NewDocumentEnvironment { shell } { }
{ 
  \VerbatimEnvironment
  \begin { minted } { bash } 
} 
{ \end   { minted } }

\NewDocumentEnvironment { notice } { }
{ 
  \begin { tcolorbox } [ colback = webgreen!5!white, colframe=webgreen!75!black ]
} 
{ \end   { tcolorbox } }

\ExplSyntaxOff

\begin{document}
	
	%前言
    \frontmatter
    \setcounter{secnumdepth}{0}

	\subfile{frontmatter/titlepage.tex}
	\subfile{frontmatter/preface.tex}
	\subfile{frontmatter/Acknowledgments-for-the-Second.tex}
	\subfile{frontmatter/Acknowledgments-for-the-First.tex}
	\subfile{frontmatter/About-This-Book.tex}
	
	\tableofcontents

    \mainmatter
    \setcounter{secnumdepth}{2}

	\subfile{content/part1.tex}

	
	% \color{white}
	% \section*{\zihao{1}第二部分:深入了解模板}
	% \pagecolor{mygray}
	% \addcontentsline{toc}{section}{第二部分:深入了解模板}
	% \textbf{\subfile{content/2/Section.tex}}
	% \newpage
	% \color{black}
	% \pagecolor{white}
	
	% \chapter{模板基础}
	% \addcontentsline{toc}{section}{第12章\hspace{0.5cm}模板基础}
	% \subfile{content/2/chapter12/0.tex}
	
	% \section{参数化的声明}
	% \addcontentsline{toc}{section}{12.1.\hspace{0.2cm}参数化的声明}
	% \subfile{content/2/chapter12/1.tex}
	
	% \section{模板形参}
	% \addcontentsline{toc}{section}{12.2.\hspace{0.2cm}模板形参}
	% \subfile{content/2/chapter12/2.tex}
	
	% \section{模板实参}
	% \addcontentsline{toc}{section}{12.3.\hspace{0.2cm}模板实参}
	% \subfile{content/2/chapter12/3.tex}
	
	% \section{可变参数模板}
	% \addcontentsline{toc}{section}{12.4.\hspace{0.2cm}可变参数模板}
	% \subfile{content/2/chapter12/4.tex}
	
	% \section{友元}
	% \addcontentsline{toc}{section}{12.5.\hspace{0.2cm}友元}
	% \subfile{content/2/chapter12/5.tex}
	
	% \section{后记}
	% \addcontentsline{toc}{section}{12.6.\hspace{0.2cm}后记}
	% \subfile{content/2/chapter12/6.tex}
	% \newpage
	
	% \chapter{模板中的名称}
	% \addcontentsline{toc}{section}{第13章\hspace{0.5cm}模板中的名称}
	% \subfile{content/2/chapter13/0.tex}
	
	% \section{名称的分类}
	% \addcontentsline{toc}{section}{13.1.\hspace{0.2cm}名称的分类}
	% \subfile{content/2/chapter13/1.tex}
	
	% \section{查找名称}
	% \addcontentsline{toc}{section}{13.2.\hspace{0.2cm}查找名称}
	% \subfile{content/2/chapter13/2.tex}
	
	% \section{解析模板}
	% \addcontentsline{toc}{section}{13.3.\hspace{0.2cm}解析模板}
	% \subfile{content/2/chapter13/3.tex}
	
	% \section{派生和类模板}
	% \addcontentsline{toc}{section}{13.4.\hspace{0.2cm}派生和类模板}
	% \subfile{content/2/chapter13/4.tex}
	
	% \section{后记}
	% \addcontentsline{toc}{section}{13.5.\hspace{0.2cm}后记}
	% \subfile{content/2/chapter13/5.tex}
	% \newpage
	
	% \chapter{实例化}
	% \addcontentsline{toc}{section}{第14章\hspace{0.5cm}实例化}
	% \subfile{content/2/chapter14/0.tex}
	
	% \section{按需实例化}
	% \addcontentsline{toc}{section}{14.1.\hspace{0.2cm}按需实例化}
	% \subfile{content/2/chapter14/1.tex}
	
	% \section{延迟实例化}
	% \addcontentsline{toc}{section}{14.2.\hspace{0.2cm}延迟实例化}
	% \subfile{content/2/chapter14/2.tex}
	
	% \section{C++实例化模型}
	% \addcontentsline{toc}{section}{14.3.\hspace{0.2cm}C++实例化模型}
	% \subfile{content/2/chapter14/3.tex}
	
	% \section{实现方案}
	% \addcontentsline{toc}{section}{14.4.\hspace{0.2cm}实现方案}
	% \subfile{content/2/chapter14/4.tex}
	
	% \section{显式实例化}
	% \addcontentsline{toc}{section}{14.5.\hspace{0.2cm}显式实例化}
	% \subfile{content/2/chapter14/5.tex}
	
	% \section{编译时if语句}
	% \addcontentsline{toc}{section}{14.6.\hspace{0.2cm}编译时if语句}
	% \subfile{content/2/chapter14/6.tex}
	
	% \section{标准库中的显式实例化}
	% \addcontentsline{toc}{section}{14.7.\hspace{0.2cm}标准库中的显式实例化}
	% \subfile{content/2/chapter14/7.tex}
	
	% \section{后记}
	% \addcontentsline{toc}{section}{14.8.\hspace{0.2cm}后记}
	% \subfile{content/2/chapter14/8.tex}
	% \newpage
	
	% \chapter{模板参数推导}
	% \addcontentsline{toc}{section}{第15章\hspace{0.5cm}模板参数推导}
	% \subfile{content/2/chapter15/0.tex}
	
	% \section{推导过程}
	% \addcontentsline{toc}{section}{15.1.\hspace{0.2cm}推导过程}
	% \subfile{content/2/chapter15/1.tex}
	
	% \section{推导上下文}
	% \addcontentsline{toc}{section}{15.2.\hspace{0.2cm}推导上下文}
	% \subfile{content/2/chapter15/2.tex}
	
	% \section{特殊的推导情况}
	% \addcontentsline{toc}{section}{15.3.\hspace{0.2cm}特殊的推导情况}
	% \subfile{content/2/chapter15/3.tex}
	
	% \section{初始化列表}
	% \addcontentsline{toc}{section}{15.4.\hspace{0.2cm}初始化列表}
	% \subfile{content/2/chapter15/4.tex}
	
	% \section{参数包}
	% \addcontentsline{toc}{section}{15.5.\hspace{0.2cm}参数包}
	% \subfile{content/2/chapter15/5.tex}
	
	% \section{右值引用}
	% \addcontentsline{toc}{section}{15.6.\hspace{0.2cm}右值引用}
	% \subfile{content/2/chapter15/6.tex}
	
	% \section{SFINAE(替换失败不为过)}
	% \addcontentsline{toc}{section}{15.7.\hspace{0.2cm}SFINAE(替换失败不为过)}
	% \subfile{content/2/chapter15/7.tex}
	
	% \section{推导的限制}
	% \addcontentsline{toc}{section}{15.8.\hspace{0.2cm}推导的限制}
	% \subfile{content/2/chapter15/8.tex}
	
	% \section{显式函数模板参数}
	% \addcontentsline{toc}{section}{15.9.\hspace{0.2cm}显式函数模板参数}
	% \subfile{content/2/chapter15/9.tex}
	
	% \section{初始化式和表达式的推导}
	% \addcontentsline{toc}{section}{15.10.\hspace{0.2cm}初始化式和表达式的推导}
	% \subfile{content/2/chapter15/10.tex}

	% \section{别名模板}
	% \addcontentsline{toc}{section}{15.11.\hspace{0.2cm}别名模板}
	% \subfile{content/2/chapter15/11.tex}
	
	% \section{类模板参数推导}
	% \addcontentsline{toc}{section}{15.12.\hspace{0.2cm}类模板参数推导}
	% \subfile{content/2/chapter15/12.tex}
	
	% \section{后记}
	% \addcontentsline{toc}{section}{15.13.\hspace{0.2cm}后记}
	% \subfile{content/2/chapter15/13.tex}
	% \newpage
	
	% \chapter{模板的特化和重载}
	% \addcontentsline{toc}{section}{第16章\hspace{0.5cm}模板的特化和重载}
	% \subfile{content/2/chapter16/0.tex}
	
	% \section{不完全契合的“泛型代码”}
	% \addcontentsline{toc}{section}{16.1.\hspace{0.2cm}不完全契合的“泛型代码”}
	% \subfile{content/2/chapter16/1.tex}
	
	% \section{重载函数模板}
	% \addcontentsline{toc}{section}{16.2.\hspace{0.2cm}重载函数模板}
	% \subfile{content/2/chapter16/2.tex}
	
	% \section{显式特化}
	% \addcontentsline{toc}{section}{16.3.\hspace{0.2cm}显式特化}
	% \subfile{content/2/chapter16/3.tex}
	
	% \section{类模板偏特化}
	% \addcontentsline{toc}{section}{16.4.\hspace{0.2cm}类模板偏特化}
	% \subfile{content/2/chapter16/4.tex}
	
	% \section{变量模板偏特化}
	% \addcontentsline{toc}{section}{16.5.\hspace{0.2cm}变量模板偏特化}
	% \subfile{content/2/chapter16/5.tex}
	
	% \section{后记}
	% \addcontentsline{toc}{section}{16.6.\hspace{0.2cm}后记}
	% \subfile{content/2/chapter16/6.tex}
	% \newpage
	
	% \chapter{未来的方向}
	% \addcontentsline{toc}{section}{第17章\hspace{0.5cm}未来的方向}
	% \subfile{content/2/chapter17/0.tex}
	
	% \section{放宽的typename规则}
	% \addcontentsline{toc}{section}{17.1.\hspace{0.2cm}放宽的typename规则}
	% \subfile{content/2/chapter17/1.tex}
	
	% \section{广义非类型模板参数}
	% \addcontentsline{toc}{section}{17.2.\hspace{0.2cm}广义非类型模板参数}
	% \subfile{content/2/chapter17/2.tex}
	
	% \section{函数模板的偏特化}
	% \addcontentsline{toc}{section}{17.3.\hspace{0.2cm}函数模板的偏特化}
	% \subfile{content/2/chapter17/3.tex}
	
	% \section{命名模板参数}
	% \addcontentsline{toc}{section}{17.4.\hspace{0.2cm}命名模板参数}
	% \subfile{content/2/chapter17/4.tex}
	
	% \section{重载类模板}
	% \addcontentsline{toc}{section}{17.5.\hspace{0.2cm}重载类模板}
	% \subfile{content/2/chapter17/5.tex}
	
	% \section{中间包扩展的推导}
	% \addcontentsline{toc}{section}{17.6.\hspace{0.2cm}中间包扩展的推导}
	% \subfile{content/2/chapter17/6.tex}
	
	% \section{void的规范化}
	% \addcontentsline{toc}{section}{17.7.\hspace{0.2cm}void的规范化}
	% \subfile{content/2/chapter17/7.tex}
	
	% \section{模板的类型检查}
	% \addcontentsline{toc}{section}{17.8.\hspace{0.2cm}模板的类型检查}
	% \subfile{content/2/chapter17/8.tex}
	
	% \section{反射元编程}
	% \addcontentsline{toc}{section}{17.9.\hspace{0.2cm}反射元编程}
	% \subfile{content/2/chapter17/9.tex}
	
	% \section{包管理工具}
	% \addcontentsline{toc}{section}{17.10.\hspace{0.2cm}包管理工具}
	% \subfile{content/2/chapter17/10.tex}
	
	% \section{模块}
	% \addcontentsline{toc}{section}{17.11.\hspace{0.2cm}模块}
	% \subfile{content/2/chapter17/11.tex}
	% \newpage
	
	% \color{white}
	% \section*{\zihao{1}第三部分:模板和设计}
	% \pagecolor{mygray}
	% \addcontentsline{toc}{section}{第三部分:模板和设计}
	% \textbf{\subfile{content/3/Section.tex}}
	% \newpage
	% \color{black}
	% \pagecolor{white}
	% \newpage
	
	% \chapter{模板的多态性}
	% \addcontentsline{toc}{section}{第18章\hspace{0.5cm}模板的多态性}
	% \subfile{content/3/chapter18/0.tex}
	
	% \section{动态多态性}
	% \addcontentsline{toc}{section}{18.1.\hspace{0.2cm}动态多态性}
	% \subfile{content/3/chapter18/1.tex}
	
	% \section{静态多态性}
	% \addcontentsline{toc}{section}{18.2.\hspace{0.2cm}静态多态性}
	% \subfile{content/3/chapter18/2.tex}
	
	% \section{动态多态性与静态多态性}
	% \addcontentsline{toc}{section}{18.3.\hspace{0.2cm}动态多态性与静态多态性}
	% \subfile{content/3/chapter18/3.tex}
	
	% \section{使用概念}
	% \addcontentsline{toc}{section}{18.4.\hspace{0.2cm}使用概念}
	% \subfile{content/3/chapter18/4.tex}
	
	% \section{设计模式的新形式}
	% \addcontentsline{toc}{section}{18.5.\hspace{0.2cm}设计模式的新形式}
	% \subfile{content/3/chapter18/5.tex}
	
	% \section{泛型编程}
	% \addcontentsline{toc}{section}{18.6.\hspace{0.2cm}泛型编程}
	% \subfile{content/3/chapter18/6.tex}
	
	% \section{后记}
	% \addcontentsline{toc}{section}{18.7.\hspace{0.2cm}后记}
	% \subfile{content/3/chapter18/7.tex}
	% \newpage
	
	% \chapter{特征的实现}
	% \addcontentsline{toc}{section}{第19章\hspace{0.5cm}特征的实现}
	% \subfile{content/3/chapter19/0.tex}
	
	% \section{实例:累加一个序列}
	% \addcontentsline{toc}{section}{19.1.\hspace{0.2cm}实例:累加一个序列}
	% \subfile{content/3/chapter19/1.tex}
	
	% \section{特征、策略和策略类}
	% \addcontentsline{toc}{section}{19.2.\hspace{0.2cm}特征、策略和策略类}
	% \subfile{content/3/chapter19/2.tex}
	
	% \section{类型函数}
	% \addcontentsline{toc}{section}{19.3.\hspace{0.2cm}类型函数}
	% \subfile{content/3/chapter19/3.tex}
	
	% \section{基于SFINAE的特征}
	% \addcontentsline{toc}{section}{19.4.\hspace{0.2cm}基于SFINAE的特征}
	% \subfile{content/3/chapter19/4.tex}
	
	% \section{IsConvertibleT}
	% \addcontentsline{toc}{section}{19.5.\hspace{0.2cm}IsConvertibleT}
	% \subfile{content/3/chapter19/5.tex}
	
	% \section{成员检查}
	% \addcontentsline{toc}{section}{19.6.\hspace{0.2cm}成员检查}
	% \subfile{content/3/chapter19/6.tex}
	
	% \section{其他方法}
	% \addcontentsline{toc}{section}{19.7.\hspace{0.2cm}其他方法}
	% \subfile{content/3/chapter19/7.tex}
	
	% \section{类型分类}
	% \addcontentsline{toc}{section}{19.8.\hspace{0.2cm}类型分类}
	% \subfile{content/3/chapter19/8.tex}
	
	% \section{策略特征}
	% \addcontentsline{toc}{section}{19.9.\hspace{0.2cm}策略特征}
	% \subfile{content/3/chapter19/9.tex}
	
	% \section{标准库}
	% \addcontentsline{toc}{section}{19.10.\hspace{0.2cm}标准库}
	% \subfile{content/3/chapter19/10.tex}
	
	% \section{后记}
	% \addcontentsline{toc}{section}{19.11.\hspace{0.2cm}后记}
	% \subfile{content/3/chapter19/11.tex}
	% \newpage
	
	% \chapter{类型属性上的重载}
	% \addcontentsline{toc}{section}{第20章\hspace{0.5cm}类型属性上的重载}
	% \subfile{content/3/chapter20/0.tex}
	
	% \section{算法特化}
	% \addcontentsline{toc}{section}{20.1.\hspace{0.2cm}算法特化}
	% \subfile{content/3/chapter20/1.tex}
	
	% \section{标签调度}
	% \addcontentsline{toc}{section}{20.2.\hspace{0.2cm}标签调度}
	% \subfile{content/3/chapter20/2.tex}
	
	% \section{启用/禁用函数模板}
	% \addcontentsline{toc}{section}{20.3.\hspace{0.2cm}启用/禁用函数模板}
	% \subfile{content/3/chapter20/3.tex}
	
	% \section{类的特化}
	% \addcontentsline{toc}{section}{20.4.\hspace{0.2cm}类的特化}
	% \subfile{content/3/chapter20/4.tex}
	
	% \section{实例化安全的模板}
	% \addcontentsline{toc}{section}{20.5.\hspace{0.2cm}实例化安全的模板}
	% \subfile{content/3/chapter20/5.tex}
	
	% \section{标准库}
	% \addcontentsline{toc}{section}{20.6.\hspace{0.2cm}标准库}
	% \subfile{content/3/chapter20/6.tex}
	
	% \section{后记}
	% \addcontentsline{toc}{section}{20.7.\hspace{0.2cm}后记}
	% \subfile{content/3/chapter20/7.tex}
	% \newpage
	
	% \chapter{模板和继承}
	% \addcontentsline{toc}{section}{第21章\hspace{0.5cm}模板和继承}
	% \subfile{content/3/chapter21/0.tex}
	
	% \section{空基类优化(EBCO)}
	% \addcontentsline{toc}{section}{21.1.\hspace{0.2cm}空基类优化(EBCO)}
	% \subfile{content/3/chapter21/1.tex}
	
	% \section{奇异递归模板模式(CRTP)}
	% \addcontentsline{toc}{section}{21.2.\hspace{0.2cm}奇异递归模板模式(CRTP)}
	% \subfile{content/3/chapter21/2.tex}
	
	% \section{混合类}
	% \addcontentsline{toc}{section}{21.3.\hspace{0.2cm}混合类}
	% \subfile{content/3/chapter21/3.tex}
	
	% \section{命名模板参数}
	% \addcontentsline{toc}{section}{21.4.\hspace{0.2cm}命名模板参数}
	% \subfile{content/3/chapter21/4.tex}
	
	% \section{后记}
	% \addcontentsline{toc}{section}{21.5.\hspace{0.2cm}后记}
	% \subfile{content/3/chapter21/5.tex}
	% \newpage
	
	% \chapter{桥接静态和动态多态性}
	% \addcontentsline{toc}{section}{第22章\hspace{0.5cm}桥接静态和动态多态性}
	% \subfile{content/3/chapter22/0.tex}
	
	% \section{函数对象、指针和std::function<>}
	% \addcontentsline{toc}{section}{22.1.\hspace{0.2cm}函数对象、指针和std::function<>}
	% \subfile{content/3/chapter22/1.tex}
	
	% \section{广义函数指针}
	% \addcontentsline{toc}{section}{22.2.\hspace{0.2cm}广义函数指针}
	% \subfile{content/3/chapter22/2.tex}
	
	% \section{桥接接口}
	% \addcontentsline{toc}{section}{22.3.\hspace{0.2cm}桥接接口}
	% \subfile{content/3/chapter22/3.tex}
	
	% \section{类型擦除}
	% \addcontentsline{toc}{section}{22.4.\hspace{0.2cm}类型擦除}
	% \subfile{content/3/chapter22/4.tex}
	
	% \section{可选桥接}
	% \addcontentsline{toc}{section}{22.5.\hspace{0.2cm}可选桥接}
	% \subfile{content/3/chapter22/5.tex}
	
	% \section{性能考量}
	% \addcontentsline{toc}{section}{22.6.\hspace{0.2cm}性能考量}
	% \subfile{content/3/chapter22/6.tex}
	
	% \section{后记}
	% \addcontentsline{toc}{section}{22.7.\hspace{0.2cm}后记}
	% \subfile{content/3/chapter22/7.tex}
	% \newpage
	
	% \chapter{元编程}
	% \addcontentsline{toc}{section}{第23章\hspace{0.5cm}元编程}
	% \subfile{content/3/chapter23/0.tex}
	
	% \section{现代C++的元编程}
	% \addcontentsline{toc}{section}{23.1.\hspace{0.2cm}现代C++的元编程}
	% \subfile{content/3/chapter23/1.tex}
	
	% \section{反射元编程的维数}
	% \addcontentsline{toc}{section}{23.2.\hspace{0.2cm}反射元编程的维数}
	% \subfile{content/3/chapter23/2.tex}
	
	% \section{递归实例化的代价}
	% \addcontentsline{toc}{section}{23.3.\hspace{0.2cm}递归实例化的代价}
	% \subfile{content/3/chapter23/3.tex}
	
	% \section{计算完整性}
	% \addcontentsline{toc}{section}{23.4.\hspace{0.2cm}计算完整性}
	% \subfile{content/3/chapter23/4.tex}
	
	% \section{递归实例化与递归模板参数}
	% \addcontentsline{toc}{section}{23.5.\hspace{0.2cm}递归实例化与递归模板参数}
	% \subfile{content/3/chapter23/5.tex}
	
	% \section{枚举值与静态常量}
	% \addcontentsline{toc}{section}{23.6.\hspace{0.2cm}枚举值与静态常量}
	% \subfile{content/3/chapter23/6.tex}
	
	% \section{后记}
	% \addcontentsline{toc}{section}{23.7.\hspace{0.2cm}后记}
	% \subfile{content/3/chapter23/7.tex}
	% \newpage
	
	% \chapter{类型列表}
	% \addcontentsline{toc}{section}{第24章\hspace{0.5cm}类型列表}
	% \subfile{content/3/chapter24/0.tex}
	
	% \section{解析}
	% \addcontentsline{toc}{section}{24.1.\hspace{0.2cm}解析}
	% \subfile{content/3/chapter24/1.tex}
	
	% \section{算法}
	% \addcontentsline{toc}{section}{24.2.\hspace{0.2cm}算法}
	% \subfile{content/3/chapter24/2.tex}
	
	% \section{非类型}
	% \addcontentsline{toc}{section}{24.3.\hspace{0.2cm}非类型}
	% \subfile{content/3/chapter24/3.tex}
	
	% \section{包扩展优化算法}
	% \addcontentsline{toc}{section}{24.4.\hspace{0.2cm}包扩展优化算法}
	% \subfile{content/3/chapter24/4.tex}
	
	% \section{Cons风格}
	% \addcontentsline{toc}{section}{24.5.\hspace{0.2cm}Cons风格}
	% \subfile{content/3/chapter24/5.tex}
	
	% \section{后记}
	% \addcontentsline{toc}{section}{24.6.\hspace{0.2cm}后记}
	% \subfile{content/3/chapter24/6.tex}
	% \newpage
	
	% \chapter{元组}
	% \addcontentsline{toc}{section}{第25章\hspace{0.5cm}元组}
	% \subfile{content/3/chapter25/0.tex}
	
	% \section{类型设计}
	% \addcontentsline{toc}{section}{25.1.\hspace{0.2cm}类型设计}
	% \subfile{content/3/chapter25/1.tex}
	
	% \section{基础操作}
	% \addcontentsline{toc}{section}{25.2.\hspace{0.2cm}基础操作}
	% \subfile{content/3/chapter25/2.tex}
	
	% \section{算法}
	% \addcontentsline{toc}{section}{25.3.\hspace{0.2cm}算法}
	% \subfile{content/3/chapter25/3.tex}
	
	% \section{扩展}
	% \addcontentsline{toc}{section}{25.4.\hspace{0.2cm}扩展}
	% \subfile{content/3/chapter25/4.tex}
	
	% \section{优化}
	% \addcontentsline{toc}{section}{25.5.\hspace{0.2cm}优化}
	% \subfile{content/3/chapter25/5.tex}
	
	% \section{下标}
	% \addcontentsline{toc}{section}{25.6.\hspace{0.2cm}下标}
	% \subfile{content/3/chapter25/6.tex}
	
	% \section{后记}
	% \addcontentsline{toc}{section}{25.7.\hspace{0.2cm}后记}
	% \subfile{content/3/chapter25/7.tex}
	% \newpage
	
	% \chapter{可辨识联合}
	% \addcontentsline{toc}{section}{第26章\hspace{0.5cm}可辨识联合}
	% \subfile{content/3/chapter26/0.tex}
	
	% \section{存储}
	% \addcontentsline{toc}{section}{26.1.\hspace{0.2cm}存储}
	% \subfile{content/3/chapter26/1.tex}
	
	% \section{设计}
	% \addcontentsline{toc}{section}{26.2.\hspace{0.2cm}设计}
	% \subfile{content/3/chapter26/2.tex}
	
	% \section{值的查询与提取}
	% \addcontentsline{toc}{section}{26.3.\hspace{0.2cm}值的查询与提取}
	% \subfile{content/3/chapter26/3.tex}
	
	% \section{元素初始化、赋值和销毁}
	% \addcontentsline{toc}{section}{26.4.\hspace{0.2cm}元素初始化、赋值和销毁}
	% \subfile{content/3/chapter26/4.tex}
	
	% \section{访问}
	% \addcontentsline{toc}{section}{26.5.\hspace{0.2cm}访问}
	% \subfile{content/3/chapter26/5.tex}
	
	% \section{变量初始化赋值}
	% \addcontentsline{toc}{section}{26.6.\hspace{0.2cm}变量初始化赋值}
	% \subfile{content/3/chapter26/6.tex}
	
	% \section{后记}
	% \addcontentsline{toc}{section}{26.7.\hspace{0.2cm}后记}
	% \subfile{content/3/chapter26/7.tex}
	% \newpage
	
	% \chapter{表达式模板}
	% \addcontentsline{toc}{section}{第27章\hspace{0.5cm}表达式模板}
	% \subfile{content/3/chapter27/0.tex}
	
	% \section{临时变量和分割循环}
	% \addcontentsline{toc}{section}{27.1.\hspace{0.2cm}临时变量和分割循环}
	% \subfile{content/3/chapter27/1.tex}
	
	% \section{模板参数中的编码表达式}
	% \addcontentsline{toc}{section}{27.2.\hspace{0.2cm}模板参数中的编码表达式}
	% \subfile{content/3/chapter27/2.tex}
	
	% \section{表达式模板的性能与约束}
	% \addcontentsline{toc}{section}{27.3.\hspace{0.2cm}表达式模板的性能与约束}
	% \subfile{content/3/chapter27/3.tex}
	
	% \section{后记}
	% \addcontentsline{toc}{section}{27.4.\hspace{0.2cm}后记}
	% \subfile{content/3/chapter27/4.tex}
	% \newpage
	
	% \chapter{调试模板}
	% \addcontentsline{toc}{section}{第28章\hspace{0.5cm}调试模板}
	% \subfile{content/3/chapter28/0.tex}
	
	% \section{浅式实例化}
	% \addcontentsline{toc}{section}{28.1.\hspace{0.2cm}浅式实例化}
	% \subfile{content/3/chapter28/1.tex}
	
	% \section{静态断言}
	% \addcontentsline{toc}{section}{28.2.\hspace{0.2cm}静态断言}
	% \subfile{content/3/chapter28/2.tex}
	
	% \section{原型}
	% \addcontentsline{toc}{section}{28.3.\hspace{0.2cm}原型}
	% \subfile{content/3/chapter28/3.tex}
	
	% \section{跟踪}
	% \addcontentsline{toc}{section}{28.4.\hspace{0.2cm}跟踪}
	% \subfile{content/3/chapter28/4.tex}
	
	% \section{动态分析}
	% \addcontentsline{toc}{section}{28.5.\hspace{0.2cm}动态分析}
	% \subfile{content/3/chapter28/5.tex}
	
	% \section{后记}
	% \addcontentsline{toc}{section}{28.6.\hspace{0.2cm}后记}
	% \subfile{content/3/chapter28/6.tex}
	% \newpage
	
	% \section*{\zihao{2} 附录A:定义原则}
	% \addcontentsline{toc}{section}{附录A:定义原则}
	% \subfile{content/Appendix/A/0.tex}
	
	% \section{翻译单元}
	% \addcontentsline{toc}{section}{A.1\hspace{0.2cm}翻译单元}
	% \subfile{content/Appendix/A/1.tex}
	
	% \section{声明和定义}
	% \addcontentsline{toc}{section}{A.2\hspace{0.2cm}声明和定义}
	% \subfile{content/Appendix/A/2.tex}
	
	% \section{定义原则的细节}
	% \addcontentsline{toc}{section}{A.3\hspace{0.2cm}定义原则的细节}
	% \subfile{content/Appendix/A/3.tex}
	% \newpage
	
	% \section*{\zihao{2} 附录B:值类别}
	% \addcontentsline{toc}{section}{附录B:值类别}
	% \subfile{content/Appendix/B/0.tex}
	
	% \section{传统的左值和右值}
	% \addcontentsline{toc}{section}{B.1\hspace{0.2cm}传统的左值和右值}
	% \subfile{content/Appendix/B/1.tex}
	
	% \section{C++11的值类别}
	% \addcontentsline{toc}{section}{B.2\hspace{0.2cm}C++11的值类别}
	% \subfile{content/Appendix/B/2.tex}
	
	% \section{使用decltype检查值类别}
	% \addcontentsline{toc}{section}{B.3\hspace{0.2cm}使用decltype检查值类别}
	% \subfile{content/Appendix/B/3.tex}
	
	% \section{引用类别}
	% \addcontentsline{toc}{section}{B.4\hspace{0.2cm}引用类别}
	% \subfile{content/Appendix/B/4.tex}
	% \newpage
	
	% \section*{\zihao{2} 附录C:重载解析}
	% \addcontentsline{toc}{section}{附录C:重载解析}
	% \subfile{content/Appendix/C/0.tex}
	
	% \section{何时应用重载解析}
	% \addcontentsline{toc}{section}{C.1\hspace{0.2cm}何时应用重载解析}
	% \subfile{content/Appendix/C/1.tex}
	
	% \section{简化的重载解析}
	% \addcontentsline{toc}{section}{C.2\hspace{0.2cm}简化的重载解析}
	% \subfile{content/Appendix/C/2.tex}
	
	% \section{重载的细节}
	% \addcontentsline{toc}{section}{C.3\hspace{0.2cm}重载的细节}
	% \subfile{content/Appendix/C/3.tex}
	% \newpage
	
	% \section*{\zihao{2} 附录D:标准类型的使用}
	% \addcontentsline{toc}{section}{附录D:标准类型的使用}
	% \subfile{content/Appendix/D/0.tex}
	
	% \section{使用类型特征}
	% \addcontentsline{toc}{section}{D.1\hspace{0.2cm}使用类型特征}
	% \subfile{content/Appendix/D/1.tex}
	
	% \section{主要类型和复合类型}
	% \addcontentsline{toc}{section}{D.2\hspace{0.2cm}主要类型和复合类型}
	% \subfile{content/Appendix/D/2.tex}
	
	% \section{类型属性和操作}
	% \addcontentsline{toc}{section}{D.3\hspace{0.2cm}类型属性和操作}
	% \subfile{content/Appendix/D/3.tex}
	
	% \section{类型结构}
	% \addcontentsline{toc}{section}{D.4\hspace{0.2cm}类型结构}
	% \subfile{content/Appendix/D/4.tex}
	
	% \section{其他特性}
	% \addcontentsline{toc}{section}{D.5\hspace{0.2cm}其他特性}
	% \subfile{content/Appendix/D/5.tex}
	
	% \section{组合类型特性}
	% \addcontentsline{toc}{section}{D.6\hspace{0.2cm}组合类型特性}
	% \subfile{content/Appendix/D/6.tex}
	
	% \section{其他应用方式}
	% \addcontentsline{toc}{section}{D.7\hspace{0.2cm}其他应用方式}
	% \subfile{content/Appendix/D/7.tex}
	% \newpage
	
	% \section*{\zihao{2} 附录E:概念}
	% \addcontentsline{toc}{section}{附录E:概念}
	% \subfile{content/Appendix/E/0.tex}
	
	% \section{使用概念}
	% \addcontentsline{toc}{section}{E.1\hspace{0.2cm}使用概念}
	% \subfile{content/Appendix/E/1.tex}
	
	% \section{定义概念}
	% \addcontentsline{toc}{section}{E.2\hspace{0.2cm}定义概念}
	% \subfile{content/Appendix/E/2.tex}
	
	% \section{重载约束}
	% \addcontentsline{toc}{section}{E.3\hspace{0.2cm}重载约束}
	% \subfile{content/Appendix/E/3.tex}
	
	% \section{实用技巧}
	% \addcontentsline{toc}{section}{E.4\hspace{0.2cm}实用技巧}
	% \subfile{content/Appendix/E/4.tex}
	% \newpage

	% \section*{\zihao{2} 文献表}
	% \addcontentsline{toc}{section}{文献表}
	% \subfile{content/Bibliography.tex}
	% \newpage
	
	% \section*{\zihao{2} 术语表}
	% \addcontentsline{toc}{section}{术语表}
	% \subfile{content/Glossary.tex}

\end{document}

