\section*{本书概述}

模板是C++中强大的特性,而对模板的误解并未随着C++语言和开发社区的发展而消弭,从而无法使其无法发挥全力。本书的三位作者,作为C++专家,展示了如何使用现代模板来构建干净、快捷、高效、容易维护的软件。

第二版对C++11、C++14和C++17标准进行了更新,对改进模板或与模板交互的特性进行了解释,包括可变参数模板、泛型Lambda、类模板参数推导、编译时if、转发引用和用户定义文字。还深入研究了一些基本的语言概念(比如值类别),并包含了所有标准类型特征。

本书从基本概念和相关语言特征开始,其余作为参考。先关注语言本书,再是编码、高级应用和复杂的惯用法。过程中,示例清楚地说明了抽象概念,并演示了模板的最佳实践。

\section*{关键特性}
\begin{itemize}
  \item 准确理解模板的行为,避免陷阱
  \item 使用模板编写有效、灵活、可维护的软件
  \item 掌握有效的习语和技巧
  \item 保持性能或安全的情况下重用源码
  \item C++标准库中的泛型编程
  \item 预览“概念"特性
\end{itemize}

\section*{作者简介}

\begin{itemize}
  \item \textbf{David Vandevoorde}在20世纪80年代后期开始用C++编程。从伦斯勒理工学院获得博士学位后,成为惠普C++编译器团队的技术负责人。1999年,加入了爱迪生设计集团(EDG),该集团的C++编译器技术是业界领先的。他是C++标准委员会的活跃成员,也是comp.lang.c++新闻组的主持人(参与创办)。也是《C++ Solutions》的作者,该书是《C++ Programming Language, 3rd Edition》的配套书籍。

  \item \textbf{Nicolai M. Josuttis}因其畅销书籍《The C++ Standard Library - A Tutorial and Reference》闻名于世,是一名独立技术顾问,为电信、交通、金融和制造业设计面向对象的软件。也是C++标准委员会的活跃成员,也是System Bauhaus的合伙人,System Bauhaus是一个由面向对象系统开发专家组成的德国团体。Josuttis还写过其他几本关于面向对象编程和C++的书。

  \item \textbf{Douglas Gregor}是苹果公司的高级Swift/C++/Objective-C编译工程师,拥有伦斯勒理工学院的计算机科学博士学位,并在印第安纳大学从事博士后工作。
\end{itemize}


\section*{本书相关}
\begin{itemize}
  \item Github地址:\\\url{https://github.com/xiaoweiChen/Cpp-Templates-2nd}
\end{itemize}