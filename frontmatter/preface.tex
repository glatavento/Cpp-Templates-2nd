\chapter{前言}

C++模板在1990年就在“带注释的C++参考手册”(ARM;参见[ellisstroustrparm])中出现了,至今已经30多年了。在此之前,有更专业的书籍中对其详细论述过。十多年后,我们发现对于这个神秘、复杂、强大的C++特性,相应的基本概念介绍和高级技术文献居然少的可怜。本书的第一版中,我们想要解决这个问题,并决定写一本关于模板的书(有点不够谦逊)。

自2002年底发布第一版本后,C++发生了很大的变化。C++标准增加了很多新特性,C++社区不断创新也有助于展示基于模板的新编程技术。因此,本书的第二版保留了与第一版相同的目标,但使用的是“现代C++”。

我们以不同背景和不同目的来完成写书的任务。David(又名“Daveed”)是一位经验丰富的编译器作者,也是发展核心语言的C++标准委员会工作组的积极参与者,他对模板的所有功能(和问题)的精确和详细描述很感兴趣。Nico是一名“普通”的程序员,C++标准委员会库工作组的成员,他对模板的技术很感兴趣。Doug是一名模板库开发人员,成为编译器作者和语言设计师后,对收集、分类和评估用于构建模板库的技术很有兴趣。此外,我们希望与读者和整个社区积极分享这些知识,以减少误解、困惑或忧虑。

因此,会有示例概念介绍和模板行为的描述。从模板的原则开始,逐步发展到“模板的艺术”,将展现(或重新了解)静态多态性、类型特征、元编程和表达式模板等编程方式。还会更深入地了解C++标准库,其中所有的代码都会涉及模板。

写这本书的过程中,我们也学到了很多,也获得了很多快乐。希望在阅读本书时,您也能享受这本书带给您的快乐。